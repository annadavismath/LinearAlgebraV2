\documentclass{ximera}
%% You can put user macros here
%% However, you cannot make new environments

\listfiles

\graphicspath{
{./}
{./LTR-0070/}
{./VEC-0060/}
{./APP-0020/}
}

\usepackage{tikz}
\usepackage{tkz-euclide}
\usepackage{tikz-3dplot}
\usepackage{tikz-cd}
\usetikzlibrary{shapes.geometric}
\usetikzlibrary{arrows}
%\usetkzobj{all}
\pgfplotsset{compat=1.13} % prevents compile error.

%\renewcommand{\vec}[1]{\mathbf{#1}}
\renewcommand{\vec}{\mathbf}
\newcommand{\RR}{\mathbb{R}}
\newcommand{\dfn}{\textit}
\newcommand{\dotp}{\cdot}
\newcommand{\id}{\text{id}}
\newcommand\norm[1]{\left\lVert#1\right\rVert}
 
\newtheorem{general}{Generalization}
\newtheorem{initprob}{Exploration Problem}

\tikzstyle geometryDiagrams=[ultra thick,color=blue!50!black]

%\DefineVerbatimEnvironment{octave}{Verbatim}{numbers=left,frame=lines,label=Octave,labelposition=topline}



\usepackage{mathtools}


\title{Index} \license{CC BY-NC-SA 4.0}



\begin{document}
\begin{abstract}
\end{abstract}
\maketitle



\section{Index}
\subsection{A}
\href{https://ximera.osu.edu/oerlinalg/LinearAlgebra/VEC-0030/main}{Addition of Vectors}
\subsection{B}

\subsection{C}

\subsection{D}
Distance in $\RR^n$
\begin{expandable}
\begin{formula}
Let $A(a_1, a_2,\ldots ,a_n)$ and $B(b_1, b_2,\ldots ,b_n)$ be points in $\RR^n$.  The distance between $A$ and $B$ is given by
$$AB=\sqrt{(a_1-b_1)^2+(a_2-b_2)^2+\ldots +(a_n-b_n)^2}$$
\end{formula}
\ref{form:distRn}
\end{expandable}

\subsection{E}

\subsection{F}

\subsection{G}

\subsection{H}
Head - Tail Formula
\begin{expandable}
    \begin{formula}
  [``Head - Tail'' Formula in $\RR^n$]
Suppose a vector's tail is at point $A(a_1, a_2, \ldots ,a_n)$ and the vector's head is at $B(b_1, b_2, \ldots ,b_n)$, then 
\begin{equation*}
\overrightarrow{AB}=\begin{bmatrix}b_1-a_1\\ b_2-a_2\\ \vdots \\b_n-a_n\end{bmatrix}
\end{equation*}
\end{formula}
\ref{form:headminustailrn}
\end{expandable}

\subsection{I}

\subsection{J}

\subsection{K}

\subsection{L}

\subsection{M}
Magnitude of a Vector
\begin{expandable}
    \begin{definition}
Let $\vec{v}=\begin{bmatrix}v_1\\ v_2\\ \vdots \\v_n\end{bmatrix}$ be a vector in $\RR^n$, then the \dfn{length}, or the \dfn{magnitude}, of $\vec{v}$ is given by
\begin{equation*}  
\norm{\vec{v}}=\sqrt{v_1^2+v_2^2+\ldots +v_n^2}
\end{equation*}
\end{definition}
\ref{def:normrn}
\end{expandable}
    


\subsection{N}

\subsection{O}

\subsection{P}

\subsection{Q}

\subsection{R}

\subsection{S}
\href{https://ximera.osu.edu/oerlinalg/LinearAlgebra/VEC-0010/main}{Scalar} 

Standard Unit Vectors
\begin{expandable}
    \begin{definition}
  Let $\vec{e}_i$ denote a vector that has $1$ as the $i^{th}$ component and zeros elsewhere.  In other words, $$\vec{e}_i=\begin{bmatrix}
0\\
0\\
\vdots\\
1\\
\vdots\\
0
\end{bmatrix}$$ 
  where $1$ is in the $i^{th}$ position.  We say that  $\vec{e}_i$ is a \dfn{standard unit vector of $\RR^n$}.
\end{definition}
\ref{def:standardunitvecrn} 
\end{expandable}

\href{https://ximera.osu.edu/oerlinalg/LinearAlgebra/VEC-0030/main}{Subtraction of vectors}

\href{https://ximera.osu.edu/oerlinalg/LinearAlgebra/VEC-0010/main}{Standard Position}

\subsection{T}

\subsection{U}

\subsection{V}
\href{https://ximera.osu.edu/oerlinalg/LinearAlgebra/VEC-0010/main}{Vector}

\subsection{W}

\subsection{X}

\subsection{Y}

\subsection{Z}

\end{document}
