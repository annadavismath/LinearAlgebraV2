\documentclass{ximera}
%% You can put user macros here
%% However, you cannot make new environments

\listfiles

\graphicspath{
{./}
{./LTR-0070/}
{./VEC-0060/}
{./APP-0020/}
}

\usepackage{tikz}
\usepackage{tkz-euclide}
\usepackage{tikz-3dplot}
\usepackage{tikz-cd}
\usetikzlibrary{shapes.geometric}
\usetikzlibrary{arrows}
%\usetkzobj{all}
\pgfplotsset{compat=1.13} % prevents compile error.

%\renewcommand{\vec}[1]{\mathbf{#1}}
\renewcommand{\vec}{\mathbf}
\newcommand{\RR}{\mathbb{R}}
\newcommand{\dfn}{\textit}
\newcommand{\dotp}{\cdot}
\newcommand{\id}{\text{id}}
\newcommand\norm[1]{\left\lVert#1\right\rVert}
 
\newtheorem{general}{Generalization}
\newtheorem{initprob}{Exploration Problem}

\tikzstyle geometryDiagrams=[ultra thick,color=blue!50!black]

%\DefineVerbatimEnvironment{octave}{Verbatim}{numbers=left,frame=lines,label=Octave,labelposition=topline}



\usepackage{mathtools}


\title{INDEX} \license{CC BY-NC-SA 4.0}



\begin{document}
\begin{abstract}
\end{abstract}
\maketitle



\section{INDEX}
A hyperlinked term will take you to the section where the term is defined.  Parenthetical hyperlink will take you to the specific definition of formula in the section.  Use arrows on the right to display the definition or formula in the index.
\subsection{A}
\href{https://ximera.osu.edu/oerlinalg/LinearAlgebra/VEC-0030/main}{addition of vectors}

adjoint of a matrix

algebraic multiplicity of an eigenvalue

\href{https://ximera.osu.edu/oerlinalg/LinearAlgebra/SYS-0020/main}{augmented matrix}


\subsection{B}

\href{https://ximera.osu.edu/oerlinalg/LinearAlgebra/SYS-0020/main}{back substitution}

basis

Basis Theorem

\subsection{C}

change-of-basis matrix

characteristic equation

characteristic polynomial

\href{https://ximera.osu.edu/oerlinalg/LinearAlgebra/SYS-0020/main}{coefficient matrix}

cofactor expansion

cofactor of a square matrix %see 7.6 LaPlace Expansion

\href{https://ximera.osu.edu/oerlinalg/LinearAlgebra/MAT-0010/main}{column matrix (vector)}

column space of a matrix

composition of linear transformations

\href{https://ximera.osu.edu/oerlinalg/LinearAlgebra/SYS-0010/main}{consistent system}

convergence

coordinate vector with respect to a basis

Cramer's Rule

cross product (\ref{def:crossproduct})
\begin{expandable}
    Let $\vec{u=\begin{bmatrix}u_1\\u_2\\u_3\end{bmatrix}}$ and $\vec{v}=\begin{bmatrix}v_1\\v_2\\v_3\end{bmatrix}$ be vectors in $\RR^3$.  The \dfn{cross product} of $\vec{u}$ and $\vec{v}$, denoted by $\vec{u}\times\vec{v}$, is given by
$$\vec{u}\times\vec{v}=(u_2v_3-u_3v_2)\vec{i}-(u_1v_3-u_3v_1)\vec{j}+(u_1v_2-u_2v_1)\vec{k}
=\begin{bmatrix}u_2v_3-u_3v_2\\-u_1v_3+u_3v_1\\u_1v_2-u_2v_1\end{bmatrix}$$
\end{expandable}

\subsection{D}
determinant ($2\times 2$) (\ref{def:twodetcrossprod})
\begin{expandable}
    A $2\times 2$ \dfn{determinant} is a number associated with a $2\times 2$ matrix

$$\det{\begin{bmatrix}
a & b\\
c & d
\end{bmatrix}}=\begin{vmatrix}
a & b\\
c & d
\end{vmatrix} =ad-bc$$
\end{expandable}

determinant ($3\times 3$) (\ref{def:threedetcrossprod})
\begin{expandable}
    A $3\times 3$ \dfn{determinant} is a number associated with a $3\times 3$ matrix
$$\det{\begin{bmatrix}
a_1 & a_2 & a_3\\
b_1 & b_2 &b_3\\
c_1 &c_2 &c_3
\end{bmatrix}}=
\begin{vmatrix}
a_1 & a_2 & a_3\\
b_1 & b_2 &b_3\\
c_1 &c_2 &c_3
\end{vmatrix} =a_1
\begin{vmatrix}
b_2 & b_3\\
c_2 & c_3
\end{vmatrix} -a_2
\begin{vmatrix}
b_1 & b_3\\
c_1 & c_3
\end{vmatrix} +a_3
\begin{vmatrix}
b_1 & b_2\\
c_1 & c_2
\end{vmatrix}
$$
\end{expandable}

diagonal matrix

diagonalizable matrix

dimension

\href{https://ximera.osu.edu/oerlinalg/LinearAlgebra/RRN-0020/main}{Direction vector}

Distance between points in $\RR^n$ (\ref{form:distRn})
\begin{expandable}
%\begin{formula}
Let $A(a_1, a_2,\ldots ,a_n)$ and $B(b_1, b_2,\ldots ,b_n)$ be points in $\RR^n$.  The distance between $A$ and $B$ is given by
$$AB=\sqrt{(a_1-b_1)^2+(a_2-b_2)^2+\ldots +(a_n-b_n)^2}$$
%\end{formula}
\end{expandable}

\href{https://ximera.osu.edu/oerlinalg/LinearAlgebra/VEC-0070/main}{Distance between point and line}

divergence

dot product (\ref{def:dotproduct})
\begin{expandable}
 %   \begin{definition}
  Let $\vec{u}$ and $\vec{v}$ be vectors in $\RR^n$.  The \dfn{dot
    product} of $\vec{u}$ and $\vec{v}$, denoted by
  $\vec{u}\dotp \vec{v}$, is given by
$$\vec{u}\dotp\vec{v}=\begin{bmatrix}u_1\\u_2\\\vdots\\u_n\end{bmatrix}\dotp\begin{bmatrix}v_1\\v_2\\\vdots\\v_n\end{bmatrix}=u_1v_1+u_2v_2+\ldots+u_nv_n$$
%\end{definition}
\end{expandable}


\subsection{E}

eigenspace

eigenvalue

eigenvector

elementary matrix

elementary row operations (\ref{def:elemrowops})
\begin{expandable}
    The following three operations performed on a linear system are called \dfn{elementary row operations}.
\begin{enumerate}
\item Switching the order of equations (rows) $i$ and $j$:
$$R_i\leftrightarrow R_j$$
\item Multiplying both sides of equation (row) $i$ by the same non-zero constant, $k$, and replacing equation $i$ with the result:
$$kR_i\rightarrow R_i$$
\item Adding $k$ times equation (row) $i$ to equation (row) $j$, and replacing equation $j$ with the result:
$$R_j+kR_i\rightarrow R_j$$
\end{enumerate}
\end{expandable}

equivalent linear systems (\ref{def:equivsystems})
\begin{expandable}
    Two systems of linear equations are said to be equivalent if they have the same solution set.
\end{expandable}

Euclidean norm


\subsection{F}

\href{https://ximera.osu.edu/oerlinalg/LinearAlgebra/SYS-0020/main}{free variable}

fundamental subspaces of a matrix


\subsection{G}

Gaussian elimination (\ref{def:GaussianElimination})
\begin{expandable}
    The process of using the elementary row operations on a matrix to transform it into row-echelon form is called \dfn{Gaussian Elimination}.
\end{expandable}

Gauss-Jordan elimination (\ref{def:GaussJordanElimination})
\begin{expandable}
    The process of using the elementary row operations on a matrix to transform it into reduced row-echelon form is called \dfn{Gauss-Jordan elimination}.
\end{expandable}

\href{https://ximera.osu.edu/oerlinalg/LinearAlgebra/SYS-0040/main}{Gauss-Seidel method}

geometric multiplicity of an eigenvalue

Gerschgorin's Disk Theorem

Gram-Schmidt Process


\subsection{H}
Head - Tail Formula (\ref{form:headminustailrn})
\begin{expandable}
%    \begin{formula}
  [``Head - Tail'' Formula in $\RR^n$]
Suppose a vector's tail is at point $A(a_1, a_2, \ldots ,a_n)$ and the vector's head is at $B(b_1, b_2, \ldots ,b_n)$, then 
$$\overrightarrow{AB}=\begin{bmatrix}b_1-a_1\\ b_2-a_2\\ \vdots \\b_n-a_n\end{bmatrix}$$

%\end{formula}
\end{expandable}

homogeneous system %we can skip this if we want

\href{https://ximera.osu.edu/oerlinalg/LinearAlgebra/RRN-0030/main}{hyperplane}

\subsection{I}

identity matrix

\href{https://ximera.osu.edu/oerlinalg/LinearAlgebra/SYS-0010/main}{inconsistent system}

inner product

inner product space

inverse of a linear transformation

isomorphism

inverse of a square matrix

iterate

\href{https://ximera.osu.edu/oerlinalg/LinearAlgebra/SYS-0040/main}{iterative methods}

\subsection{J}

\href{https://ximera.osu.edu/oerlinalg/LinearAlgebra/SYS-0040/main}{Jacobi's method}


\subsection{K}

kernel of a linear transformation

\subsection{L}

Laplace Expansion Theorem

leading entry (leading 1) (\ref{def:leadentry})
\begin{expandable}
    The first non-zero entry in a row of a matrix (when read from left to right) is called the \dfn{leading entry}.  When the leading entry is 1, we refer to it as a \dfn{leading 1}.
\end{expandable}

\href{https://ximera.osu.edu/oerlinalg/LinearAlgebra/SYS-0020/main}{leading variable}

least squares error %wait and see

least squares solution %wait and see

linear combination of vectors (\ref{def:lincomb})
\begin{expandable}
    A vector $\vec{v}$ is said to be a \dfn{linear combination} of vectors $\vec{v}_1, \vec{v}_2,\ldots, \vec{v}_n$ if 
$$\vec{v}=a_1\vec{v}_1+ a_2\vec{v}_2+\ldots + a_n\vec{v}_n$$
for some scalars $a_1, a_2, \ldots ,a_n$.
\end{expandable}

linear dependence/independence of matrices

linear equation (\ref{def:lineq})
\begin{expandable}
    A \dfn{linear equation} in variables $x_1, \ldots, x_n$ is an equation that can be written in the form
$$a_1x_1+a_2x_2+\ldots +a_nx_n=b$$
where $a_1,\ldots ,a_n$ and $b$ are constants.
\end{expandable}

linear transformation

linearly dependent vectors (\ref{def:linearindependence})
\begin{expandable}
    Let $\vec{v}_1, \vec{v}_2,\ldots ,\vec{v}_k$ be vectors of $\RR^n$.  We say that the set $\{\vec{v}_1, \vec{v}_2,\ldots ,\vec{v}_k\}$ is \dfn{linearly independent} if the only solution to 
\begin{equation}\label{eq:defLinInd}c_1\vec{v}_1+c_2\vec{v}_2+\ldots +c_p\vec{v}_k=\vec{0}\end{equation}
is the \dfn{trivial solution} $c_1=c_2=\ldots =c_k=0$.

If, in addition to the trivial solution, a \dfn{non-trivial solution} (not all $c_1, c_2,\ldots ,c_k$ are zero) exists, then we say that the set $\{\vec{v}_1, \vec{v}_2,\ldots ,\vec{v}_k\}$ is \dfn{linearly dependent}.
\end{expandable}

linearly independent vectors (\ref{def:linearindependence})
\begin{expandable}
    Let $\vec{v}_1, \vec{v}_2,\ldots ,\vec{v}_k$ be vectors of $\RR^n$.  We say that the set $\{\vec{v}_1, \vec{v}_2,\ldots ,\vec{v}_k\}$ is \dfn{linearly independent} if the only solution to 
\begin{equation}\label{eq:defLinInd}c_1\vec{v}_1+c_2\vec{v}_2+\ldots +c_p\vec{v}_k=\vec{0}\end{equation}
is the \dfn{trivial solution} $c_1=c_2=\ldots =c_k=0$.

If, in addition to the trivial solution, a \dfn{non-trivial solution} (not all $c_1, c_2,\ldots ,c_k$ are zero) exists, then we say that the set $\{\vec{v}_1, \vec{v}_2,\ldots ,\vec{v}_k\}$ is \dfn{linearly dependent}.
\end{expandable}

LU factorization

\subsection{M}
Magnitude of a vector (\ref{def:normrn})
\begin{expandable}
  %  \begin{definition}
Let $\vec{v}=\begin{bmatrix}v_1\\ v_2\\ \vdots \\v_n\end{bmatrix}$ be a vector in $\RR^n$, then the \dfn{length}, or the \dfn{magnitude}, of $\vec{v}$ is given by
$$  \norm{\vec{v}}=\sqrt{v_1^2+v_2^2+\ldots +v_n^2}$$
%\end{definition}
\end{expandable}

\href{https://ximera.osu.edu/oerlinalg/LinearAlgebra/MAT-0010/main}{main diagonal}
    
\href{https://ximera.osu.edu/oerlinalg/LinearAlgebra/MAT-0010/main}{matrix}

matrix addition (\ref{def:additionofmatrices})
\begin{expandable}
    Let $A=\begin{bmatrix} a_{ij}\end{bmatrix} $ and $B=\begin{bmatrix} b_{ij}\end{bmatrix}$ be two
$m\times n$ matrices. Then the \dfn{sum of matrices} $A$ and $B$, denoted by $A+B$,  is an $m \times n$
matrix  given by 
$$A+B=\begin{bmatrix}a_{ij}+b_{ij}\end{bmatrix}$$
\end{expandable}

matrix equality (\ref{def:equalityofmatrices})
\begin{expandable}
    Let $A=\begin{bmatrix} a_{ij}\end{bmatrix}$ and $B=\begin{bmatrix} b_{ij}\end{bmatrix}$ be two $m \times n$ matrices. Then $A=B$ means
that $a_{ij}=b_{ij}$ for all $1\leq i\leq m$ and 
$1\leq j\leq n$.
\end{expandable}

matrix factorization

matrix multiplication (by a matrix)

matrix multiplication (by a scalar) (\ref{def:scalarmultofmatrices})
\begin{expandable}
    If $A=\begin{bmatrix} a_{ij}\end{bmatrix} $ and $k$ is a scalar,
then $kA=\begin{bmatrix} ka_{ij}\end{bmatrix}$. 
\end{expandable}

matrix multiplication (by a vector)

matrix of a linear transformation

matrix powers

minor of a square matrix

\subsection{N}

negative of a matrix

norm

\href{https://ximera.osu.edu/oerlinalg/LinearAlgebra/RRN-0030/main}{normal vector}

null space of a matrix

nullity of a matrix


\subsection{O}
one-to-one

onto

orthogonal basis

orthogonal complement of a subspace

Orthogonal Decomposition Theorem

orthogonal matrix

Orthogonal projection onto a subspace

Orthogonal projection onto a vector (\ref{def:projection})
\begin{expandable}
    Let $\vec{v}$ be a vector, and let $\vec{d}$ be a non-zero vector.  The \dfn{projection of $\vec{v}$ onto $\vec{d}$} is given by 
$$\mbox{proj}_{\vec{d}}\vec{v}=\left(\frac{\vec{v}\dotp\vec{d}}{\norm{\vec{d}}^2}\right)\vec{d}$$
\end{expandable}

Orthogonal vectors (\ref{def:orthovectors}) 
\begin{expandable}
 %   \begin{definition}
Let $\vec{u}$ and $\vec{v}$ be vectors in $\RR^n$. We say $\vec{u}$ and $\vec{v}$ are \dfn{orthogonal} if $\vec{u}\dotp \vec{v}=0$.
%\end{definition}
\end{expandable}


orthogonally diagonalizable matrix

orthonormal basis

orthonormal set of vectors

outer product



\subsection{P}



parametric equation of a line (\ref{form:paramlinend})
\begin{expandable}
    Let $\vec{v}=\begin{bmatrix}v_1\\v_2\\\vdots\\v_n\end{bmatrix}$ be a direction vector for line $l$ in $\RR^n$, and let $(a_1, a_2,\ldots , a_n)$ be an arbitrary point on $l$.  Then the following parametric equations describe $l$:
\[
x_1=v_1t+a_1\]
\[x_2=v_2t+a_2\]
\[\vdots\]
\[x_n=v_nt+a_n
\]
\end{expandable}

partitioned matrices (block multiplication)

permutation matrix

\href{https://ximera.osu.edu/oerlinalg/LinearAlgebra/SYS-0030/main}{pivot}

power method (and its variants)

properties of determinants

properties of matrix algebra

properties of orthogonal matrices

\subsection{Q}

QR factorization

\subsection{R}

range of a linear transformation

rank of a matrix (\ref{def:rankofamatrix})
\begin{expandable}
    The \dfn{rank} of matrix $A$, denoted by $\mbox{rank}(A)$, is the number of nonzero rows that remain after we reduce $A$ to row-echelon form by elementary row operations.
\end{expandable}

Rank Theorem

reduced row echelon form (\ref{def:rref})
\begin{expandable}
    A matrix that is already in \dfn{row-echelon} form is said to be in \dfn{reduced row-echelon form} if:
\begin{enumerate}
\item Each leading entry is $1$
\item All entries {\it above} and below each leading $1$ are $0$
\end{enumerate}
\end{expandable}

redundant vectors (\ref{def:redundant})
\begin{expandable}
    Let $\{\vec{v}_1,\vec{v}_2,\dots,\vec{v}_k\}$ be a set of vectors in $\RR^n$.  If we can remove one vector without changing the span of this set, then that vector is \dfn{redundant}.  In other words, if $$\mbox{span}\left(\vec{v}_1,\vec{v}_2,\dots,\vec{v}_k\right)=\mbox{span}\left(\vec{v}_1,\vec{v}_2,\dots,\vec{v}_{j-1},\vec{v}_{j+1},\dots,\vec{v}_k\right)$$ we say that $\vec{v}_j$ is a redundant element of $\{\vec{v}_1,\vec{v}_2,\dots,\vec{v}_k\}$, or simply redundant.
\end{expandable}

representation of matrix products

row echelon form (\ref{def:ref})
\begin{expandable}
    A matrix is said to be in \dfn{row-echelon form} if:
\begin{enumerate}
\item All entries below each leading entry are 0.
\item Each leading entry is in a column to the right of the leading entries in the rows above it.
\item All rows of zeros, if there are any, are located below non-zero rows.
\end{enumerate}
\end{expandable}

\href{https://ximera.osu.edu/oerlinalg/LinearAlgebra/SYS-0020/main}{row equivalent matrices}

\href{https://ximera.osu.edu/oerlinalg/LinearAlgebra/MAT-0010/main}{row matrix (vector)}

row space of a matrix


\subsection{S}
\href{https://ximera.osu.edu/oerlinalg/LinearAlgebra/VEC-0010/main}{Scalar} 

Standard Unit Vectors (\ref{def:standardunitvecrn})
\begin{expandable}
    %\begin{definition}
  Let $\vec{e}_i$ denote a vector that has $1$ as the $i^{th}$ component and zeros elsewhere.  In other words, $$\vec{e}_i=\begin{bmatrix}
0\\
0\\
\vdots\\
1\\
\vdots\\
0
\end{bmatrix}$$ 
  where $1$ is in the $i^{th}$ position.  We say that  $\vec{e}_i$ is a \dfn{standard unit vector of $\RR^n$}.
%\end{definition}
\end{expandable}

scalar multiple of a matrix

similar matrices

singular value decomposition (SVD)

singular values

span of a set of matrices

span of a set of vectors (\ref{def:span})
\begin{expandable}
    Let $\vec{v}_1, \vec{v}_2,\ldots ,\vec{v}_p$ be vectors in $\RR^n$.  The set $S$ of all linear combinations of $\vec{v}_1, \vec{v}_2,\ldots ,\vec{v}_p$ is called the \dfn{span} of $\vec{v}_1, \vec{v}_2,\ldots ,\vec{v}_p$.  We write 
$$S=\mbox{span}(\vec{v}_1, \vec{v}_2,\ldots ,\vec{v}_p)$$
and we say that vectors $\vec{v}_1, \vec{v}_2,\ldots ,\vec{v}_p$ \dfn{span} $S$.  Any vector in $S$ is said to be \dfn{in the span} of $\vec{v}_1, \vec{v}_2,\ldots ,\vec{v}_p$.  The set $\{\vec{v}_1, \vec{v}_2,\ldots ,\vec{v}_p\}$ is called a \dfn{spanning set} for $S$.
\end{expandable}

\href{https://ximera.osu.edu/oerlinalg/LinearAlgebra/VEC-0090/main}{spanning set}

spectral decomposition

Spectral Theorem

standard basis

strictly diagonally dominant (\ref{def:strict_diag_dom})
\begin{expandable}
    Let $A=[a_{ij}]$ be the $n\times n$ matrix which is the coefficient matrix of the linear system $A \vec{x}= \vec{b}$.  Let
$$
r_i(A):= \sum_{j \ne i} |a_{ij}|
$$
denote the sum of the absolute values of the non-diagonal entries in row $i$.  We say that $A$ is \dfn{strictly diagonally dominant} if 
$$|a_{ii}|>r_i(A)$$
for all values of $i$ from $i=1$ to $i=n$.
\end{expandable}

\href{https://ximera.osu.edu/oerlinalg/LinearAlgebra/MAT-0010/main}{square matrix}

standard matrix of a linear transformation

\href{https://ximera.osu.edu/oerlinalg/LinearAlgebra/VEC-0010/main}{Standard Position}

subspace

\href{https://ximera.osu.edu/oerlinalg/LinearAlgebra/VEC-0030/main}{Subtraction of vectors}

symmetric matrix

system of linear equations



\subsection{T}

transpose of a matrix

triangle inequality

\subsection{U}
\href{https://ximera.osu.edu/oerlinalg/LinearAlgebra/VEC-0036/main}{Unit Vector}

\subsection{V}
\href{https://ximera.osu.edu/oerlinalg/LinearAlgebra/VEC-0010/main}{Vector}

Vector equation of a line (\ref{form:vectorlinend})
\begin{expandable}
    Let $\vec{v}=\begin{bmatrix}v_1\\v_2\\\vdots\\v_n\end{bmatrix}$ be a direction vector for line $l$ in $\RR^n$, and let $(a_1, a_2,\ldots , a_n)$ be an arbitrary point on $l$.  Then the following vector equation describes $l$:
$$\vec{x}(t)=\begin{bmatrix}v_1\\v_2\\\vdots\\v_n\end{bmatrix}t+\begin{bmatrix}a_1\\a_2\\\vdots\\a_n\end{bmatrix}$$
\end{expandable}

vector space

\subsection{W}

\subsection{X}

\subsection{Y}

\subsection{Z}

zero matrix (\ref{def:zeromatrix})
\begin{expandable}
    The $m\times n$ \dfn{zero matrix} is the $m\times n$ matrix
having every entry equal to zero. The zero matrix is
denoted by $O$.
\end{expandable}

\end{document}
