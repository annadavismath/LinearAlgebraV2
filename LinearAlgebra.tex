\documentclass[10pt,handout,twocolumn,twoside,wordchoicegiven]{xourse}

%% You can put user macros here
%% However, you cannot make new environments

\listfiles

\graphicspath{
{./}
{./LTR-0070/}
{./VEC-0060/}
{./APP-0020/}
}

\usepackage{tikz}
\usepackage{tkz-euclide}
\usepackage{tikz-3dplot}
\usepackage{tikz-cd}
\usetikzlibrary{shapes.geometric}
\usetikzlibrary{arrows}
%\usetkzobj{all}
\pgfplotsset{compat=1.13} % prevents compile error.

%\renewcommand{\vec}[1]{\mathbf{#1}}
\renewcommand{\vec}{\mathbf}
\newcommand{\RR}{\mathbb{R}}
\newcommand{\dfn}{\textit}
\newcommand{\dotp}{\cdot}
\newcommand{\id}{\text{id}}
\newcommand\norm[1]{\left\lVert#1\right\rVert}
 
\newtheorem{general}{Generalization}
\newtheorem{initprob}{Exploration Problem}

\tikzstyle geometryDiagrams=[ultra thick,color=blue!50!black]

%\DefineVerbatimEnvironment{octave}{Verbatim}{numbers=left,frame=lines,label=Octave,labelposition=topline}



\usepackage{mathtools}


%\pdfOnly{\usepackage{printStyles/lulu1}}

\title{Linear Algebra OER}
\begin{document}
\maketitle

\setcounter{tocdepth}{2}
\begin{onlineOnly}
\activity{ABOUT/main.tex} 
\end{onlineOnly}

%\part{Functions, limits, and continuity}
%\part{Content for the First Exam}


%% Understanding Functions
\chapterstyle
\activity{XLAChapter_prelim/main.tex}
\sectionstyle
\activity{RRN-0010/main.tex}  
\activity{VEC-0010/main.tex}  
\activity{VEC-0020/main.tex}  
\activity{VEC-0030/main.tex}  
\activity{VEC-0035/main.tex} 
\activity{VEC-0036/main.tex} 
\activity{VEC-0050/main.tex}  
\activity{VEC-0060/main.tex}  
\activity{VEC-0070/main.tex}  
\activity{RRN-0020/main.tex}  
\activity{RRN-0030/main.tex}  

%% Review of famous functions
\chapterstyle
\activity{XLAChapter_systems/main.tex}
\sectionstyle
\activity{SYS-0010/main.tex}  
\activity{SYS-0020/main.tex} 
\activity{SYS-0030/main.tex}  

%% What is a limit
\chapterstyle
\activity{XLAChapter_bigIdeas/main.tex}
\sectionstyle
\activity{VEC-0040/main.tex} 
\activity{VEC-0090/main.tex} 
\activity{VEC-0100/main.tex} 

%% Indeterminant forms
\chapterstyle
\activity{XLAChapter_matrices/main.tex}
\sectionstyle
\activity{MAT-0010/main.tex}  
\activity{MAT-0020/main.tex}
\activity{MAT-0023/main.tex}  
\activity{MAT-0025/main.tex}  
\activity{MAT-0030/main.tex}
\activity{MAT-0050/main.tex}  
\activity{LTR-0005/main.tex}
\activity{LTR-0070/main.tex}
\activity{MAT-0060/main.tex}  
\activity{MAT-0070/main.tex}
\activity{VEC-0110/main.tex}
\activity{SYS-0050/main.tex}

%% Using limits to detect asymptotes
\chapterstyle
\activity{XLAChapter_subspacesRn/main.tex}
\sectionstyle
\activity{VSP-0020/main.tex}  
\activity{VSP-0030/main.tex}  
\activity{VSP-0035/main.tex}  
\activity{VSP-0040/main.tex}

%% The intermediate value theorem
\chapterstyle
\activity{XLAChapter_linTrans/main.tex}
\sectionstyle
\activity{LTR-0010/main.tex}  
\activity{LTR-0020/main.tex} 
\activity{LTR-0030/main.tex}  
\activity{LTR-0035/main.tex}  
\activity{LTR-0050/main.tex}  

%% An application of limits
\chapterstyle
\activity{XLAChapter_vecSpaces/main.tex}
\sectionstyle
\activity{VSP-0050/main.tex}
\activity{VSP-0060/main.tex}  
\activity{LTR-0022/main.tex}
\activity{LTR-0025/main.tex}   
\activity{LTR-0060/main.tex}  
\activity{LTR-0080/main.tex}  

%\part{Derivatives}

%% Definition of the derivative
\chapterstyle
\activity{XLAChapter_det/main.tex}
\sectionstyle
\activity{DET-0010/main.tex}  
\activity{DET-0020/main.tex}  
\activity{DET-0030/main.tex}  
\activity{DET-0040/main.tex}  
\activity{DET-0050/main.tex}  
\activity{DET-0060/main.tex}  
\activity{VEC-0080/main.tex}  
\activity{DET-0070/main.tex}  



%% The derivative as a function
\chapterstyle
\activity{XLAChapter_eigenvalues/main.tex}
\sectionstyle
\activity{EIG-0010/main.tex} 
\activity{EIG-0020/main.tex}  
\activity{EIG-0040/main.tex}
\activity{EIG-0050/main.tex}

%% The derivative as a function
\chapterstyle
\activity{XLAChapter_orthogonality/main.tex}
\sectionstyle
\activity{RTH-0010/main.tex} 
\activity{RTH-0015/main.tex} 
\activity{RTH-0020/main.tex} 

%% Rules of differentiation
\chapterstyle
\activity{XLAChapter_applications/main.tex}
\sectionstyle
\activity{APP-0010/main.tex}
\activity{APP-0020/main.tex}
\activity{APP-0030/main.tex}
\activity{APP-0050/main.tex}
\activity{APP-0060/main.tex}



\end{document}