\documentclass{ximera}
%% You can put user macros here
%% However, you cannot make new environments

\listfiles

\graphicspath{
{./}
{./LTR-0070/}
{./VEC-0060/}
{./APP-0020/}
}

\usepackage{tikz}
\usepackage{tkz-euclide}
\usepackage{tikz-3dplot}
\usepackage{tikz-cd}
\usetikzlibrary{shapes.geometric}
\usetikzlibrary{arrows}
%\usetkzobj{all}
\pgfplotsset{compat=1.13} % prevents compile error.

%\renewcommand{\vec}[1]{\mathbf{#1}}
\renewcommand{\vec}{\mathbf}
\newcommand{\RR}{\mathbb{R}}
\newcommand{\dfn}{\textit}
\newcommand{\dotp}{\cdot}
\newcommand{\id}{\text{id}}
\newcommand\norm[1]{\left\lVert#1\right\rVert}
 
\newtheorem{general}{Generalization}
\newtheorem{initprob}{Exploration Problem}

\tikzstyle geometryDiagrams=[ultra thick,color=blue!50!black]

%\DefineVerbatimEnvironment{octave}{Verbatim}{numbers=left,frame=lines,label=Octave,labelposition=topline}



\usepackage{mathtools}


\title{Additional Exercises for Ch 10} \license{CC BY-NC-SA 4.0}

\begin{document}

\begin{abstract}
\end{abstract}
\maketitle

\section*{Additional Exercises for Chapter 10: Abstract Vector Spaces}


\begin{problem}\label{prb:10.1} Suppose you have $\mathbb{R}^{2}$ and the $+$ operation is as
follows:\
\begin{equation*}
\left( a,b\right) +\left( c,d\right) =\left( a+d,b+c\right) .
\end{equation*}
Scalar multiplication is defined in the usual way. Is this a vector space?
Explain why or why not.
%\begin{hint}
%\end{hint}
\end{problem}

\begin{problem}\label{prb:10.2} Suppose you have $\mathbb{R}^{2}$ and the $+$ operation is defined as
follows.
\begin{equation*}
\left( a,b\right) +\left( c,d\right) =\left( 0,b+d\right)
\end{equation*}
Scalar multiplication is defined in the usual way. Is this a vector space?
Explain why or why not.
%\begin{hint}
%\end{hint}
\end{problem}

\begin{problem}\label{prb:10.3} Suppose you have $\mathbb{R}^{2}$ and scalar multiplication is defined
as $c\left( a,b\right) =\left( a,cb\right) $ while vector addition is
defined as usual. Is this a vector space? Explain why or why not.
%\begin{hint}
%\end{hint}
\end{problem}

\begin{problem}\label{prb:10.4} Suppose you have $\mathbb{R}^{2}$ and the $+$ operation is defined as
follows.
\begin{equation*}
\left( a,b\right) +\left( c,d\right) =\left( a-c,b-d\right)
\end{equation*}
Scalar multiplication is same as usual. Is this a vector space? Explain why
or why not.
%\begin{hint}
%\end{hint}
\end{problem}

\begin{problem}\label{prb:10.5} \label{functions}Consider all the functions defined on a non empty set
which have values in $\mathbb{R}$. Is this a vector space? Explain.
The operations are defined as follows. Here $f,g$ signify functions and $a$
is a scalar.
\begin{eqnarray*}
\left( f+g\right) \left( x\right) &=&f\left( x\right) +g\left( x\right) \\
\left( af\right) \left( x\right) &=&a\left( f\left( x\right) \right)
\end{eqnarray*}
%\begin{hint}
%\end{hint}
\end{problem}




\begin{problem}\label{prb:10.10} Consider the set of $n\times n$ symmetric matrices. That is, $A=A^{T}.$
In other words, $A_{ij}=A_{ji}$. Show that this set of symmetric matrices is
a vector space and a subspace of the vector space of $n\times n$ matrices.
%\begin{hint}
%\end{hint}
\end{problem}

\begin{problem}\label{prb:10.11} Consider the set of all vectors $\begin{bmatrix}x\\y\end{bmatrix}$ in $\mathbb{R}^{2}$
such that $x+y\geq 0.$ Let the vector space operations be the usual ones. Is
this a vector space? Is it a subspace of $\mathbb{R}^{2}$?
%\begin{hint}
%\end{hint}
\end{problem}

\begin{problem}\label{prb:10.12} Consider the vectors $\begin{bmatrix}x\\y\end{bmatrix}$ in $\mathbb{R}^{2}$ such that $xy=0$. Is this a subspace of $\mathbb{R}^{2}?$ Is it a vector space? The
addition and scalar multiplication are the usual operations.
%\begin{hint}
%\end{hint}
\end{problem}

\begin{problem}\label{prb:10.13} Define the operation of vector addition on $\mathbb{R}^{2}$ by $\left(
x,y\right) +\left( u,v\right) =\left( x+u,y+v+1\right) .$ Let scalar
multiplication be the usual operation. Is this a vector space with these
operations? Explain.
%\begin{hint}
%\end{hint}
\end{problem}

\begin{problem}\label{prb:10.14} Let the vectors be real numbers. Define vector space operations in the
usual way. That is $x+y$ means to add the two numbers and $xy$ means to
multiply them. Is $\mathbb{R}$ with these operations a vector space? Explain.
%\begin{hint}
%\end{hint}
\end{problem}

\begin{problem}\label{prb:10.15} Let the scalars be the rational numbers and let the vectors be
real numbers which are the form $a+b\sqrt{2}$ for $a,b$ rational numbers.
Show that with the usual operations, this is a vector space.
%\begin{hint}
%\end{hint}
\end{problem}

\begin{problem}\label{prb:10.16} Let $\mathbb{P}_{2}$ be the set of all polynomials of degree 2 or
less. That is, these are of the form $a+bx+cx^{2}$. Addition is defined as
\begin{equation*}
\left( a+bx+cx^{2}\right) +\left( \hat{a}+\hat{b}x+\hat{c}x^{2}\right)
=\left( a+\hat{a}\right) +\left( b+\hat{b}\right) x+\left( c+\hat{c}\right)
x^{2}
\end{equation*}
and scalar multiplication is defined as
\begin{equation*}
d\left( a+bx+cx^{2}\right) =da+dbx+cdx^{2}
\end{equation*}
Show that, with this definition of the vector space operations that $\mathbb{
P}_{2}$ is a vector space. Now let $V$ denote those polynomials $a+bx+cx^{2}$
such that $a+b+c=0$. Is $V$ a subspace of $\mathbb{P}_{2}?$ Explain.
%\begin{hint}
%\end{hint}
\end{problem}

\begin{problem}\label{prb:10.17} Let $M,N$ be subspaces of a vector space $V$ and consider $M+N$
defined as the set of all $m+n$ where $m\in M$ and $n\in N$. Show that $M+N$
is a subspace of $V$.
%\begin{hint}
%\end{hint}
\end{problem}

\begin{problem}\label{prb:10.18} Let $M,N$ be subspaces of a vector space $V$. Then $M\cap N$ consists
of all vectors which are in both $M$ and $N$. Show that $M\cap N$ is a
subspace of $V$.
%\begin{hint}
%\end{hint}
\end{problem}

\begin{problem}\label{prb:10.19} Let $M,N$ be subspaces of a vector space $\mathbb{R}^{2}.$ Then $N\cup
M$ consists of all vectors which are in either $M$ or $N$. Show that $N\cup
M $ is not necessarily a subspace of $\mathbb{R}^{2}$ by giving an example
where $N\cup M$ fails to be a subspace.
%\begin{hint}
%\end{hint}
\end{problem}


\begin{problem}\label{prb:10.20} \label{4julyprob1}Let $X$ consist of the real valued functions which
are defined on an interval $\left[ a,b\right] .$ For $f,g\in X,\;f+g$ is the
name of the function which satisfies $\left( f+g\right) \left( x\right)
=f\left( x\right) +g\left( x\right)$. For $s$ a real number, $
\left( s f\right) \left( x\right) = s \left( f\left( x\right)
\right) $. Show this is a vector space.
\begin{hint}
The axioms of a vector space all hold because they
hold for a vector space. The only thing left to verify is the
assertions about the things which are supposed to exist. $0$ would
be the zero function which sends everything to $0$. This is an additive
identity. Now if $f$ is a function, $-f\left( x\right) \equiv \left(
-f\left( x\right) \right) $. Then
\[
\left( f+\left( -f\right) \right) \left( x\right) \equiv f\left( x\right)
+\left( -f\right) \left( x\right) \equiv f\left( x\right) +\left( -f\left(
x\right) \right) =0
\]
Hence $f+-f=\mathbf{0}$. For each $x\in \left[ a,b\right] ,$ let $%
f_{x}\left( x\right) =1$ and $f_{x}\left( y\right) =0$ if $y\neq x.$ Then
these vectors are obviously linearly independent.
\end{hint}
\end{problem}



\begin{problem}\label{prb:10.22} Let the vectors be polynomials of degree no more than 3. Show that
with the usual definitions of scalar multiplication and addition wherein,
for $p\left( x\right) $ a polynomial, $\left( a p\right) \left(
x\right) = a p\left( x\right) $ and for $p,q$ polynomials $\left(
p+q\right) \left( x\right) =  p\left( x\right) +q\left( x\right) ,$ this
is a vector space.
\begin{hint}
This is just a subspace of the vector space of functions
because it is closed with respect to vector addition and scalar
multiplication. Hence this is a vector space.
\end{hint}
\end{problem}

% \begin{problem}\label{prb:10.23} Let $V$ be a vector space and suppose $\left\{ \vec{x}_{1},\cdots ,
% \vec{x}_{k}\right\}$ is a set of vectors in $V$. Show that $\vec{0}$
% is in $\mbox{span}\left\{ \vec{x}_{1},\cdots ,\vec{x}_{k}\right\} .$
% \begin{hint}
% $\sum_{i=1}^{k}0\vec{x}_{k}=\vec{0}$
% \end{hint}
% \end{problem}

\begin{problem}\label{prb:10.24} Let $p(x) = 4x^2-x$. Is $p(x)$ in
 $\mbox{span} \left( x^2+x, x^2-1, -x + 2 \right)$
% %\begin{hint}
% %\end{hint}
\end{problem}

\begin{problem}\label{prb:10.25} Let $p(x) = - x^2 + x + 2 $. Is $p(x)$ in $\mbox{span} \left( x^2 + x + 1, 2x^2 + x \right)$?
%\begin{hint}
%\end{hint}
\end{problem}

\begin{problem}\label{prb:10.26} Let $A = \left[ \begin{array}{rr}
1 & 3 \\
0 & 0
\end{array} \right]$. Is $A$ in
\[
\mbox{span} \left(
\left[ \begin{array}{rr}
1 & 0 \\
0 & 1
\end{array} \right], \left[ \begin{array}{rr}
0 & 1 \\
1 & 0
\end{array} \right], \left[ \begin{array}{rr}
1 & 0 \\
1 & 1
\end{array} \right], \left[ \begin{array}{rr}
0 & 1 \\
1 & 1
\end{array} \right]
\right)?
\]
%\begin{hint}
%\end{hint}
\end{problem}


\begin{problem}\label{prb:10.27} Show that the spanning set in Question \ref{prb:10.26} is a spanning set for $\mathbb{M}_{22}$, the vector space of all $2 \times 2$ matrices.
%\begin{hint}
%\end{hint}
\end{problem}

\begin{problem}\label{prb:10.28} Consider the vector space of polynomials of degree at most $2,$ $%
\mathbb{P}_{2}$. Determine whether the following is a basis for $\mathbb{P}%
_{2}$.
\begin{equation*}
\left\{ x^{2}+x+1,2x^{2}+2x+1,x+1\right\}
\end{equation*}
\textbf{Hint:\ }There is a isomorphism from $\mathbb{R}^{3}$ to $\mathbb{P}
_{2}$. It is defined as follows:\
\begin{equation*}
T(\vec{e}_{1})=1,\quad T(\vec{e}_{2})=x,\quad T(\vec{e}_{3})=x^{2}
\end{equation*}
Then extend $T$ linearly. Thus
\begin{equation*}
T\left(\left[
\begin{array}{c}
1 \\
1 \\
1
\end{array}
\right]\right) =x^{2}+x+1,\quad T\left(\left[
\begin{array}{c}
1 \\
2 \\
2
\end{array}
\right]\right) =2x^{2}+2x+1,\quad T\left(\left[
\begin{array}{c}
1 \\
1 \\
0
\end{array}
\right]\right) =1+x
\end{equation*}
It follows that if
\begin{equation*}
\left\{ \left[
\begin{array}{c}
1 \\
1 \\
1
\end{array}
\right] ,\left[
\begin{array}{c}
1 \\
2 \\
2
\end{array}
\right] ,\left[
\begin{array}{c}
1 \\
1 \\
0
\end{array}
\right] \right\}
\end{equation*}
is a basis for $\mathbb{R}^{3},$ then the polynomials will be a basis for $
\mathbb{P}_{2}$ because they will be independent. Recall that an isomorphism
takes a linearly independent set to a linearly independent set. Also, since $
T$ is an isomorphism, it preserves all linear relations.
%\begin{hint}
%\end{hint}
\end{problem}

\begin{problem}\label{prb:10.29} Find a basis in $\mathbb{P}_{2}$ for the subspace
\begin{equation*}
\mbox{span}\left( 1+x+x^{2},1+2x,1+5x-3x^{2}\right)
\end{equation*}
If the above three vectors do not yield a basis, exhibit one of them as a
linear combination of the others. 
\begin{hint}
This is the situation in
which you have a spanning set and you want to cut it down to form a linearly
independent set which is also a spanning set. Use the same isomorphism
above. Since $T$ is an isomorphism, it preserves all linear relations so if
such can be found in $\mathbb{R}^{3},$ the same linear relations will be
present in $\mathbb{P}_{2}$.
\end{hint}
\end{problem}


\begin{problem}\label{prb:10.30} Find a basis in $\mathbb{P}_{3}$ for the subspace
\begin{equation*}
\mbox{span}\left(
1+x-x^{2}+x^{3},1+2x+3x^{3},-1+3x+5x^{2}+7x^{3},1+6x+4x^{2}+11x^{3}\right)
\end{equation*}
If the above three vectors do not yield a basis, exhibit one of them as a
linear combination of the others.
%\begin{hint}
%\end{hint}
\end{problem}


\begin{problem}\label{prb:10.31} Find a basis in $\mathbb{P}_{3}$ for the subspace
\begin{equation*}
\mbox{span}\left(
1+x-x^{2}+x^{3},1+2x+3x^{3},-1+3x+5x^{2}+7x^{3},1+6x+4x^{2}+11x^{3}\right)
\end{equation*}
If the above three vectors do not yield a basis, exhibit one of them as a
linear combination of the others.
%\begin{hint}
%\end{hint}
\end{problem}


\begin{problem}\label{prb:10.32} Find a basis in $\mathbb{P}_{3}$ for the subspace
\begin{equation*}
\mbox{span}\left(
x^{3}-2x^{2}+x+2,3x^{3}-x^{2}+2x+2,7x^{3}+x^{2}+4x+2,5x^{3}+3x+2\right)
\end{equation*}
If the above three vectors do not yield a basis, exhibit one of them as a
linear combination of the others.
%\begin{hint}
%\end{hint}
\end{problem}


\begin{problem}\label{prb:10.33} Find a basis in $\mathbb{P}_{3}$ for the subspace
\begin{equation*}
\mbox{span}\left(
x^{3}+2x^{2}+x-2,3x^{3}+3x^{2}+2x-2,3x^{3}+x+2,3x^{3}+x+2\right)
\end{equation*}
If the above three vectors do not yield a basis, exhibit one of them as a
linear combination of the others.
%\begin{hint}
%\end{hint}
\end{problem}


\begin{problem}\label{prb:10.34} Find a basis in $\mathbb{P}_{3}$ for the subspace
\begin{equation*}
\mbox{span}\left(
x^{3}-5x^{2}+x+5,3x^{3}-4x^{2}+2x+5,5x^{3}+8x^{2}+2x-5,11x^{3}+6x+5\right)
\end{equation*}
If the above three vectors do not yield a basis, exhibit one of them as a
linear combination of the others.
%\begin{hint}
%\end{hint}
\end{problem}


\begin{problem}\label{prb:10.35} Find a basis in $\mathbb{P}_{3}$ for the subspace
\begin{equation*}
\mbox{span}\left(
x^{3}-3x^{2}+x+3,3x^{3}-2x^{2}+2x+3,7x^{3}+7x^{2}+3x-3,7x^{3}+4x+3\right)
\end{equation*}
If the above three vectors do not yield a basis, exhibit one
of them as a linear combination of the others.
%\begin{hint}
%\end{hint}
\end{problem}


% \begin{problem}\label{prb:10.36} Find a basis in $\mathbb{P}_{3}$ for the subspace
% \begin{equation*}
% \mbox{span}\left\{
% x^{3}-x^{2}+x+1,3x^{3}+2x+1,4x^{3}+x^{2}+2x+1,3x^{3}+2x-1\right\}
% \end{equation*}
% If the above three vectors do not yield a basis, exhibit one
% of them as a linear combination of the others.
% %\begin{hint}
% %\end{hint}
% \end{problem}


% \begin{problem}\label{prb:10.37} Find a basis in $\mathbb{P}_{3}$ for the subspace
% \begin{equation*}
% \mbox{span}\left\{
% x^{3}-x^{2}+x+1,3x^{3}+2x+1,13x^{3}+x^{2}+8x+4,3x^{3}+2x-1\right\}
% \end{equation*}
% If the above three vectors do not yield a basis, exhibit one
% of them as a linear combination of the others.
% %\begin{hint}
% %\end{hint}
% \end{problem}


% \begin{problem}\label{prb:10.38} Find a basis in $\mathbb{P}_{3}$ for the subspace
% \begin{equation*}
% \mbox{span}\left\{
% x^{3}-3x^{2}+x+3,3x^{3}-2x^{2}+2x+3,-5x^{3}+5x^{2}-4x-6,7x^{3}+4x-3\right\}
% \end{equation*}
% If the above three vectors do not yield a basis, exhibit one
% of them as a linear combination of the others.
% %\begin{hint}
% %\end{hint}
% \end{problem}


% \begin{problem}\label{prb:10.39} Find a basis in $\mathbb{P}_{3}$ for the subspace
% \begin{equation*}
% \mbox{span}\left\{
% x^{3}-2x^{2}+x+2,3x^{3}-x^{2}+2x+2,7x^{3}-x^{2}+4x+4,5x^{3}+3x-2\right\}
% \end{equation*}
% If the above three vectors do not yield a basis, exhibit one
% of them as a linear combination of the others.
% %\begin{hint}
% %\end{hint}
% \end{problem}


% \begin{problem}\label{prb:10.40} Find a basis in $\mathbb{P}_{3}$ for the subspace
% \begin{equation*}
% \mbox{span}\left\{
% x^{3}-2x^{2}+x+2,3x^{3}-x^{2}+2x+2,3x^{3}+4x^{2}+x-2,7x^{3}-x^{2}+4x+4\right
% \}
% \end{equation*}
% If the above three vectors do not yield a basis, exhibit one
% of them as a linear combination of the others.
% %\begin{hint}
% %\end{hint}
% \end{problem}


% \begin{problem}\label{prb:10.41} Find a basis in $\mathbb{P}_{3}$ for the subspace
% \begin{equation*}
% \mbox{span}\left\{
% x^{3}-4x^{2}+x+4,3x^{3}-3x^{2}+2x+4,-3x^{3}+3x^{2}-2x-4,-2x^{3}+4x^{2}-2x-4
% \right\}
% \end{equation*}
% If the above three vectors do not yield a basis, exhibit one
% of them as a linear combination of the others.
% %\begin{hint}
% %\end{hint}
% \end{problem}


% \begin{problem}\label{prb:10.42} Find a basis in $\mathbb{P}_{3}$ for the subspace
% \begin{equation*}
% \mbox{span}\left\{
% x^{3}+2x^{2}+x-2,3x^{3}+3x^{2}+2x-2,5x^{3}+x^{2}+2x+2,10x^{3}+10x^{2}+6x-6
% \right\}
% \end{equation*}
% If the above three vectors do not yield a basis, exhibit one
% of them as a linear combination of the others.
% %\begin{hint}
% %\end{hint}
% \end{problem}


% \begin{problem}\label{prb:10.43} Find a basis in $\mathbb{P}_{3}$ for the subspace
% \begin{equation*}
% \mbox{span}\left\{
% x^{3}+x^{2}+x-1,3x^{3}+2x^{2}+2x-1,x^{3}+1,4x^{3}+3x^{2}+2x-1\right\}
% \end{equation*}
% If the above three vectors do not yield a basis, exhibit one
% of them as a linear combination of the others.
% %\begin{hint}
% %\end{hint}
% \end{problem}


% \begin{problem}\label{prb:10.44} Find a basis in $\mathbb{P}_{3}$ for the subspace
% \begin{equation*}
% \mbox{span}\left\{
% x^{3}-x^{2}+x+1,3x^{3}+2x+1,x^{3}+2x^{2}-1,4x^{3}+x^{2}+2x+1\right\}
% \end{equation*}
% If the above three vectors do not yield a basis, exhibit one
% of them as a linear combination of the others.
% %\begin{hint}
% %\end{hint}
% \end{problem}


\begin{problem}\label{prb:10.45} Here are some vectors.
\begin{equation*}
\left\{ x^{3}+x^{2}-x-1,3x^{3}+2x^{2}+2x-1\right\}
\end{equation*}
If these are linearly independent, extend to a basis for all of $\mathbb{P}
_{3}$.
%\begin{hint}
%\end{hint}
\end{problem}


\begin{problem}\label{prb:10.46} Here are some vectors.
\begin{equation*}
\left\{ x^{3}-2x^{2}-x+2,3x^{3}-x^{2}+2x+2\right\}
\end{equation*}
If these are linearly independent, extend to a basis for all of $\mathbb{P}
_{3}$.
%\begin{hint}
%\end{hint}
\end{problem}


\begin{problem}\label{prb:10.47} Here are some vectors.
\begin{equation*}
\left\{ x^{3}-3x^{2}-x+3,3x^{3}-2x^{2}+2x+3\right\}
\end{equation*}
If these are linearly independent, extend to a basis for all of $\mathbb{P}
_{3}$.
%\begin{hint}
%\end{hint}
\end{problem}


\begin{problem}\label{prb:10.48} Here are some vectors.
\begin{equation*}
\left\{ x^{3}-2x^{2}-3x+2,3x^{3}-x^{2}-6x+2,-8x^{3}+18x+10\right\}
\end{equation*}
If these are linearly independent, extend to a basis for all of $\mathbb{P}
_{3}$.
%\begin{hint}
%\end{hint}
\end{problem}


% \begin{problem}\label{prb:10.49} Here are some vectors.
% \begin{equation*}
% \left\{ x^{3}-3x^{2}-3x+3,3x^{3}-2x^{2}-6x+3,-8x^{3}+18x+40\right\}
% \end{equation*}
% If these are linearly independent, extend to a basis for all of $\mathbb{P}
% _{3}$.
% %\begin{hint}
% %\end{hint}
% \end{problem}


% \begin{problem}\label{prb:10.50} Here are some vectors.
% \begin{equation*}
% \left\{ x^{3}-x^{2}+x+1,3x^{3}+2x+1,4x^{3}+2x+2\right\}
% \end{equation*}
% If these are linearly independent, extend to a basis for all of $\mathbb{P}
% _{3}$.
% %\begin{hint}
% %\end{hint}
% \end{problem}


% \begin{problem}\label{prb:10.51} Here are some vectors.
% \begin{equation*}
% \left\{ x^{3}+x^{2}+2x-1,3x^{3}+2x^{2}+4x-1,7x^{3}+8x+23\right\}
% \end{equation*}
% If these are linearly independent, extend to a basis for all of $\mathbb{P}
% _{3}$.
% %\begin{hint}
% %\end{hint}
% \end{problem}


\begin{problem}\label{prb:10.52} Determine if the following set is linearly independent. If it is linearly dependent, write one vector as a linear combination of the other vectors in the set.
\[
\left\{ x+1, x^2 + 2, x^2 - x -3 \right\}
\]
%\begin{hint}
%\end{hint}
\end{problem}

\begin{problem}\label{prb:10.53} Determine if the following set is linearly independent. If it is linearly dependent, write one vector as a linear combination of the other vectors in the set.
\[
\left\{ x^2 + x, -2x^2 -4x -6 , 2x - 2 \right\}
\]
%\begin{hint}
%\end{hint}
\end{problem}

\begin{problem}\label{prb:10.54} Determine if the following set is linearly independent. If it is linearly dependent, write one vector as a linear combination of the other vectors in the set.
\[
\left\{ \left[ \begin{array}{rr}
1 & 2 \\
0 & 1
\end{array} \right], \left[ \begin{array}{rr}
-7 & 2 \\
-2 & -3
\end{array} \right], \left[ \begin{array}{rr}
4 & 0 \\
1 & 2
\end{array} \right]
 \right\}
\]
%\begin{hint}
%\end{hint}
\end{problem}

\begin{problem}\label{prb:10.55} Determine if the following set is linearly independent. If it is linearly dependent, write one vector as a linear combination of the other vectors in the set.
\[
\left\{ \left[ \begin{array}{rr}
1 & 0 \\
0 & 1
\end{array} \right], \left[ \begin{array}{rr}
0 & 1 \\
0 & 1
\end{array} \right], \left[ \begin{array}{rr}
1 & 0 \\
1 & 0
\end{array} \right], \left[ \begin{array}{rr}
0 & 0 \\
1 & 1
\end{array} \right]
 \right\}
\]
%\begin{hint}
%\end{hint}
\end{problem}

\begin{problem}\label{prb:10.56} If you have $5$ vectors in $\mathbb{R}^{5}$ and the vectors are
linearly independent, can it always be concluded they span $\mathbb{R}^{5}?$
\begin{hint}
Yes. If not, there would exist a vector not in the span. But then
you could add in this vector and obtain a linearly independent set of
vectors with more vectors than a basis.
\end{hint}
\end{problem}

\begin{problem}\label{prb:10.57} If you have $6$ vectors in $\mathbb{R}^{5},$ is it possible they are
linearly independent? Explain.
\begin{hint}
No. They can't be.
\end{hint}
\end{problem}

\begin{problem}\label{prb:10.58} Let $\mathbb{P}_3$ be the polynomials of degree no more than 3. Determine which
of the following are bases for this vector space.

\begin{enumerate}
\item $\left\{ x+1,x^{3}+x^{2}+2x,x^{2}+x,x^{3}+x^{2}+x\right\} $

\item $\left\{ x^{3}+1,x^{2}+x,2x^{3}+x^{2},2x^{3}-x^{2}-3x+1\right\} $
\end{enumerate}

\begin{hint}
\begin{enumerate}
\item
\item
Suppose
\[
c_{1}\left( x^{3}+1\right) +c_{2}\left( x^{2}+x\right) +c_{3}\left(
2x^{3}+x^{2}\right) +c_{4}\left( 2x^{3}-x^{2}-3x+1\right) =0
\]
Then combine the terms according to power of $x.$
\[
\left( c_{1}+2c_{3}+2c_{4}\right) x^{3}+\left( c_{2}+c_{3}-c_{4}\right)
x^{2}+\left( c_{2}-3c_{4}\right) x+\left( c_{1}+c_{4}\right) =0
\]
Is there a non zero solution to the system 
$$\begin{array}{c}
c_{1}+2c_{3}+2c_{4}=0 \\
c_{2}+c_{3}-c_{4}=0 \\
c_{2}-3c_{4}=0 \\
c_{1}+c_{4}=0
\end{array}$$
Solution is:
\[
c_{1}=0,c_{2}=0,c_{3}=0,c_{4}=0
\]
Therefore, these are linearly independent.
\end{enumerate}
\end{hint}
\end{problem}

\begin{problem}
Define $T:\mathbb{P}^2\rightarrow\mathbb{P}^3$ by $T(p(x))=xp(x)$.  

\begin{problem}\label{prob:lintransmultbyx1}
Find $T(-x^2+2x-4)$.

Answer: $$T(-x^2+2x-4)=\answer{-x^3+2x^2-4x}$$
\end{problem}

\begin{problem}\label{prob:lintransmultbyx2}
Is $T$ a linear transformation?  If so, prove it.  If not, give a counterexample.
\end{problem}
\end{problem}

\begin{problem}
Let $$\vec{v}_1=\begin{bmatrix}1\\3\\0\end{bmatrix},\quad \vec{v}_2=\begin{bmatrix}0\\1\\-2\end{bmatrix}$$
$$\vec{w}_1=\begin{bmatrix}1\\-1\\4\end{bmatrix},\quad \vec{w}_2=\begin{bmatrix}0\\2\\-1\end{bmatrix}$$

Let $V=\text{span}(\vec{v}_1, \vec{v}_2)$ and $W=\text{span}(\vec{w}_1, \vec{w}_2)$.

Suppose $T:V\rightarrow W$ is a linear transformation such that 
$$T(\vec{v}_1)=\begin{bmatrix}2\\0\\7\end{bmatrix},\quad T(\vec{v}_2)=\begin{bmatrix}-1\\7\\1\end{bmatrix}$$
  \begin{problem}\label{prob:lintransandbasis1}
  Verify that vectors $\begin{bmatrix}2\\0\\7\end{bmatrix}$ and $\begin{bmatrix}-1\\7\\1\end{bmatrix}$ are in $W$ by expressing each as a linear combination of $\vec{w}_1$ and $\vec{w}_2$.
  $$\begin{bmatrix}2\\0\\7\end{bmatrix}=\answer{2}\vec{w}_1+\answer{1}\vec{w}_2$$
  $$\begin{bmatrix}-1\\7\\1\end{bmatrix}=\answer{-1}\vec{w}_1+\answer{3}\vec{w}_2$$
  \end{problem}
  
  \begin{problem}\label{prob:lintransandbasis2}
  Show that $\vec{v}=\begin{bmatrix}1\\2\\2\end{bmatrix}$ is in $V$ by expressing it as a linear combination of $\vec{v}_1$ and $\vec{v}_2$.
  $$\vec{v}=\answer{1}\vec{v}_1+\answer{-1}\vec{v}_2$$
  \end{problem}
  
  \begin{problem}\label{prob:lintransandbasis3}
  Find $T(\vec{v})$ and express it as a linear combination of $\vec{w}_1$ and $\vec{w}_2$.
  $$T(\vec{v})=\answer{3}\vec{w}_1+\answer{-2}\vec{w}_2$$
  \end{problem}
  \end{problem}






\begin{problem}\label{prb:10.59} In the context of the above problem, consider polynomials
\begin{equation*}
\left\{ a_{i}x^{3}+b_{i}x^{2}+c_{i}x+d_{i},\ i=1,2,3,4\right\}
\end{equation*}
Show that this collection of polynomials is linearly independent on an
interval $\left[ s,t\right] $ if and only if
\begin{equation*}
\left[
\begin{array}{cccc}
a_{1} & b_{1} & c_{1} & d_{1} \\
a_{2} & b_{2} & c_{2} & d_{2} \\
a_{3} & b_{3} & c_{3} & d_{3} \\
a_{4} & b_{4} & c_{4} & d_{4}
\end{array}
\right]
\end{equation*}
is an invertible matrix.
\begin{hint}
Let $p_{i}\left( x\right) $ denote the $i^{th}$ of
these polynomials. Suppose $\sum_{i}C_{i}p_{i}\left( x\right) =0.$ Then
collecting terms according to the exponent of $x,$ you need to have
\begin{eqnarray*}
C_{1}a_{1}+C_{2}a_{2}+C_{3}a_{3}+C_{4}a_{4} &=&0 \\
C_{1}b_{1}+C_{2}b_{2}+C_{3}b_{3}+C_{4}b_{4} &=&0 \\
C_{1}c_{1}+C_{2}c_{2}+C_{3}c_{3}+C_{4}c_{4} &=&0 \\
C_{1}d_{1}+C_{2}d_{2}+C_{3}d_{3}+C_{4}d_{4} &=&0
\end{eqnarray*}
The matrix of coefficients is just the transpose of the above matrix. There
exists a non trivial solution if and only if the determinant of this matrix
equals 0.
\end{hint}
\end{problem}





\begin{problem}\label{prb:10.62} Let $M=\left\{ \vec{u}=\begin{bmatrix}u_{1}\\u_{2}\\u_{3}\\u_{4}\end{bmatrix} \in
\mathbb{R}^{4}:\left| u_{1}\right| \leq 4\right\} .$ Is $M$ a subspace of $\mathbb{R}^4$?
\begin{hint}
This is not a subspace. $\left[ \begin{array}{r}
1 \\
1 \\
1 \\
1
\end{array}
\right] $ is in
it, but $20\left[
\begin{array}{r}
1 \\
1 \\
1 \\
1
\end{array}
\right] $ is not.
\end{hint}
\end{problem}

\begin{problem}\label{prb:10.63} Let $M=\left\{ \vec{u}=\begin{bmatrix} u_{1}\\u_{2}\\u_{3}\\u_{4}\end{bmatrix} \in
\mathbb{R}^{4}:\sin \left( u_{1}\right) =1\right\} .$ Is $M$ a subspace of $\mathbb{R}^4$?
\begin{hint}
This is not a subspace.
\end{hint}
\end{problem}

\begin{problem}\label{prb:10.64} Let $W$ be a subset of $\mathbb{M}_{22}$ given by
\[
W = \left\{ A | A \in \mathbb{M}_{22}, A^T = A \right\}
\]
In words, $W$ is the set of all symmetric $2 \times 2$ matrices. Is $W$ a subspace of $\mathbb{M}_{22}$?
%\begin{hint}
%\end{hint}
\end{problem}

\begin{problem}\label{prb:10.65} Let $W$ be a subset of $\mathbb{M}_{22}$ given by
\[
W = \left\{ \left[ \begin{array}{rr}
a  & b \\
c & d
\end{array} \right] | a,b,c,d \in \mathbb{R}, a + b = c + d \right\}
\]
Is $W$ a subspace of $\mathbb{M}_{22}$?
%\begin{hint}
%\end{hint}
\end{problem}

\begin{problem}\label{prb:10.66} Let $W$ be a subset of $P_3$ given by
\[
W = \left\{
ax^3 + bx^2 + cx + d | a,b,c,d \in \mathbb{R}, d = 0 \right\}
\]
Is $W$ a subspace of $P_3$?
%\begin{hint}
%\end{hint}
\end{problem}

\begin{problem}\label{prb:10.67} Let $W$ be a subset of $P_3$ given by
\[
W = \left\{
p(x) = ax^3 + bx^2 + cx + d | a,b,c,d \in \mathbb{R}, p(2) = 1 \right\}
\]
Is $W$ a subspace of $P_3$?
%\begin{hint}
%\end{hint}
\end{problem}

\begin{problem}\label{prb:10.68}
Let $T:\mathbb{P}_2 \to \mathbb{R}$ be a linear transformation such that
\[ T(x^2)=1; T(x^2+x)=5; T(x^2+x+1)=-1.\]
Find $T(ax^2+bx+c)$.
\begin{hint}
By linearity we have
$T(x^2)=1$, $T(x) = T(x^2+x - x^2)= T(x^2+x) - T(x^2)= 5-1=5$, and
$T(1) = T(x^2+x+1 -(x^2+x))=T(x^2+x+1) -T(x^2+x))= -1-5=-6$.

Thus$T(ax^2+bx+c) = aT(x^2) + bT(x) + cT(1) = a+5b-6c$.
\end{hint}
\end{problem}

\begin{problem}\label{prb:10.69} Consider the following functions $T:\mathbb{R}^{3}\rightarrow \mathbb{R}^{2}.$
Explain why each of these functions $T$ is not linear.

\begin{enumerate}
\item $T\left(\left[
\begin{array}{c}
x \\
y \\
z
\end{array}
\right]\right) =\left[
\begin{array}{c}
x+2y+3z+1 \\
2y-3x+z
\end{array}
\right] $

\item $T\left(\left[
\begin{array}{c}
x \\
y \\
z
\end{array}
\right]\right) =\left[
\begin{array}{c}
x+2y^{2}+3z \\
2y+3x+z
\end{array}
\right] $

\item $T\left(\left[
\begin{array}{c}
x \\
y \\
z
\end{array}
\right]\right) =\left[
\begin{array}{c}
\sin x+2y+3z \\
2y+3x+z
\end{array}
\right] $

\item $T\left(\left[
\begin{array}{c}
x \\
y \\
z
\end{array}
\right]\right) =\left[
\begin{array}{c}
x+2y+3z \\
2y+3x-\ln z
\end{array}
\right] $
\end{enumerate}
%\begin{hint}
%\end{hint}
\end{problem}


\begin{problem}\label{prb:10.70} Suppose $T$ is a linear transformation such that
\begin{eqnarray*}
T\left(\left[
\begin{array}{r}
1 \\
1 \\
-7
\end{array}
\right]\right) &=&\left[
\begin{array}{r}
3 \\
3 \\
3
\end{array}
\right] \\
T\left(\left[
\begin{array}{r}
-1 \\
0 \\
6
\end{array}
\right]\right) &=&\left[
\begin{array}{r}
1 \\
2 \\
3
\end{array}
\right] \\
T\left(\left[
\begin{array}{r}
0 \\
-1 \\
2
\end{array}
\right]\right) &=&\left[
\begin{array}{r}
1 \\
3 \\
-1
\end{array}
\right]
\end{eqnarray*}
Find the matrix of $T$. That is find $A$ such that $T(\vec{x})=A\vec{x}$. \vspace{1mm}
\begin{hint}
\[
\left[
\begin{array}{rrr}
3 & 1 & 1 \\
3 & 2 & 3 \\
3 & 3 & -1
\end{array}
\right] \left[
\begin{array}{ccc}
6 & 2 & 1 \\
5 & 2 & 1 \\
6 & 1 & 1
\end{array}
\right] =\left[
\begin{array}{ccc}
29 & 9 & 5 \\
46 & 13 & 8 \\
27 & 11 & 5
\end{array}
\right]
\]
\end{hint}
\end{problem}

\begin{problem}\label{prb:10.71} Suppose $T$ is a linear transformation such that
\begin{eqnarray*}
T\left(\left[
\begin{array}{r}
1 \\
2 \\
-18
\end{array}
\right]\right) &=&\left[
\begin{array}{r}
5 \\
2 \\
5
\end{array}
\right] \\
T\left(\left[
\begin{array}{r}
-1 \\
-1 \\
15
\end{array}
\right]\right) &=&\left[
\begin{array}{r}
3 \\
3 \\
5
\end{array}
\right] \\
T\left(\left[
\begin{array}{r}
0 \\
-1 \\
4
\end{array}
\right]\right) &=&\left[
\begin{array}{r}
2 \\
5 \\
-2
\end{array}
\right]
\end{eqnarray*}
Find the matrix of $T$. That is find $A$ such that $T(\vec{x})=A\vec{x}$. \vspace{1mm}
\begin{hint}
\[
\left[
\begin{array}{rrr}
5 & 3 & 2 \\
2 & 3 & 5 \\
5 & 5 & -2
\end{array}
\right] \left[
\begin{array}{ccc}
11 & 4 & 1 \\
10 & 4 & 1 \\
12 & 3 & 1
\end{array}
\right] =\left[
\begin{array}{ccc}
109 & 38 & 10 \\
112 & 35 & 10 \\
81 & 34 & 8
\end{array}
\right]
\]
\end{hint}
\end{problem}


\begin{problem}\label{prb:10.72} Consider the following functions $T:\mathbb{R}^{3}\rightarrow \mathbb{R}^{2}$.
Show that each is a linear transformation and determine for each the matrix $A$ such that
$T(\vec{x})=A\vec{x}$.

\begin{enumerate}
\item $T\left(\left[
\begin{array}{c}
x \\
y \\
z
\end{array}
\right]\right) =\left[
\begin{array}{c}
x+2y+3z \\
2y-3x+z
\end{array}
\right] $

\item $T\left(\left[
\begin{array}{c}
x \\
y \\
z
\end{array}
\right]\right) =\left[
\begin{array}{c}
7x+2y+z \\
3x-11y+2z
\end{array}
\right] $

\item $T\left(\left[
\begin{array}{c}
x \\
y \\
z
\end{array}
\right]\right) =\left[
\begin{array}{c}
3x+2y+z \\
x+2y+6z
\end{array}
\right] $

\item $T\left(\left[
\begin{array}{c}
x \\
y \\
z
\end{array}
\right]\right) =\left[
\begin{array}{c}
2y-5x+z \\
x+y+z
\end{array}
\right] $
\end{enumerate}
%\begin{hint}
%\end{hint}
\end{problem}


% \begin{problem}\label{prb:10.73} Suppose
% \begin{equation*}
% \left[
% \begin{array}{ccc}
% A_{1} & \cdots & A_{n}
% \end{array}
% \right] ^{-1}
% \end{equation*}
%  exists where each $A_{j}\in \mathbb{R}^{n}$ and let
% vectors  $\left\{ B_{1},\cdots ,B_{n}\right\} $ in $\mathbb{R}^{m}$ be given.
% Show that there \textbf{always }exists a linear
% transformation $T$ such that $T(A_{i})=B_{i}$.
% %\begin{hint}
% %\end{hint}
% \end{problem}

\begin{problem}\label{prb:10.74} Let $V$ and $W$ be subspaces of $\mathbb{R}^{n}$ and $\mathbb{R}^{m}$
respectively and let $T:V\rightarrow W$ be a linear transformation. Suppose
that $\left\{ T(\vec{v}_{1}),\cdots ,T(\vec{v}_{r})\right\} $ is linearly
independent. Show that it must be the case that $\left\{ \vec{v}_{1},\cdots ,
\vec{v}_{r}\right\} $ is also linearly independent.
\begin{hint}
If $\sum_i^r a_i \vec{v}_r =0$, then using linearity properties of $T$ we get
\[ 0 = T(0) =  T\left(\sum_i^r a_i \vec{v}_r\right) =
\sum_i^r a_i T(\vec{v}_r).\]
Since we assume that  $\left\{ T\vec{v}_{1},\cdots ,T\vec{v}_{r}\right\} $ is linearly
independent, we must have all $a_i=0$, and therefore we conclude that
 $\left\{ \vec{v}_{1},\cdots ,
\vec{v}_{r}\right\} $ is also linearly independent.
\end{hint}
\end{problem}


\begin{problem}\label{prb:10.75} Let
\begin{equation*}
V=\mbox{span}\left\{ \left[
\begin{array}{c}
1 \\
1 \\
2 \\
0
\end{array}
\right] ,\left[
\begin{array}{c}
0 \\
1 \\
1 \\
1
\end{array}
\right] ,\left[
\begin{array}{c}
1 \\
1 \\
0 \\
1
\end{array}
\right] \right\}
\end{equation*}
Let $T\vec{x}=A\vec{x}$ where $A$ is the matrix
\begin{equation*}
\left[
\begin{array}{cccc}
1 & 1 & 1 & 1 \\
0 & 1 & 1 & 0 \\
0 & 1 & 2 & 1 \\
1 & 1 & 1 & 2
\end{array}
\right]
\end{equation*}
Give a basis for $\mbox{im}\left( T\right) $.
%\begin{hint}
%\end{hint}
\end{problem}


\begin{problem}\label{prb:10.76} Let
\begin{equation*}
V=\mbox{span}\left\{ \left[
\begin{array}{c}
1 \\
0 \\
0 \\
1
\end{array}
\right] ,\left[
\begin{array}{c}
1 \\
1 \\
1 \\
1
\end{array}
\right] ,\left[
\begin{array}{c}
1 \\
4 \\
4 \\
1
\end{array}
\right] \right\}
\end{equation*}
Let $T\vec{x}=A\vec{x}$ where $A$ is the matrix
\begin{equation*}
\left[
\begin{array}{cccc}
1 & 1 & 1 & 1 \\
0 & 1 & 1 & 0 \\
0 & 1 & 2 & 1 \\
1 & 1 & 1 & 2
\end{array}
\right]
\end{equation*}
Find a basis for $\mbox{im}\left( T\right) $. In this case, the original
vectors do not form an independent set.

\begin{hint}
Since the third vector is a linear combinations of the first two, then
the image of the third vector will also be a linear combinations of
the image of the first two.  However the image of the first two
vectors are linearly independent (check!), and hence form a basis of
the image.

Thus a basis for $\mbox{im}\left( T\right) $ is:

\begin{equation*}
V=\mbox{span}\left\{ \left[
\begin{array}{c}
2 \\
0 \\
1 \\
3
\end{array}
\right] ,\left[
\begin{array}{c}
4 \\
2 \\
4 \\
5
\end{array}
\right]  \right\}
\end{equation*}

\end{hint}
\end{problem}


\begin{problem}\label{prb:10.77} If $\left\{ \vec{v}_{1},\cdots ,\vec{v}_{r}\right\} $ is linearly
independent and $T$ is a one to one linear transformation, show that $
\left\{ T(\vec{v}_{1}),\cdots ,T(\vec{v}_{r})\right\} $ is also linearly
independent. Give an example which shows that if $T$ is only linear, it can
happen that, although $\left\{ \vec{v}_{1},\cdots ,\vec{v}_{r}\right\} $ is
linearly independent, $\left\{ T\vec{v}_{1},\cdots ,T\vec{v}_{r}\right\} $
is not. In fact, show that it can happen that each of the $T\vec{v}_{j}$
equals 0.
%\begin{hint}
%\end{hint}
\end{problem}


\begin{problem}\label{prb:10.78} Let $V$ and $W$ be subspaces of $\mathbb{R}^{n}$ and $\mathbb{R}^{m}$
respectively and let $T:V\rightarrow W$ be a linear transformation. Show
that if $T$ is onto $W$ and if $\left\{ \vec{v}_{1},\cdots ,\vec{v}
_{r}\right\} $ is a basis for $V,$ then $\mbox{span}\left\{ T(\vec{v})
_{1},\cdots ,T(\vec{v}_{r})\right\} =W$.
%\begin{hint}
%\end{hint}
\end{problem}


\begin{problem}\label{prb:10.79} Define $T:\mathbb{R}^{4}\rightarrow \mathbb{R}^{3}$ as follows.
\begin{equation*}
T(\vec{x})=\left[
\begin{array}{rrrr}
3 & 2 & 1 & 8 \\
2 & 2 & -2 & 6 \\
1 & 1 & -1 & 3
\end{array}
\right] \vec{x}
\end{equation*}
Find a basis for $\mbox{im}\left( T\right) $. Also find a basis for $\ker
\left( T\right) .$
%\begin{hint}
%\end{hint}
\end{problem}


\begin{problem}\label{prb:10.80} Define $T:\mathbb{R}^{3}\rightarrow \mathbb{R}^{3}$ as follows.
\begin{equation*}
T(\vec{x})=\left[
\begin{array}{ccc}
1 & 2 & 0 \\
1 & 1 & 1 \\
0 & 1 & 1
\end{array}
\right] \vec{x}
\end{equation*}
where on the right, it is just matrix multiplication of the vector $\vec{x}$
which is meant. Explain why $T$ is an isomorphism of $\mathbb{R}^{3}$ to $
\mathbb{R}^{3}$.
%\begin{hint}
%\end{hint}
\end{problem}


\begin{problem}\label{prb:10.81} Suppose $T:\mathbb{R}^{3}\rightarrow \mathbb{R}^{3}$ is a linear
transformation given by
\begin{equation*}
T(\vec{x})=A\vec{x}
\end{equation*}
where $A$ is a $3\times 3$ matrix. Show that $T$ is an isomorphism if and
only if $A$ is invertible.
%\begin{hint}
%\end{hint}
\end{problem}


\begin{problem}\label{prb:10.82} Suppose $T:\mathbb{R}^{n}\rightarrow \mathbb{R}^{m}$ is a linear
transformation given by
\begin{equation*}
T(\vec{x})=A\vec{x}
\end{equation*}
where $A$ is an $m\times n$ matrix. Show that $T$ is never an isomorphism if
$m\neq n$. In particular, show that if $m>n,$ $T$ cannot be onto and if $
m<n, $ then $T$ cannot be one to one.
%\begin{hint}
%\end{hint}
\end{problem}


\begin{problem}\label{prb:10.83} Define $T:\mathbb{R}^{2}\rightarrow \mathbb{R}^{3}$ as follows.
\begin{equation*}
T(\vec{x})=\left[
\begin{array}{cc}
1 & 0 \\
1 & 1 \\
0 & 1
\end{array}
\right] \vec{x}
\end{equation*}
where on the right, it is just matrix multiplication of the vector $\vec{x}$
which is meant. Show that $T$ is one to one. Next let $W=\mbox{im}\left(
T\right) .$ Show that $T$ is an isomorphism of $\mathbb{R}^{2}$ and $\mbox{im
}\left( T\right) $.
%\begin{hint}
%\end{hint}
\end{problem}


\begin{problem}\label{prb:10.84} In the above problem, find a $2\times 3$ matrix $A$ such that the
restriction of $A$ to $\mbox{im}\left( T\right) $ gives the same result as $
T^{-1}$ on $\mbox{im}\left( T\right) $. \textbf{Hint:\ }You might let $A$ be
such that
\begin{equation*}
A\left[
\begin{array}{c}
1 \\
1 \\
0
\end{array}
\right] =\left[
\begin{array}{c}
1 \\
0
\end{array}
\right] ,\ A\left[
\begin{array}{c}
0 \\
1 \\
1
\end{array}
\right] =\left[
\begin{array}{c}
0 \\
1
\end{array}
\right]
\end{equation*}
now find another vector $\vec{v}\in \mathbb{R}^{3}$ such that
\begin{equation*}
\left\{ \left[
\begin{array}{c}
1 \\
1 \\
0
\end{array}
\right] ,\left[
\begin{array}{c}
0 \\
1 \\
1
\end{array}
\right] ,\vec{v}\right\}
\end{equation*}
is a basis. You could pick
\begin{equation*}
\vec{v}=\left[
\begin{array}{c}
0 \\
0 \\
1
\end{array}
\right]
\end{equation*}
for example. Explain why this one works or one of your choice works. Then
you could define $A\vec{v}$ to equal some vector in $\mathbb{R}^{2}.$
Explain why there will be more than one such matrix $A$ which will deliver
the inverse isomorphism $T^{-1}$ on $\mbox{im}\left( T\right) $.
%\begin{hint}
%\end{hint}
\end{problem}


\begin{problem}\label{prb:10.85} Now let $V$ equal $\mbox{span}\left\{ \left[
\begin{array}{c}
1 \\
0 \\
1
\end{array}
\right] ,\left[
\begin{array}{c}
0 \\
1 \\
1
\end{array}
\right] \right\} $ and let $T:V\rightarrow W$ be a linear transformation
where
\begin{equation*}
W=\mbox{span}\left\{ \left[
\begin{array}{c}
1 \\
0 \\
1 \\
0
\end{array}
\right] ,\left[
\begin{array}{c}
0 \\
1 \\
1 \\
1
\end{array}
\right] \right\}
\end{equation*}
$\ $\ and
\begin{equation*}
T\left[
\begin{array}{c}
1 \\
0 \\
1
\end{array}
\right] =\left[
\begin{array}{c}
1 \\
0 \\
1 \\
0
\end{array}
\right] ,T\left[
\begin{array}{c}
0 \\
1 \\
1
\end{array}
\right] =\left[
\begin{array}{c}
0 \\
1 \\
1 \\
1
\end{array}
\right]
\end{equation*}
Explain why $T$ is an isomorphism. Determine a matrix $A$ which, when
multiplied on the left gives the same result as $T$ on $V$ and a matrix $B$
which delivers $T^{-1}$ on $W$. 
\begin{hint}You need to have
\begin{equation*}
A\left[
\begin{array}{cc}
1 & 0 \\
0 & 1 \\
1 & 1
\end{array}
\right] =\left[
\begin{array}{cc}
1 & 0 \\
0 & 1 \\
1 & 1 \\
0 & 1
\end{array}
\right]
\end{equation*}
Now enlarge $\left[
\begin{array}{c}
1 \\
0 \\
1
\end{array}
\right] ,\left[
\begin{array}{c}
0 \\
1 \\
1
\end{array}
\right] $ to obtain a basis for $\mathbb{R}^{3}$. You could add in $\left[
\begin{array}{c}
0 \\
0 \\
1
\end{array}
\right] $ for example, and then pick another vector in $\mathbb{R}^{4}$ and
let $A\left[
\begin{array}{c}
0 \\
0 \\
1
\end{array}
\right] $ equal this other vector. Then you would have
\begin{equation*}
A\left[
\begin{array}{ccc}
1 & 0 & 0 \\
0 & 1 & 0 \\
1 & 1 & 1
\end{array}
\right] =\left[
\begin{array}{ccc}
1 & 0 & 0 \\
0 & 1 & 0 \\
1 & 1 & 0 \\
0 & 1 & 1
\end{array}
\right]
\end{equation*}
This would involve picking for the new vector in $\mathbb{R}^{4}$ the vector
$\left[
\begin{array}{cccc}
0 & 0 & 0 & 1
\end{array}
\right] ^{T}.$ Then you could find $A$. You can do something similar to find
a matrix for $T^{-1}$ denoted as $B$.

\end{hint}
\end{problem}

\begin{problem}\label{prb:10.86}
Let $V=\mathbb{R}^{3}$ and let
\begin{equation*}
W=\mbox{span} \left( S \right),  \mbox{ where } S=\left\{ \left[
\begin{array}{r}
1 \\
-1 \\
1
\end{array}
\right] ,\left[
\begin{array}{r}
-2 \\
2 \\
-2
\end{array}
\right],\left[
\begin{array}{r}
-1 \\
1 \\
1
\end{array}
\right],\left[
\begin{array}{r}
1 \\
-1 \\
3
\end{array}
\right] \right\}
\end{equation*}
Find a basis of $W$ consisting of vectors in $S$.

\begin{hint}
In this case $\dim (W)=1$ and a basis for $W$ consisting of vectors in $S$ can be obtained by taking any (nonzero) vector from $S$.
\end{hint}
\end{problem}


\begin{problem}\label{prb:10.87}
 Let $T$ be a linear transformation given by
\[
T\left( \left[ \begin{array}{r}
x\\
y
\end{array}\right]\right) = \left[ \begin{array}{rrr}
1 &1  \\
1 & 1
\end{array}\right]
\left[ \begin{array}{r}
x\\
y
\end{array}\right]
\]
Find a basis for $\ker \left( T\right)$ and $\mbox{im} \left( T\right) $.

\begin{hint}
A basis for $\ker \left( T\right)$ is
$\left\{ \left[
\begin{array}{r}
1 \\
-1
\end{array}
\right] \right\}$
and a basis for $\mbox{im} \left( T\right)$ is
$\left\{ \left[
\begin{array}{r}
1 \\
1
\end{array}
\right] \right\}$. \\
There are many other possibilities for the specific bases, but in this case
$\dim \left( \ker \left( T\right) \right)=1 $ and $\dim \left( \mbox{im} \left( T\right) \right)=1$.
\end{hint}

\end{problem}


\begin{problem}\label{prb:10.88}
 Let $T$ be a linear transformation given by
\[
T\left( \left[ \begin{array}{r}
x\\
y
\end{array}\right]\right) = \left[ \begin{array}{rrr}
1 & 0  \\
1 & 1
\end{array}\right]
\left[ \begin{array}{r}
x\\
y
\end{array}\right]
\]
Find a basis for $\ker \left( T\right)$ and $\mbox{im}
\left( T\right) $.

\begin{hint}
In this case $\ker \left( T\right) =\{0\}$
and $\mbox{im} \left( T\right) = \mathbb{R}^2$ (pick any basis of $\mathbb{R}^2$).
\end{hint}

\end{problem}



\begin{problem}\label{prb:10.89}
Let $V=\mathbb{R}^{3}$ and let
\begin{equation*}
W=\mbox{span}\left\{ \left[
\begin{array}{r}
1 \\
1 \\
1
\end{array}
\right] ,\left[
\begin{array}{r}
-1 \\
2 \\
-1
\end{array}
\right] \right\}
\end{equation*}
Extend this basis of $W$ to a basis of $V$.

\begin{hint}
There are many possible such extensions, one is (how do we know?):
\begin{equation*}
\left\{ \left[
\begin{array}{r}
1 \\
1 \\
1
\end{array}
\right] ,\left[
\begin{array}{r}
-1 \\
2 \\
-1
\end{array}
\right] ,\left[
\begin{array}{r}
0  \\
0\\
1
\end{array}
\right]
\right\}
\end{equation*}
\end{hint}
\end{problem}

\begin{problem}\label{prb:10.90}
 Let $T$ be a linear transformation given by
\[
T \left[ \begin{array}{r}
x\\
y \\
z
\end{array}\right] = \left[ \begin{array}{rrr}
1 & 1 & 1 \\
1 & 1 & 1
\end{array}\right]
\left[ \begin{array}{r}
x\\
y \\
z
\end{array}\right]
\]
What is $\dim  ( \ker \left( T \right) )$?

\begin{hint}
We can easily see that $\dim  ( \mbox{im} \left( T \right) ) =1$, and thus
$\dim  ( \ker \left( T \right) ) = 3 - \dim  ( \mbox{im} \left( T \right) ) = 3- 1 = 2$.
\end{hint}
\end{problem}

\begin{problem}\label{prb:10.91} \label{exerlineartransf}
Consider the following functions which map $\mathbb{R}^{n}$ to $\mathbb{R}^{n}$.

\begin{enumerate}
\item $T$ multiplies the $j^{th}$ component of $\vec{x}$ by a nonzero
number $b.$

\item $T$ replaces the $i^{th}$ component of $\vec{x}$ with $b$ times the
$j^{th}$ component added to the $i^{th}$ component.

\item $T$ switches the $i^{th}$ and $j^{th}$ components.
\end{enumerate}

Show these functions are linear transformations and describe their matrices $A$ such that $T\left(\vec{x}\right) = A\vec{x}$.
\begin{hint}
\begin{enumerate}
\item The matrix of $T$ is the elementary matrix which multiplies
the $j^{th}$ diagonal entry of the identity matrix by $b$.
\item The matrix of $T$ is the
elementary matrix which takes $b$ times the $j^{th}$ row and adds to the $%
i^{th}$ row.
\item The matrix of $T$ is the elementary matrix which switches the $%
i^{th}$ and the $j^{th}$ rows where the two components are in the $i^{th}$
and $j^{th}$ positions.
\end{enumerate}
\end{hint}
\end{problem}

% \begin{problem}\label{prb:10.92} You are given a linear transformation $T:\mathbb{R}^{n}\rightarrow
% \mathbb{R}^{m}$ and you know that
% \begin{equation*}
% T\left(A_{i}\right)=B_{i}
% \end{equation*}
% where $\left[
% \begin{array}{ccc}
% A_{1} & \cdots & A_{n}
% \end{array}
% \right] ^{-1}$ exists. Show that the matrix of $T$ is of the form
% \begin{equation*}
% \left[
% \begin{array}{ccc}
% B_{1} & \cdots & B_{n}
% \end{array}
% \right] \left[
% \begin{array}{ccc}
% A_{1} & \cdots & A_{n}
% \end{array}
% \right] ^{-1}
% \end{equation*}
% \begin{hint}
% Suppose
% \[
% \left[
% \begin{array}{c}
% \vec{c}_{1}^{T} \\
% \vdots \\
% \vec{c}_{n}^{T}
% \end{array}
% \right] =\left[
% \begin{array}{ccc}
% \vec{a}_{1} & \cdots & \vec{a}_{n}
% \end{array}
% \right]^{-1}
% \]
% Thus $\vec{c}_{i}^{T}\vec{a}_{j}=\delta _{ij}$. Therefore,
% \begin{eqnarray*}
% \left[
% \begin{array}{ccc}
% \vec{b}_{1} & \cdots & \vec{b}_{n}
% \end{array}
% \right] \left[
% \begin{array}{ccc}
% \vec{a}_{1} & \cdots & \vec{a}_{n}
% \end{array}
% \right] ^{-1}\vec{a}_{i} &=&
% \left[
% \begin{array}{ccc}
% \vec{b}_{1} & \cdots & \vec{b}_{n}
% \end{array}
% \right] \left[
% \begin{array}{c}
% \vec{c}_{1}^{T} \\
% \vdots \\
% \vec{c}_{n}^{T}
% \end{array}
% \right] \vec{a}_{i} \\
% &=&\left[
% \begin{array}{ccc}
% \vec{b}_{1} & \cdots & \vec{b}_{n}
% \end{array}
% \right] \vec{e}_{i} \\
% &=&\vec{b}_{i}
% \end{eqnarray*}
% Thus $T\vec{a}_{i}=\left[
% \begin{array}{ccc}
% \vec{b}_{1} & \cdots & \vec{b}_{n}
% \end{array}
% \right] \left[
% \begin{array}{ccc}
% \vec{a}_{1} & \cdots & \vec{a}_{n}
% \end{array}
% \right] ^{-1}\vec{a}_{i} =  A\vec{a}_{i}.$ If $\vec{x}$ is
% arbitrary, then since the matrix $\left[
% \begin{array}{ccc}
% \vec{a}_{1} & \cdots & \vec{a}_{n}
% \end{array}
% \right] $ is invertible, there exists a unique $\vec{y}$ such that $
% \left[
% \begin{array}{ccc}
% \vec{a}_{1} & \cdots & \vec{a}_{n}
% \end{array}
% \right] \vec{y}=\vec{x}$ Hence
% \[
% T\vec{x}=T\left( \sum_{i=1}^{n}y_{i}\vec{a}_{i}\right)
% =\sum_{i=1}^{n}y_{i}T\vec{a}_{i}=\sum_{i=1}^{n}y_{i}A\vec{a}
% _{i}=A\left( \sum_{i=1}^{n}y_{i}\vec{a}_{i}\right) =A\vec{x}
% \]

% \end{hint}
% \end{problem}

\begin{problem}\label{prb:10.93} Suppose $T$ is a linear transformation such that
\begin{eqnarray*}
T\left[
\begin{array}{r}
1 \\
2 \\
-6
\end{array}
\right] &=&\left[
\begin{array}{r}
5 \\
1 \\
3
\end{array}
\right] \\
T\left[
\begin{array}{r}
-1 \\
-1 \\
5
\end{array}
\right] &=&\left[
\begin{array}{r}
1 \\
1 \\
5
\end{array}
\right] \\
T\left[
\begin{array}{r}
0 \\
-1 \\
2
\end{array}
\right] &=&\left[
\begin{array}{r}
5 \\
3 \\
-2
\end{array}
\right]
\end{eqnarray*}
Find the matrix of $T$. That is find $A$ such that $T(\vec{x})=A\vec{x}$. \vspace{1mm}
\begin{hint}
\[
\left[
\begin{array}{rrr}
5 & 1 & 5 \\
1 & 1 & 3 \\
3 & 5 & -2
\end{array}
\right] \left[
\begin{array}{ccc}
3 & 2 & 1 \\
2 & 2 & 1 \\
4 & 1 & 1
\end{array}
\right] =\left[
\begin{array}{ccc}
37 & 17 & 11 \\
17 & 7 & 5 \\
11 & 14 & 6
\end{array}
\right]
\]
\end{hint}
\end{problem}

\begin{problem}\label{prb:10.94} Suppose $T$ is a linear transformation such that
\begin{eqnarray*}
T\left[
\begin{array}{r}
1 \\
1 \\
-8
\end{array}
\right] &=&\left[
\begin{array}{r}
1 \\
3 \\
1
\end{array}
\right] \\
T\left[
\begin{array}{r}
-1 \\
0 \\
6
\end{array}
\right] &=&\left[
\begin{array}{r}
2 \\
4 \\
1
\end{array}
\right] \\
T\left[
\begin{array}{r}
0 \\
-1 \\
3
\end{array}
\right] &=&\left[
\begin{array}{r}
6 \\
1 \\
-1
\end{array}
\right]
\end{eqnarray*}
Find the matrix of $T$. That is find $A$ such that $T(\vec{x})=A\vec{x}$. \vspace{1mm}
\begin{hint}
\[
\left[
\begin{array}{rrr}
1 & 2 & 6 \\
3 & 4 & 1 \\
1 & 1 & -1
\end{array}
\right] \left[
\begin{array}{ccc}
6 & 3 & 1 \\
5 & 3 & 1 \\
6 & 2 & 1
\end{array}
\right] =\left[
\begin{array}{ccc}
52 & 21 & 9 \\
44 & 23 & 8 \\
5 & 4 & 1
\end{array}
\right]
\]
\end{hint}
\end{problem}

\begin{problem}\label{prb:10.95} Suppose $T$ is a linear transformation such that
\begin{eqnarray*}
T\left[
\begin{array}{r}
1 \\
3 \\
-7
\end{array}
\right] &=&\left[
\begin{array}{r}
-3 \\
1 \\
3
\end{array}
\right] \\
T\left[
\begin{array}{r}
-1 \\
-2 \\
6
\end{array}
\right] &=&\left[
\begin{array}{r}
1 \\
3 \\
-3
\end{array}
\right] \\
T\left[
\begin{array}{r}
0 \\
-1 \\
2
\end{array}
\right] &=&\left[
\begin{array}{r}
5 \\
3 \\
-3
\end{array}
\right]
\end{eqnarray*}
Find the matrix of $T$. That is find $A$ such that $T(\vec{x})=A\vec{x}$. \vspace{1mm}\vspace{1mm}
\begin{hint}
\[
\left[
\begin{array}{rrr}
-3 & 1 & 5 \\
1 & 3 & 3 \\
3 & -3 & -3
\end{array}
\right] \left[
\begin{array}{ccc}
2 & 2 & 1 \\
1 & 2 & 1 \\
4 & 1 & 1
\end{array}
\right] = \left[
\begin{array}{rrr}
15 & 1 & 3 \\
17 & 11 & 7 \\
-9 & -3 & -3
\end{array}
\right]
\]
\end{hint}
\end{problem}

\begin{problem}\label{prb:10.96} Suppose $T$ is a linear transformation such that
\begin{eqnarray*}
T\left[
\begin{array}{r}
1 \\
1 \\
-7
\end{array}
\right] &=&\left[
\begin{array}{r}
3 \\
3 \\
3
\end{array}
\right] \\
T\left[
\begin{array}{r}
-1 \\
0 \\
6
\end{array}
\right] &=&\left[
\begin{array}{r}
1 \\
2 \\
3
\end{array}
\right] \\
T\left[
\begin{array}{r}
0 \\
-1 \\
2
\end{array}
\right] &=&\left[
\begin{array}{r}
1 \\
3 \\
-1
\end{array}
\right]
\end{eqnarray*}
Find the matrix of $T$. That is find $A$ such that $T(\vec{x})=A\vec{x}$. \vspace{1mm}
\begin{hint}
\[
\left[
\begin{array}{rrr}
3 & 1 & 1 \\
3 & 2 & 3 \\
3 & 3 & -1
\end{array}
\right] \left[
\begin{array}{ccc}
6 & 2 & 1 \\
5 & 2 & 1 \\
6 & 1 & 1
\end{array}
\right] =\left[
\begin{array}{ccc}
29 & 9 & 5 \\
46 & 13 & 8 \\
27 & 11 & 5
\end{array}
\right]
\]
\end{hint}
\end{problem}

\begin{problem}\label{prb:10.97} Suppose $T$ is a linear transformation such that
\begin{eqnarray*}
T\left[
\begin{array}{r}
1 \\
2 \\
-18
\end{array}
\right] &=&\left[
\begin{array}{r}
5 \\
2 \\
5
\end{array}
\right] \\
T\left[
\begin{array}{r}
-1 \\
-1 \\
15
\end{array}
\right] &=&\left[
\begin{array}{r}
3 \\
3 \\
5
\end{array}
\right] \\
T\left[
\begin{array}{r}
0 \\
-1 \\
4
\end{array}
\right] &=&\left[
\begin{array}{r}
2 \\
5 \\
-2
\end{array}
\right]
\end{eqnarray*}
Find the matrix of $T$. That is find $A$ such that $T(\vec{x})=A\vec{x}$. \vspace{1mm}
\begin{hint}
\[
\left[
\begin{array}{rrr}
5 & 3 & 2 \\
2 & 3 & 5 \\
5 & 5 & -2
\end{array}
\right] \left[
\begin{array}{ccc}
11 & 4 & 1 \\
10 & 4 & 1 \\
12 & 3 & 1
\end{array}
\right] =\left[
\begin{array}{ccc}
109 & 38 & 10 \\
112 & 35 & 10 \\
81 & 34 & 8
\end{array}
\right]
\]
\end{hint}
\end{problem}


\begin{problem}\label{prb:10.98} Consider the following functions $T:\mathbb{R}^{3}\rightarrow \mathbb{R}^{2}$.
Show that each is a linear transformation and determine for each the matrix $A$ such that
$T(\vec{x})=A\vec{x}$.

\begin{enumerate}
\item $T\left[
\begin{array}{c}
x \\
y \\
z
\end{array}
\right] =\left[
\begin{array}{c}
x+2y+3z \\
2y-3x+z
\end{array}
\right] $

\item $T\left[
\begin{array}{c}
x \\
y \\
z
\end{array}
\right] =\left[
\begin{array}{c}
7x+2y+z \\
3x-11y+2z
\end{array}
\right] $

\item $T\left[
\begin{array}{c}
x \\
y \\
z
\end{array}
\right] =\left[
\begin{array}{c}
3x+2y+z \\
x+2y+6z
\end{array}
\right] $

\item $T\left[
\begin{array}{c}
x \\
y \\
z
\end{array}
\right] =\left[
\begin{array}{c}
2y-5x+z \\
x+y+z
\end{array}
\right] $
\end{enumerate}
%\begin{hint}
%\end{hint}
\end{problem}

\begin{problem}\label{prb:10.99} Consider the following functions $T:\mathbb{R}^{3}\rightarrow \mathbb{R}^{2}.$
Explain why each of these functions $T$ is not linear.

\begin{enumerate}
\item $T\left[
\begin{array}{c}
x \\
y \\
z
\end{array}
\right] =\left[
\begin{array}{c}
x+2y+3z+1 \\
2y-3x+z
\end{array}
\right] $

\item $T\left[
\begin{array}{c}
x \\
y \\
z
\end{array}
\right] =\left[
\begin{array}{c}
x+2y^{2}+3z \\
2y+3x+z
\end{array}
\right] $

\item $T\left[
\begin{array}{c}
x \\
y \\
z
\end{array}
\right] =\left[
\begin{array}{c}
\sin x+2y+3z \\
2y+3x+z
\end{array}
\right] $

\item $T\left[
\begin{array}{c}
x \\
y \\
z
\end{array}
\right] =\left[
\begin{array}{c}
x+2y+3z \\
2y+3x-\ln z
\end{array}
\right] $
\end{enumerate}
%\begin{hint}
%\end{hint}
\end{problem}


% \begin{problem}\label{prb:10.100} Suppose
% \begin{equation*}
% \left[
% \begin{array}{ccc}
% A_{1} & \cdots & A_{n}
% \end{array}
% \right] ^{-1}
% \end{equation*}
%  exists where each $A_{j}\in \mathbb{R}^{n}$ and let
% vectors  $\left\{ B_{1},\cdots ,B_{n}\right\} $ in $\mathbb{R}^{m}$ be given.
% Show that there \textbf{always }exists a linear
% transformation $T$ such that $T(A_{i})=B_{i}$.
% %\begin{hint}
% %\end{hint}
% \end{problem}


\begin{problem}\label{prb:10.101}  Find the matrix for $T\left(\vec{w} \right) = \mbox{proj}_{\vec{v}}\left( \vec{w}\right) $
where $\vec{v}=\left[
\begin{array}{rrr}
1 & -2 & 3
\end{array}
\right] ^{T}.$
\begin{hint}
 Recall that $\mbox{proj}_{\vec{u}}\left( \vec{v}\right) =\frac{\vec{v}\dotp\vec{u} }{\norm{\vec{u}}^{2}}\vec{u}$ and so the desired matrix
has $i^{th}$ column equal to $\mbox{proj}_{\vec{u}}\left( \vec{e}_{i}\right) .$ Therefore, the matrix desired is
\[
\frac{1}{14}\left[
\begin{array}{rrr}
1 & -2 & 3 \\
-2 & 4 & -6 \\
3 & -6 & 9
\end{array}
\right]
\]
\end{hint}
\end{problem}

\begin{problem}\label{prb:10.102}  Find the matrix for $T\left(\vec{w} \right) = \mbox{proj}_{\vec{v}}\left( \vec{w}\right) $
where $\vec{v}=\left[
\begin{array}{rrr}
1 & 5 & 3
\end{array}
\right] ^{T}.$
\begin{hint}
\[
\frac{1}{35}\left[
\begin{array}{rrr}
1 & 5 & 3 \\
5 & 25 & 15 \\
3 & 15 & 9
\end{array}
\right]
\]
\end{hint}
\end{problem}

\begin{problem}\label{prb:10.103} Find the matrix for $T\left(\vec{w} \right) = \mbox{proj}_{\vec{v}}\left( \vec{w}\right) $
where $\vec{v}=\left[
\begin{array}{rrr}
1 & 0 & 3
\end{array}
\right] ^{T}.$
\begin{hint}
\[
\frac{1}{10}\left[
\begin{array}{ccc}
1 & 0 & 3 \\
0 & 0 & 0 \\
3 & 0 & 9
\end{array}
\right]
\]
\end{hint}
\end{problem}





\begin{problem}\label{prb:10.104}
Let $B = \left\{ \left[ \begin{array}{r}
2 \\
-1
\end{array} \right], \left[ \begin{array}{r}
3 \\
2
\end{array} \right] \right\}$ be a basis of $\mathbb{R}^2$ and let $\vec{x} = \left[
\begin{array}{r}
5 \\
-7
\end{array}
\right]$ be a vector in $\mathbb{R}^2$. Find $[\vec{x}]_B$.
\end{problem}

\begin{problem}\label{prb:10.105}
Let $B = \left\{ \left[ \begin{array}{r}
1 \\
-1 \\
2
\end{array} \right], \left[ \begin{array}{r}
2 \\
1 \\
2
 \end{array} \right], \left[ \begin{array}{r}
-1 \\
0 \\
2
\end{array} \right] \right\}$
be a basis of $\mathbb{R}^3$ and let $\vec{x} = \left[
\begin{array}{r}
5 \\
-1 \\
4
\end{array}
\right]$ be a vector in $\mathbb{R}^2$. Find $[\vec{x}]_B$.
\begin{hint}
 $[\vec{x}]_B =
\left[ \begin{array}{r}
2 \\
1 \\
-1
 \end{array} \right] $.
\end{hint}
\end{problem}


\begin{problem}\label{prb:10.106}
Let $T: \mathbb{R}^2 \mapsto \mathbb{R}^2$ be a linear transformation defined by $T \left( \left[ \begin{array}{r}
a \\
b
\end{array} \right] \right) = \left[ \begin{array}{r}
a+b \\
a-b
\end{array} \right]$.

Consider the two bases
\[
B_1 = \left\{ \vec{v}_{1}, \vec{v}_{2} \right\} = \left\{ \left[ \begin{array}{r}
1 \\
0
\end{array}\right], \left[ \begin{array}{r}
-1 \\
1
\end{array}
\right]
\right\}
\]
 and
\[
B_2 = \left\{ \left[ \begin{array}{r}
1 \\
1
\end{array}
\right], \left[ \begin{array}{r}
1 \\
-1
\end{array}
\right]
\right\}
\]

Find the matrix $M_{B_2,B_1}$ of $T$ with respect to the ordered bases $B_1$ and $B_2$.
\begin{hint}
$
M_{B_{2} B_{1}} = \left[
\begin{array}{rr}
 1 & 0 \\
 -1 & 1
\end{array}
\right] $
\end{hint}
\end{problem}

\subsection*{Challenge Exercises}
\begin{problem}\label{prb:10.6} Denote by $\mathbb{R}^{\mathbb{N}}$ the set of real valued sequences.
For $\vec{a}\equiv \left\{ a_{n}\right\} _{n=1}^{\infty },\vec{b}\equiv
\left\{ b_{n}\right\} _{n=1}^{\infty }$ two of these, define their sum to be
given by
\begin{equation*}
\vec{a}+\vec{b} =  \left\{ a_{n}+b_{n}\right\} _{n=1}^{\infty }
\end{equation*}
and define scalar multiplication by
\begin{equation*}
c\vec{a}=\left\{ ca_{n}\right\} _{n=1}^{\infty }\text{ where }\vec{a}
=\left\{ a_{n}\right\} _{n=1}^{\infty }
\end{equation*}
Is this a special case of Problem \ref{prb:10.5}? Is this a vector space?
%\begin{hint}
%\end{hint}
\end{problem}

\begin{problem}\label{prb:10.7} Let $\mathbb{C}^{2}$ be the set of ordered pairs of complex numbers.
Define addition and scalar multiplication in the usual way.
\begin{equation*}
\left( z,w\right) +\left( \hat{z},\hat{w}\right) = \left( z+\hat{z},w+
\hat{w}\right) ,\ u\left( z,w\right) \equiv \left( uz,uw\right)
\end{equation*}
Here the scalars are from $\mathbb{C}$. Show this is a vector space.
%\begin{hint}
%\end{hint}
\end{problem}

\begin{problem}\label{prb:10.8} Let $V$ be the set of functions defined on a nonempty set which have
values in a vector space $W.$ Is this a vector space? Explain.
%\begin{hint}
%\end{hint}
\end{problem}

\begin{problem}\label{prb:10.21} Consider functions defined on $\left\{ 1,2,\cdots ,n\right\} $ having
values in $\mathbb{R}$. Explain how, if $V$ is the set of all such
functions, $V$ can be considered as $\mathbb{R}^{n}$.
\begin{hint}
Let $f\left( i\right) $ be the $i^{th}$ component of a vector $
\vec{x}\in \mathbb{R}^{n}$. Thus a typical element in $\mathbb{R}^{n}$ is $
\left( f\left( 1\right) ,\cdots ,f\left( n\right) \right) $.
\end{hint}
\end{problem}

\begin{problem}\label{prb:10.60} Let the field of scalars be $\mathbb{Q}$, the rational numbers and let
the vectors be of the form $a+b\sqrt{2}$ where $a,b$ are rational numbers.
Show that this collection of vectors is a vector space with field of scalars
$\mathbb{Q}$ and give a basis for this vector space.
\begin{hint}
When you add two of these you get one and when you multiply one of these by
a scalar, you get another one. A basis is $\left\{ 1,\sqrt{2}\right\} $. By
definition, the span of these gives the collection of vectors. Are they
independent? Say $a+b\sqrt{2}=0$ where $a,b$ are rational numbers. If $a\neq
0,$ then $b\sqrt{2}=-a$ which can't happen since $a$ is rational. If $b\neq
0,$ then $-a=b\sqrt{2}$ which again can't happen because on the left is a
rational number and on the right is an irrational. Hence both $a,b=0$ and so
this is a basis.
\end{hint}
\end{problem}

\begin{problem}\label{prb:10.61} Suppose $V$ is a finite dimensional vector space. Based on the
exchange theorem above, it was shown that any two bases have the same number
of vectors in them. Give a different proof of this fact using the earlier
material in the book. 
\begin{hint}Suppose $\left\{ \vec{x}_{1}\vec{
,\cdots ,x}_{n}\right\} $ and $\left\{ \vec{y}_{1}\vec{,\cdots ,y}
_{m}\right\} $ are two bases with $m<n.$ Then define
\begin{equation*}
\phi :\mathbb{R}^{n}\mapsto V,\psi :\mathbb{R}^{m}\mapsto V
\end{equation*}
by
\begin{equation*}
\phi \left( \vec{a}\right) = \sum_{k=1}^{n}a_{k}\vec{x}
_{k},\;\psi \left( \vec{b}\right) = \sum_{j=1}^{m}b_{j}\vec{y}_{j}
\end{equation*}
Consider the linear transformation, $\psi ^{-1}\circ \phi .$ Argue it is a
one to one and onto mapping from $\mathbb{R}^{n}$ to $\mathbb{R}^{m}.$ Now
consider a matrix of this linear transformation and its reduced row echelon form.
\end{hint}
\begin{hint}
This is obvious because
when you add two of these you get one and when you multiply one of these by
a scalar, you get another one. A basis is $\left\{ 1,\sqrt{2}\right\} $. By
definition, the span of these gives the collection of vectors. Are they
independent? Say $a+b\sqrt{2}=0$ where $a,b$ are rational numbers. If $a\neq
0,$ then $b\sqrt{2}=-a$ which can't happen since $a$ is rational. If $b\neq
0,$ then $-a=b\sqrt{2}$ which again can't happen because on the left is a
rational number and on the right is an irrational. Hence both $a,b=0$ and so
this is a basis.
\end{hint}
\end{problem}

% \begin{problem}\label{prb:10.9} Consider the space of $m\times n$ matrices with operation of addition
% and scalar multiplication defined the usual way. That is, if $A,B$ are two $
% m\times n$ matrices and $c$ a scalar,
% \begin{equation*}
% \left( A+B\right) _{ij}=A_{ij}+B_{ij},\ \left( cA\right) _{ij}\equiv c\left(
% A_{ij}\right)
% \end{equation*}
% %\begin{hint}
% %\end{hint}
% \end{problem}


\begin{problem}\label{prob:inner_prod_7}
If $p = p(x)$ and $q = q(x)$ are polynomials in $\mathbb{P}^{n}$, define
\begin{equation*}
\langle p, q \rangle = p(0)q(0) + p(1)q(1) + \dots + p(n)q(n)
\end{equation*}
Show that this is an inner product on $\mathbb{P}^{n}$.
\end{problem}

\begin{problem}\label{prob:inner_prod_8}
Let $\mathcal{D}_{n}$ denote the space of all functions from the set
$\{1, 2, 3, \dots, n\}$ to $\RR$ with pointwise addition and
scalar multiplication. Show
that $\langle\ , \rangle$ is an inner product on $\mathcal{D}_{n}$ if \newline $\langle\vec{f}, \vec{g}\rangle = f(1)g(1) + f(2)g(2) + \dots + f(n)g(n)$.

\begin{hint}
P1 and P2 are clear since $f(i)$ and $g(i)$ are real numbers.

\begin{flalign*}
\mbox{P3: } \langle f + g, h \rangle &= \sum_{i}(f + g)(i) \dotp h(i) &\\
&= \sum_{i}(f(i) + g(i)) \dotp h(i) &\\
&= \sum_{i}[f(i)h(i) + g(i)h(i)] &\\
&= \sum_{i}f(i)h(i) + \sum_{i}g(i)h(i) &\\
&= \langle f, h \rangle + \langle g, h \rangle. &\\
\mbox{P4: } \hspace{1em}\langle rf, g \rangle &= \sum_{i}(rf)(i) \dotp g(i) &\\
&= \sum_{i}rf(i) \dotp g(i) &\\
&= r \sum_{i}f(i) \dotp g(i) &\\
&= r\langle f, g \rangle &\\
\end{flalign*}

P5: If $ f \neq 0 $, then $\langle f, f \rangle = \displaystyle \sum_{i}f(i)^2 > 0 $ because some $f(i) \neq 0$.
\end{hint}
\end{problem}



\section*{Practice Problem Source}
Problems \ref{prb:10.1} to \ref{prb:10.106} come from Chapter 9 of Ken Kuttler's \href{https://open.umn.edu/opentextbooks/textbooks/a-first-course-in-linear-algebra-2017}{\it A First Course in Linear Algebra}. (CC-BY)

Ken Kuttler, {\it  A First Course in Linear Algebra}, Lyryx 2017, Open Edition, pp. 469--535.

Problems \ref{prob:inner_prod_7} to \ref{} come from Section 10.1 of Keith Nicholson's \href{https://open.umn.edu/opentextbooks/textbooks/linear-algebra-with-applications}{\it Linear Algebra with Applications}. (CC-BY-NC-SA)

W. Keith Nicholson, {\it Linear Algebra with Applications}, Lyryx 2018, Open Edition, pp. 528--530.

\end{document}