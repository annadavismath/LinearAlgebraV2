\documentclass{ximera}
%% You can put user macros here
%% However, you cannot make new environments

\listfiles

\graphicspath{
{./}
{./LTR-0070/}
{./VEC-0060/}
{./APP-0020/}
}

\usepackage{tikz}
\usepackage{tkz-euclide}
\usepackage{tikz-3dplot}
\usepackage{tikz-cd}
\usetikzlibrary{shapes.geometric}
\usetikzlibrary{arrows}
%\usetkzobj{all}
\pgfplotsset{compat=1.13} % prevents compile error.

%\renewcommand{\vec}[1]{\mathbf{#1}}
\renewcommand{\vec}{\mathbf}
\newcommand{\RR}{\mathbb{R}}
\newcommand{\dfn}{\textit}
\newcommand{\dotp}{\cdot}
\newcommand{\id}{\text{id}}
\newcommand\norm[1]{\left\lVert#1\right\rVert}
 
\newtheorem{general}{Generalization}
\newtheorem{initprob}{Exploration Problem}

\tikzstyle geometryDiagrams=[ultra thick,color=blue!50!black]

%\DefineVerbatimEnvironment{octave}{Verbatim}{numbers=left,frame=lines,label=Octave,labelposition=topline}



\usepackage{mathtools}


\title{Curve Fitting} \license{CC BY-NC-SA 4.0}

\begin{document}

\begin{abstract}
\end{abstract}
\maketitle

\begin{onlineOnly}
\section*{Curve Fitting}
\end{onlineOnly}

We know that two points determine a line.  Do you know how many points determine a parabola?  What about a cubic?  To address these questions we will start with an alternative way of finding an equation of a line.

\begin{exploration}\label{exp:curveFitLine}
Consider two points $A(-3, 2)$ and $B(1,-2)$.  We will find a function $f$ whose graph is a line that passes through these points.  We know that $f(x)=ax+b$ for some constants $a$ and $b$.  Because the graph of $f$ passes through $A$ and $B$, we must have the following:
$$\begin{matrix}
      f(-3)&=&a(-3)&+&b&=&2\\
      f(1)&=&a(1)&+&b&=&-2
    \end{matrix}$$
To solve for $a$ and $b$, we need to solve the following matrix equation:
$$\begin{bmatrix}\answer{-3} & \answer{1}\\\answer{1} & \answer{1}\end{bmatrix}\begin{bmatrix}a\\b\end{bmatrix}=\begin{bmatrix}\answer{2}\\\answer{-2}\end{bmatrix}$$
Solving the equation, we find that $a=-1$ and $b=-1$.  This gives us:
$$f(x)=-x-1$$

The GeoGebra interactive below shows two points $A$ and $B$, together with the matrix equation that produces function coefficients for the function whose graph passes through $A$ and $B$.  Drag the points around the plane to see how the matrix equation changes.

% \begin{pdfOnly}
% Access GeoGebra interactives through the online version of this text at 

% \href{https://ximera.osu.edu/oerlinalg}{https://ximera.osu.edu/oerlinalg}.
% \end{pdfOnly}

\begin{onlineOnly}
\begin{center}
\geogebra{aupuxe5j}{950}{650}
\end{center}
\end{onlineOnly}

From a purely formal standpoint, we observe that the matrix equation has the form:
$$\begin{bmatrix}x\text{-coordinate} & 1\\x\text{-coordinate} & 1\end{bmatrix}\begin{bmatrix}a\\b\end{bmatrix}=\begin{bmatrix}y\text{-coordinate}\\y\text{-coordinate}\end{bmatrix}$$
where each row corresponds to one point.
\end{exploration}

Now we are ready to move to quadratic, and higher degree polynomial functions.

Linear function in Exploration \ref{exp:curveFitLine} had two unknown coefficients that we needed to find in order to determine the function. Two points gave us a system of two equations and two unknowns.

A quadratic polynomial function, whose graph is a parabola, has the form:
$$f(x)=ax^2+bx+c$$
Three unknown coefficients will require three points to determine them.

\begin{exploration}\label{exp:curveFitParabola}
We will find a quadratic function of the form $f(x)=ax^2+bx+c$ whose graph passes through $A(-2,2)$, $B(0, -1)$ and $C(1, 5)$.
To do this, we need to find coefficients $a$, $b$ and $c$ such that
$$\begin{matrix}f(-2)&=&a(-2)^2&+&b(-2)&+&c&=&2\\f(0)&=&a(0)^2&+&b(0)&+&c&=&-1\\f(1)&=&a(1)^2&+&b(1)&+&c&=&5\end{matrix}$$

The following GeoGebra interactive shows points $A$, $B$, and $C$, together with the matrix equation, and its solution.

%     \begin{pdfOnly}
% Access GeoGebra interactives through the online version of this text at 

% \href{https://ximera.osu.edu/oerlinalg}{https://ximera.osu.edu/oerlinalg}.
% \end{pdfOnly}

\begin{onlineOnly}
\begin{center}
\geogebra{m6mbcykf}{950}{650}
\end{center}
\end{onlineOnly}

Drag the points around the plane to observe changes in the coefficient matrix.  Think geometrically to find locations of $A$, $B$ and $C$ such that
\begin{itemize}
    \item $a=b=0$; $c\neq 0$.
    \item $a=0$; $b, c\neq 0$.
\end{itemize}
\end{exploration}

\section*{Practice Problems}
\begin{problem}\label{prob:systemProblems}
Two GeoGebra screenshots are shown below:
    \begin{image}
         \includegraphics[height=1.5in]{pointsCoincide.jpg}\quad
         \includegraphics[height=1.5in]{verticalPoints.jpg}
\end{image}
In the first screenshot, points $A$ and $B$ coincide.  In the second screenshot, point $A$ is located directly above point $B$. 
In both cases, GeoGebra failed to produce a linear function whose graph passes through $A$ and $B$.

Based on what you know about functions and geometry, explain why the process fails for these two examples. How do your observations correspond to what happens from an algebraic standpoint?
\end{problem}





\end{document}