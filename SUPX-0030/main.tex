\documentclass{ximera}
%% You can put user macros here
%% However, you cannot make new environments

\listfiles

\graphicspath{
{./}
{./LTR-0070/}
{./VEC-0060/}
{./APP-0020/}
}

\usepackage{tikz}
\usepackage{tkz-euclide}
\usepackage{tikz-3dplot}
\usepackage{tikz-cd}
\usetikzlibrary{shapes.geometric}
\usetikzlibrary{arrows}
%\usetkzobj{all}
\pgfplotsset{compat=1.13} % prevents compile error.

%\renewcommand{\vec}[1]{\mathbf{#1}}
\renewcommand{\vec}{\mathbf}
\newcommand{\RR}{\mathbb{R}}
\newcommand{\dfn}{\textit}
\newcommand{\dotp}{\cdot}
\newcommand{\id}{\text{id}}
\newcommand\norm[1]{\left\lVert#1\right\rVert}
 
\newtheorem{general}{Generalization}
\newtheorem{initprob}{Exploration Problem}

\tikzstyle geometryDiagrams=[ultra thick,color=blue!50!black]

%\DefineVerbatimEnvironment{octave}{Verbatim}{numbers=left,frame=lines,label=Octave,labelposition=topline}



\usepackage{mathtools}


\author{Paul Zachlin \and Anna Davis} \title{Supplementary Exercises for Ch 3} \license{CC-BY 4.0}

\begin{document}

\begin{abstract}
\end{abstract}
\maketitle

\section*{Exercises for Ch 3}

\begin{problem}\label{prb:3.1} Find $-3
\begin{bmatrix}
5 \\
-1 \\
2 \\
-3
\end{bmatrix}
 +5
\begin{bmatrix}
-8 \\
2 \\
-3 \\
6
\end{bmatrix}.$
$$\begin{bmatrix}
\answer{-55} \\
\answer{13} \\
\answer{-21} \\
\answer{39}
\end{bmatrix}$$
\end{problem}

\begin{problem}\label{prb:3.2} Find $-7
\begin{bmatrix}
6 \\
0 \\
4 \\
-1
\end{bmatrix} +6
\begin{bmatrix}
-13 \\
-1 \\
1 \\
6
\end{bmatrix}.$
%\begin{hint}
%\end{hint}
\end{problem}


\begin{problem}\label{prb:3.3}
Decide whether
\begin{equation*}
\vec{v}= 
\begin{bmatrix}
4 \\
4 \\
-3
\end{bmatrix}
\end{equation*}
is a linear combination of the vectors

\begin{equation*}
\vec{u}_1 = 
\begin{bmatrix}
3 \\
1 \\
-1
\end{bmatrix}
\mbox{ and  }
\vec{u}_2 =
\begin{bmatrix}
2 \\
-2\\
1
\end{bmatrix}.
\end{equation*}

\begin{hint}
YES.  In fact, $\vec{v}=\answer{2}\vec{u}_1 +\answer{-1}\vec{u}_2$.
\end{hint}
\end{problem}


\begin{problem}\label{prb:3.4}
Decide whether
\begin{equation*}
\vec{v}= 
\begin{bmatrix}
4 \\
4 \\
4
\end{bmatrix}
\end{equation*}
is a linear combination of the vectors
\begin{equation*}
\vec{u}_1 = 
\begin{bmatrix}
3 \\
1 \\
-1
\end{bmatrix}
\mbox{ and  }
\vec{u}_2 =
\begin{bmatrix}
2 \\
-2\\
1
\end{bmatrix}.
\end{equation*}

\begin{hint}
The system
\begin{equation*}
\left[
\begin{array}{r}
4 \\
4 \\
4
\end{array}
\right]
=
a_1
\left[
\begin{array}{r}
3 \\
1 \\
-1
\end{array}
\right]
+a_2
\left[
\begin{array}{r}
2 \\
-2\\
1
\end{array}
\right]
\end{equation*}
has no solution.
\end{hint}
\end{problem}

\begin{problem}\label{prb:3.5} Here are some vectors.
\begin{equation*}
\left[
\begin{array}{r}
1 \\
1 \\
-2
\end{array}
\right] ,\left[
\begin{array}{r}
1 \\
2 \\
-2
\end{array}
\right] ,\left[
\begin{array}{r}
2 \\
7 \\
-4
\end{array}
\right] ,\left[
\begin{array}{r}
5 \\
7 \\
-10
\end{array}
\right] ,\left[
\begin{array}{r}
12 \\
17 \\
-24
\end{array}
\right]
\end{equation*}
Describe the span of these vectors as the span of as few vectors as possible.
%\begin{hint}
%\end{hint}
\end{problem}

\begin{problem}\label{prb:3.6} Here are some vectors.
\begin{equation*}
\left[
\begin{array}{r}
1 \\
2 \\
-2
\end{array}
\right] ,\left[
\begin{array}{r}
12 \\
29 \\
-24
\end{array}
\right] ,\left[
\begin{array}{r}
1 \\
3 \\
-2
\end{array}
\right] ,\left[
\begin{array}{r}
2 \\
9 \\
-4
\end{array}
\right] ,\left[
\begin{array}{r}
5 \\
12 \\
-10
\end{array}
\right] ,
\end{equation*}
Describe the span of these vectors as the span of as few vectors as possible.
%\begin{hint}
%\end{hint}
\end{problem}

\begin{problem}\label{prb:3.7} Here are some vectors.
\begin{equation*}
\left[
\begin{array}{r}
1 \\
2 \\
-2
\end{array}
\right] ,\left[
\begin{array}{r}
1 \\
3 \\
-2
\end{array}
\right] ,\left[
\begin{array}{r}
1 \\
-2 \\
-2
\end{array}
\right] ,\left[
\begin{array}{r}
-1 \\
0 \\
2
\end{array}
\right] ,\left[
\begin{array}{r}
1 \\
3 \\
-1
\end{array}
\right]
\end{equation*}
Describe the span of these vectors as the span of as few vectors as possible.
%\begin{hint}
%\end{hint}
\end{problem}

\begin{problem}\label{prb:3.8} Here are some vectors.
\begin{equation*}
\left[
\begin{array}{r}
1 \\
1 \\
-2
\end{array}
\right] ,\left[
\begin{array}{r}
1 \\
2 \\
-2
\end{array}
\right] ,\left[
\begin{array}{r}
1 \\
-3 \\
-2
\end{array}
\right] ,\left[
\begin{array}{r}
-1 \\
1 \\
2
\end{array}
\right]
\end{equation*}
Now here is another vector:\
\begin{equation*}
\left[
\begin{array}{r}
1 \\
2 \\
-1
\end{array}
\right]
\end{equation*}
Is this vector in the span of the first four vectors? If it is, exhibit a
linear combination of the first four vectors which equals this vector, using
as few vectors as possible in the linear combination.
%\begin{hint}
%\end{hint}
\end{problem}

\begin{problem}\label{prb:3.9} Here are some vectors.
\begin{equation*}
\left[
\begin{array}{r}
1 \\
1 \\
-2
\end{array}
\right] ,\left[
\begin{array}{r}
1 \\
2 \\
-2
\end{array}
\right] ,\left[
\begin{array}{r}
1 \\
-3 \\
-2
\end{array}
\right] ,\left[
\begin{array}{r}
-1 \\
1 \\
2
\end{array}
\right]
\end{equation*}
Now here is another vector:\
\begin{equation*}
\left[
\begin{array}{r}
2 \\
-3 \\
-4
\end{array}
\right]
\end{equation*}
Is this vector in the span of the first four vectors? If it is, exhibit a
linear combination of the first four vectors which equals this vector, using
as few vectors as possible in the linear combination.
%\begin{hint}
%\end{hint}
\end{problem}

\begin{problem}\label{prb:3.10} Here are some vectors.
\begin{equation*}
\left[
\begin{array}{r}
1 \\
1 \\
-2
\end{array}
\right] ,\left[
\begin{array}{r}
1 \\
2 \\
-2
\end{array}
\right] ,\left[
\begin{array}{r}
1 \\
-3 \\
-2
\end{array}
\right] ,\left[
\begin{array}{r}
1 \\
2 \\
-1
\end{array}
\right]
\end{equation*}
Now here is another vector:\
\begin{equation*}
\left[
\begin{array}{r}
1 \\
9 \\
1
\end{array}
\right]
\end{equation*}
Is this vector in the span of the first four vectors? If it is, exhibit a
linear combination of the first four vectors which equals this vector, using
as few vectors as possible in the linear combination.
%\begin{hint}
%\end{hint}
\end{problem}

\begin{problem}\label{prb:3.11} Here are some vectors.
\begin{equation*}
\left[
\begin{array}{r}
1 \\
-1 \\
-2
\end{array}
\right] ,\left[
\begin{array}{r}
1 \\
0 \\
-2
\end{array}
\right] ,\left[
\begin{array}{r}
1 \\
-5 \\
-2
\end{array}
\right] ,\left[
\begin{array}{r}
-1 \\
5 \\
2
\end{array}
\right]
\end{equation*}
Now here is another vector:\
\begin{equation*}
\left[
\begin{array}{r}
1 \\
1 \\
-1
\end{array}
\right]
\end{equation*}
Is this vector in the span of the first four vectors? If it is, exhibit a
linear combination of the first four vectors which equals this vector, using
as few vectors as possible in the linear combination.
%\begin{hint}
%\end{hint}
\end{problem}

\begin{problem}\label{prb:3.12} Here are some vectors.
\begin{equation*}
\left[
\begin{array}{r}
1 \\
-1 \\
-2
\end{array}
\right] ,\left[
\begin{array}{r}
1 \\
0 \\
-2
\end{array}
\right] ,\left[
\begin{array}{r}
1 \\
-5 \\
-2
\end{array}
\right] ,\left[
\begin{array}{r}
-1 \\
5 \\
2
\end{array}
\right]
\end{equation*}
Now here is another vector:\
\begin{equation*}
\left[
\begin{array}{r}
1 \\
1 \\
-1
\end{array}
\right]
\end{equation*}
Is this vector in the span of the first four vectors? If it is, exhibit a
linear combination of the first four vectors which equals this vector, using
as few vectors as possible in the linear combination.
%\begin{hint}
%\end{hint}
\end{problem}

\begin{problem}\label{prb:3.13} Here are some vectors.
\begin{equation*}
\left[
\begin{array}{r}
1 \\
0 \\
-2
\end{array}
\right] ,\left[
\begin{array}{r}
1 \\
1 \\
-2
\end{array}
\right] ,\left[
\begin{array}{r}
2 \\
-2 \\
-3
\end{array}
\right] ,\left[
\begin{array}{r}
-1 \\
4 \\
2
\end{array}
\right]
\end{equation*}
Now here is another vector:\
\begin{equation*}
\left[
\begin{array}{r}
-1 \\
-4 \\
2
\end{array}
\right]
\end{equation*}
Is this vector in the span of the first four vectors? If it is, exhibit a
linear combination of the first four vectors which equals this vector, using
as few vectors as possible in the linear combination.
%\begin{hint}
%\end{hint}
\end{problem}


\begin{problem}\label{prb:3.14} Suppose $\left\{ \vec{x}_{1},\ldots ,\vec{x}_{k}\right\} $ is a
set of vectors from $\RR^{n}.$ Show that $\vec{0}$ is in $\mbox{
span}\left\{ \vec{x}_{1},\ldots ,\vec{x}_{k}\right\} .$
\begin{hint}
$\sum_{i=1}^{k}0\vec{x}_{k}=\vec{0}$
\end{hint}
\end{problem}

\begin{problem}\label{prb:3.15} Are the following vectors linearly independent? If they are, explain
why and if they are not, exhibit one of them as a linear combination of the
others. Also give a linearly independent set of vectors which has the same
span as the given vectors.
\begin{equation*}
\left[
\begin{array}{r}
1 \\
3 \\
-1 \\
1
\end{array}
\right] ,\left[
\begin{array}{r}
1 \\
4 \\
-1 \\
1
\end{array}
\right] ,\left[
\begin{array}{r}
1 \\
4 \\
0 \\
1
\end{array}
\right] ,\left[
\begin{array}{r}
1 \\
10 \\
2 \\
1
\end{array}
\right]
\end{equation*}
%\begin{hint}
%\end{hint}
\end{problem}

\begin{problem}\label{prb:3.16} Are the following vectors linearly independent? If they are, explain
why and if they are not, exhibit one of them as a linear combination of the
others. Also give a linearly independent set of vectors which has the same
span as the given vectors.
\begin{equation*}
\left[
\begin{array}{r}
-1 \\
-2 \\
2 \\
3
\end{array}
\right] ,\left[
\begin{array}{r}
-3 \\
-4 \\
3 \\
3
\end{array}
\right] ,\left[
\begin{array}{r}
0 \\
-1 \\
4 \\
3
\end{array}
\right] ,\left[
\begin{array}{r}
0 \\
-1 \\
6 \\
4
\end{array}
\right]
\end{equation*}
%\begin{hint}
%\end{hint}
\end{problem}

\begin{problem}\label{prb:3.17} Are the following vectors linearly independent? If they are, explain
why and if they are not, exhibit one of them as a linear combination of the
others. Also give a linearly independent set of vectors which has the same
span as the given vectors.
\begin{equation*}
\left[
\begin{array}{r}
1 \\
5 \\
-2 \\
1
\end{array}
\right] ,\left[
\begin{array}{r}
1 \\
6 \\
-3 \\
1
\end{array}
\right] ,\left[
\begin{array}{r}
-1 \\
-4 \\
1 \\
-1
\end{array}
\right] ,\left[
\begin{array}{r}
1 \\
6 \\
-2 \\
1
\end{array}
\right]
\end{equation*}
%\begin{hint}
%\end{hint}
\end{problem}

\begin{problem}\label{prb:3.18} Are the following vectors linearly independent? If they are, explain
why and if they are not, exhibit one of them as a linear combination of the
others. Also give a linearly independent set of vectors which has the same
span as the given vectors.
\begin{equation*}
\left[
\begin{array}{r}
1 \\
-1 \\
3 \\
1
\end{array}
\right] ,\left[
\begin{array}{r}
1 \\
6 \\
34 \\
1
\end{array}
\right] ,\left[
\begin{array}{r}
1 \\
0 \\
7 \\
1
\end{array}
\right] ,\left[
\begin{array}{r}
1 \\
0 \\
8 \\
1
\end{array}
\right]
\end{equation*}
%\begin{hint}
%\end{hint}
\end{problem}

\begin{problem}\label{prb:3.19} Are the following vectors linearly independent? If they are, explain
why and if they are not, exhibit one of them as a linear combination of the
others.
\begin{equation*}
\left[
\begin{array}{r}
1 \\
3 \\
-1 \\
1
\end{array}
\right] ,\left[
\begin{array}{r}
1 \\
4 \\
-1 \\
1
\end{array}
\right] ,\left[
\begin{array}{r}
-3 \\
-10 \\
3 \\
-3
\end{array}
\right] ,\left[
\begin{array}{r}
1 \\
4 \\
0 \\
1
\end{array}
\right]
\end{equation*}
%\begin{hint}
%\end{hint}
\end{problem}

\begin{problem}\label{prb:3.20} Are the following vectors linearly independent? If they are, explain
why and if they are not, exhibit one of them as a linear combination of the
others. Also give a linearly independent set of vectors which has the same
span as the given vectors.
\begin{equation*}
\left[
\begin{array}{r}
1 \\
3 \\
-3 \\
1
\end{array}
\right] ,\left[
\begin{array}{r}
1 \\
4 \\
-5 \\
1
\end{array}
\right] ,\left[
\begin{array}{r}
1 \\
4 \\
-4 \\
1
\end{array}
\right] ,\left[
\begin{array}{r}
1 \\
10 \\
-14 \\
1
\end{array}
\right]
\end{equation*}
%\begin{hint}
%\end{hint}
\end{problem}

\begin{problem}\label{prb:3.21} Are the following vectors linearly independent? If they are, explain
why and if they are not, exhibit one of them as a linear combination of the
others. Also give a linearly independent set of vectors which has the same
span as the given vectors.
\begin{equation*}
\left[
\begin{array}{r}
1 \\
0 \\
3 \\
1
\end{array}
\right] ,\left[
\begin{array}{r}
1 \\
1 \\
8 \\
1
\end{array}
\right] ,\left[
\begin{array}{r}
1 \\
7 \\
34 \\
1
\end{array}
\right] ,\left[
\begin{array}{r}
1 \\
1 \\
7 \\
1
\end{array}
\right]
\end{equation*}
%\begin{hint}
%\end{hint}
\end{problem}

\begin{problem}\label{prb:3.22} Are the following vectors linearly independent? If they are, explain
why and if they are not, exhibit one of them as a linear combination of the
others. Also give a linearly independent set of vectors which has the same
span as the given vectors.
\begin{equation*}
\left[
\begin{array}{r}
1 \\
4 \\
-2 \\
1
\end{array}
\right] ,\left[
\begin{array}{r}
1 \\
5 \\
-3 \\
1
\end{array}
\right] ,\left[
\begin{array}{r}
1 \\
7 \\
-5 \\
1
\end{array}
\right] ,\left[
\begin{array}{r}
1 \\
5 \\
-2 \\
1
\end{array}
\right]
\end{equation*}
%\begin{hint}
%\end{hint}
\end{problem}

\begin{problem}\label{prb:3.23} Are the following vectors linearly independent? If they are, explain
why and if they are not, exhibit one of them as a linear combination of the
others.
\begin{equation*}
\left[
\begin{array}{r}
1 \\
2 \\
2 \\
-4
\end{array}
\right] ,\left[
\begin{array}{r}
3 \\
4 \\
1 \\
-4
\end{array}
\right] ,\left[
\begin{array}{r}
0 \\
-1 \\
0 \\
4
\end{array}
\right] ,\left[
\begin{array}{r}
0 \\
-1 \\
-2 \\
5
\end{array}
\right]
\end{equation*}
%\begin{hint}
%\end{hint}
\end{problem}

\begin{problem}\label{prb:3.24} Are the following vectors linearly independent? If they are, explain
why and if they are not, exhibit one of them as a linear combination of the
others. Also give a linearly independent set of vectors which has the same
span as the given vectors.
\begin{equation*}
\left[
\begin{array}{r}
2 \\
3 \\
1 \\
-3
\end{array}
\right] ,\left[
\begin{array}{r}
-5 \\
-6 \\
0 \\
3
\end{array}
\right] ,\left[
\begin{array}{r}
-1 \\
-2 \\
1 \\
3
\end{array}
\right] ,\left[
\begin{array}{r}
-1 \\
-2 \\
0 \\
4
\end{array}
\right]
\end{equation*}
%\begin{hint}
%\end{hint}
\end{problem}

\begin{problem}\label{prb:3.25} Here are some vectors in $\RR^{4}$.
\begin{equation*}
\left[
\begin{array}{r}
1 \\
1 \\
-1 \\
1
\end{array}
\right] ,\left[
\begin{array}{r}
1 \\
2 \\
-1 \\
1
\end{array}
\right] ,\left[
\begin{array}{r}
1 \\
-2 \\
-1 \\
1
\end{array}
\right] ,\left[
\begin{array}{r}
1 \\
2 \\
0 \\
1
\end{array}
\right] ,\left[
\begin{array}{r}
1 \\
-1 \\
-1 \\
1
\end{array}
\right]
\end{equation*}
These vectors can't possibly be linearly independent. Tell why. Next obtain a
linearly independent subset of these vectors which has the same span as
these vectors. In other words, find a basis for the span of these vectors.
%\begin{hint}
%\end{hint}
\end{problem}

\begin{problem}\label{prb:3.26} Here are some vectors in $\RR^{4}$.
\begin{equation*}
\left[
\begin{array}{r}
1 \\
2 \\
-2 \\
1
\end{array}
\right] ,\left[
\begin{array}{r}
1 \\
3 \\
-3 \\
1
\end{array}
\right] ,\left[
\begin{array}{r}
1 \\
3 \\
-2 \\
1
\end{array}
\right] ,\left[
\begin{array}{r}
4 \\
3 \\
-1 \\
4
\end{array}
\right] ,\left[
\begin{array}{r}
1 \\
3 \\
-2 \\
1
\end{array}
\right]
\end{equation*}
These vectors can't possibly be linearly independent. Tell why. Next obtain a
linearly independent subset of these vectors which has the same span as
these vectors. In other words, find a basis for the span of these vectors.
%\begin{hint}
%\end{hint}
\end{problem}

\begin{problem}\label{prb:3.27} Here are some vectors in $\RR^{4}$.
\begin{equation*}
\left[
\begin{array}{r}
1 \\
1 \\
0 \\
1
\end{array}
\right] ,\left[
\begin{array}{r}
1 \\
2 \\
1 \\
1
\end{array}
\right] ,\left[
\begin{array}{r}
1 \\
-2 \\
-3 \\
1
\end{array}
\right] ,\left[
\begin{array}{r}
2 \\
-5 \\
-7 \\
2
\end{array}
\right] ,\left[
\begin{array}{r}
1 \\
2 \\
2 \\
1
\end{array}
\right]
\end{equation*}
These vectors can't possibly be linearly independent. Tell why. Next obtain a
linearly independent subset of these vectors which has the same span as
these vectors. In other words, find a basis for the span of these vectors.
%\begin{hint}
%\end{hint}
\end{problem}

\begin{problem}\label{prb:3.28} Here are some vectors in $\RR^{4}$.
\begin{equation*}
\left[
\begin{array}{r}
1 \\
2 \\
-2 \\
1
\end{array}
\right] ,\left[
\begin{array}{r}
1 \\
3 \\
-3 \\
1
\end{array}
\right] ,\left[
\begin{array}{r}
1 \\
-1 \\
1 \\
1
\end{array}
\right] ,\left[
\begin{array}{r}
2 \\
-3 \\
3 \\
2
\end{array}
\right] ,\left[
\begin{array}{r}
1 \\
3 \\
-2 \\
1
\end{array}
\right]
\end{equation*}
These vectors can't possibly be linearly independent. Tell why. Next obtain a
linearly independent subset of these vectors which has the same span as
these vectors. In other words, find a basis for the span of these vectors.
%\begin{hint}
%\end{hint}
\end{problem}

\begin{problem}\label{prb:3.29} Here are some vectors in $\RR^{4}$.
\begin{equation*}
\left[
\begin{array}{r}
1 \\
4 \\
-2 \\
1
\end{array}
\right] ,\left[
\begin{array}{r}
1 \\
5 \\
-3 \\
1
\end{array}
\right] ,\left[
\begin{array}{r}
1 \\
5 \\
-2 \\
1
\end{array}
\right] ,\left[
\begin{array}{r}
4 \\
11 \\
-1 \\
4
\end{array}
\right] ,\left[
\begin{array}{r}
1 \\
5 \\
-2 \\
1
\end{array}
\right]
\end{equation*}
These vectors can't possibly be linearly independent. Tell why. Next obtain a
linearly independent subset of these vectors which has the same span as
these vectors. In other words, find a basis for the span of these vectors.
%\begin{hint}
%\end{hint}
\end{problem}

\begin{problem}\label{prb:3.30} Here are some vectors in $\RR^{4}$.
\begin{equation*}
\left[
\begin{array}{r}
1 \\
3 \\
-1 \\
1
\end{array}
\right] ,\left[
\begin{array}{r}
-\frac{3}{2} \\
-\frac{9}{2} \\
\frac{3}{2} \\
-\frac{3}{2}
\end{array}
\right] ,\left[
\begin{array}{r}
1 \\
4 \\
-1 \\
1
\end{array}
\right] ,\left[
\begin{array}{r}
2 \\
-1 \\
-2 \\
2
\end{array}
\right] ,\left[
\begin{array}{r}
1 \\
4 \\
0 \\
1
\end{array}
\right]
\end{equation*}
These vectors can't possibly be linearly independent. Tell why. Next obtain a
linearly independent subset of these vectors which has the same span as
these vectors. In other words, find a basis for the span of these vectors.
%\begin{hint}
%\end{hint}
\end{problem}


\begin{problem}\label{prb:3.31} Here are some vectors in $\RR^{4}$.
\begin{equation*}
\left[
\begin{array}{r}
1 \\
3 \\
-1 \\
1
\end{array}
\right] ,\left[
\begin{array}{r}
1 \\
4 \\
-1 \\
1
\end{array}
\right] ,\left[
\begin{array}{r}
1 \\
0 \\
-1 \\
1
\end{array}
\right] ,\left[
\begin{array}{r}
2 \\
-1 \\
-2 \\
2
\end{array}
\right] ,\left[
\begin{array}{r}
1 \\
4 \\
0 \\
1
\end{array}
\right]
\end{equation*}
These vectors can't possibly be linearly independent. Tell why. Next obtain a
linearly independent subset of these vectors which has the same span as
these vectors. In other words, find a basis for the span of these vectors.
%\begin{hint}
%\end{hint}
\end{problem}

\begin{problem}\label{prb:3.32} Here are some vectors in $\RR^{4}$.
\begin{equation*}
\left[
\begin{array}{r}
1 \\
4 \\
-2 \\
1
\end{array}
\right] ,\left[
\begin{array}{r}
1 \\
5 \\
-3 \\
1
\end{array}
\right] ,\left[
\begin{array}{r}
1 \\
1 \\
1 \\
1
\end{array}
\right] ,\left[
\begin{array}{r}
2 \\
1 \\
3 \\
2
\end{array}
\right] ,\left[
\begin{array}{r}
1 \\
5 \\
-2 \\
1
\end{array}
\right]
\end{equation*}
These vectors can't possibly be linearly independent. Tell why. Next obtain a
linearly independent subset of these vectors which has the same span as
these vectors. In other words, find a basis for the span of these vectors.
%\begin{hint}
%\end{hint}
\end{problem}

\begin{problem}\label{prb:3.33} Here are some vectors in $\RR^{4}$.
\begin{equation*}
\left[
\begin{array}{r}
1 \\
-1 \\
3 \\
1
\end{array}
\right] ,\left[
\begin{array}{r}
1 \\
0 \\
7 \\
1
\end{array}
\right] ,\left[
\begin{array}{r}
1 \\
0 \\
8 \\
1
\end{array}
\right] ,\left[
\begin{array}{r}
4 \\
-9 \\
-6 \\
4
\end{array}
\right] ,\left[
\begin{array}{r}
1 \\
0 \\
8 \\
1
\end{array}
\right]
\end{equation*}
These vectors can't possibly be linearly independent. Tell why. Next obtain a
linearly independent subset of these vectors which has the same span as
these vectors. In other words, find a basis for the span of these vectors.
%\begin{hint}
%\end{hint}
\end{problem}

\begin{problem}\label{prb:3.34} Here are some vectors in $\RR^{4}$.
\begin{equation*}
\left[
\begin{array}{r}
1 \\
-1 \\
-1 \\
1
\end{array}
\right] ,\left[
\begin{array}{r}
-3 \\
3 \\
3 \\
-3
\end{array}
\right] ,\left[
\begin{array}{r}
1 \\
0 \\
-1 \\
1
\end{array}
\right] ,\left[
\begin{array}{r}
2 \\
-9 \\
-2 \\
2
\end{array}
\right] ,\left[
\begin{array}{r}
1 \\
0 \\
0 \\
1
\end{array}
\right]
\end{equation*}
These vectors can't possibly be linearly independent. Tell why. Next obtain a
linearly independent subset of these vectors which has the same span as
these vectors. In other words, find a basis for the span of these vectors.
%\begin{hint}
%\end{hint}
\end{problem}

\begin{problem}\label{prb:3.35} Here are some vectors in $\RR^{4}$.
\begin{equation*}
\left[
\begin{array}{r}
1 \\
b+1 \\
a \\
1
\end{array}
\right] ,\left[
\begin{array}{r}
3 \\
3b+3 \\
3a \\
3
\end{array}
\right] ,\left[
\begin{array}{r}
1 \\
b+2 \\
2a+1 \\
1
\end{array}
\right] ,\left[
\begin{array}{r}
2 \\
2b-5 \\
-5a-7 \\
2
\end{array}
\right] ,\left[
\begin{array}{r}
1 \\
b+2 \\
2a+2 \\
1
\end{array}
\right]
\end{equation*}
These vectors can't possibly be linearly independent. Tell why. Next obtain a
linearly independent subset of these vectors which has the same span as
these vectors. In other words, find a basis for the span of these vectors.
%\begin{hint}
%\end{hint}
\end{problem}







\section*{Text Source} This section was adapted from the second part of Section 8.1 of Keith Nicholson's \href{https://open.umn.edu/opentextbooks/textbooks/linear-algebra-with-applications}{\it Linear Algebra with Applications}. (CC-BY-NC-SA)

W. Keith Nicholson, {\it Linear Algebra with Applications}, Lyryx 2018, Open Edition, p. 415 

%\section*{Example Source}
Example \ref{ex:OrthogDecomp}  was adapted from Example 4.148  of Ken Kuttler's \href{https://open.umn.edu/opentextbooks/textbooks/a-first-course-in-linear-algebra-2017}{\it A First Course in Linear Algebra}. (CC-BY)

Ken Kuttler, {\it  A First Course in Linear Algebra}, Lyryx 2017, Open Edition, p. 249. 

\end{document}