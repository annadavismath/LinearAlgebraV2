\documentclass{ximera}
%% You can put user macros here
%% However, you cannot make new environments

\listfiles

\graphicspath{
{./}
{./LTR-0070/}
{./VEC-0060/}
{./APP-0020/}
}

\usepackage{tikz}
\usepackage{tkz-euclide}
\usepackage{tikz-3dplot}
\usepackage{tikz-cd}
\usetikzlibrary{shapes.geometric}
\usetikzlibrary{arrows}
%\usetkzobj{all}
\pgfplotsset{compat=1.13} % prevents compile error.

%\renewcommand{\vec}[1]{\mathbf{#1}}
\renewcommand{\vec}{\mathbf}
\newcommand{\RR}{\mathbb{R}}
\newcommand{\dfn}{\textit}
\newcommand{\dotp}{\cdot}
\newcommand{\id}{\text{id}}
\newcommand\norm[1]{\left\lVert#1\right\rVert}
 
\newtheorem{general}{Generalization}
\newtheorem{initprob}{Exploration Problem}

\tikzstyle geometryDiagrams=[ultra thick,color=blue!50!black]

%\DefineVerbatimEnvironment{octave}{Verbatim}{numbers=left,frame=lines,label=Octave,labelposition=topline}



\usepackage{mathtools}


\author{Paul Zachlin \and Anna Davis} \title{Supplementary Exercises for Ch 2} \license{CC-BY 4.0}

\begin{document}

\begin{abstract}
\end{abstract}
\maketitle

\section*{Exercises for Ch 2 Systems of Linear Equations}

\begin{problem}\label{prb:2.2}
Graphically, find the point of intersection of the two lines $
3x+y=3$ and $x+2y=1.$ That is, graph each line
and see where they intersect.

 $$x=\answer{1},y=\answer{0}$$
\end{problem}

\begin{problem}\label{prb:2.3} You have a system of $k$ equations in two variables, $k\geq 2$.
Explain the geometric significance of

\begin{enumerate}
\item No solution.

\item A unique solution.

\item An infinite number of solutions.
\end{enumerate}

%\begin{hint}
%\end{hint}
\end{problem}

\begin{problem}\label{prb:2.4}
\label{Chapter1Q8}Consider the following augmented matrix in which $\ast $ denotes an
arbitrary number and $\blacksquare $ denotes a nonzero number. Determine
whether the given augmented matrix is consistent. If consistent, is the
solution unique?
\begin{equation*}
\left[
\begin{array}{ccccc|c}
\blacksquare & \ast & \ast & \ast & \ast & \ast \\
0 & \blacksquare & \ast & \ast & 0 & \ast \\
0 & 0 & \blacksquare & \ast & \ast & \ast \\
0 & 0 & 0 & 0 & \blacksquare & \ast
\end{array}
\right] 
\end{equation*}

The solution \wordChoice{\choice[correct]{exists} \choice{does not exist}} and is \wordChoice{\choice{unique} \choice[correct]{not unique}}.
\end{problem}

\begin{problem}\label{prb:2.5}
Consider the following augmented matrix in which $\ast $ denotes an arbitrary
number and $\blacksquare $ denotes a nonzero number. Determine whether the
given augmented matrix is consistent. If consistent, is the solution unique?
\begin{equation*}
\left[
\begin{array}{ccc|c}
\blacksquare & \ast & \ast & \ast \\
0 & \blacksquare & \ast & \ast \\
0 & 0 & \blacksquare & \ast
\end{array}
\right]
\end{equation*}
The solution \wordChoice{\choice[correct]{exists} \choice{does not exist}} and is \wordChoice{[correct]\choice{unique} \choice{not unique}}.
\end{problem}


\begin{problem}\label{prb:2.6}
Consider the following augmented matrix in which $\ast $ denotes an arbitrary
number and $\blacksquare $ denotes a nonzero number. Determine whether the
given augmented matrix is consistent. If consistent, is the solution unique?
\begin{equation*}
\left[
\begin{array}{ccccc|c}
\blacksquare & \ast & \ast & \ast & \ast & \ast \\
0 & \blacksquare & 0 & \ast & 0 & \ast \\
0 & 0 & 0 & \blacksquare & \ast & \ast \\
0 & 0 & 0 & 0 & \blacksquare & \ast
\end{array}
\right]
\end{equation*}
The solution \wordChoice{\choice[correct]{exists} \choice{does not exist}} and is \wordChoice{\choice{unique} \choice[correct]{not unique}}.
\end{problem}

\begin{problem}\label{prb:2.7}
Consider the following augmented matrix in which $\ast $ denotes an arbitrary
number and $\blacksquare $ denotes a nonzero number. Determine whether the
given augmented matrix is consistent. If consistent, is the solution unique?
\begin{equation*}
\left[
\begin{array}{ccccc|c}
\blacksquare & \ast & \ast & \ast & \ast & \ast \\
0 & \blacksquare & \ast & \ast & 0 & \ast \\
0 & 0 & 0 & 0 & \blacksquare & 0 \\
0 & 0 & 0 & 0 & \ast & \blacksquare
\end{array}
\right]
\end{equation*}
\begin{hint}
There might be a solution. If so, there are infinitely many.
\end{hint}
\end{problem}

\begin{problem}\label{prb:2.8}
Suppose a system of equations has fewer equations than variables. Will such a system necessarily be consistent? If so, explain why and if not, give an example
which is not consistent.
\begin{hint}
No. Consider $x+y+z=2$ and $x+y+z=1.$
\end{hint}
\end{problem}

\begin{problem}\label{prb:2.9}
If a system of equations has more equations than variables, can it
have a solution? If so, give an example and if not, tell why not.
\begin{hint}
These can
have a solution. For example, $x+y=1,2x+2y=2,3x+3y=3$ even has an infinite
set of solutions.
\end{hint}
\end{problem}

\begin{problem}\label{prb:2.10}
Find $h$ such that
\begin{equation*}
\left[
\begin{array}{rr|r}
2 & h & 4 \\
3 & 6 & 7
\end{array}
\right]
\end{equation*}
is the augmented matrix of an \textit{inconsistent} system.

$h=\answer{4}$

\end{problem}

\begin{problem}\label{prb:2.11}
Find $h$ such that
\begin{equation*}
\left[
\begin{array}{rr|r}
1 & h & 3 \\
2 & 4 & 6
\end{array}
\right]
\end{equation*}
is the augmented matrix of a \textit{consistent} system.
\begin{hint}
 Any $h$ will work.
\end{hint}
\end{problem}

\begin{problem}\label{prb:2.12}
Find $h$ such that
\begin{equation*}
\left[
\begin{array}{rr|r}
1 & 1 & 4 \\
3 & h & 12
\end{array}
\right]
\end{equation*}
is the augmented matrix of a \textit{consistent} system.
\begin{hint}
 Any $h$ will work.
\end{hint}
\end{problem}


\begin{problem}\label{prb:2.13}
Choose $h$ and $k$ such that the augmented matrix shown has each of the following:
\begin{enumerate}
\item one solution
\item no solution
\item infinitely many solutions
\end{enumerate}
\begin{equation*}
\left[
\begin{array}{rr|r}
1 & h & 2 \\
2 & 4 & k
\end{array}
\right]
\end{equation*}
\begin{hint}
If $h\neq 2$ there will be a unique solution for any $k$. If $h=2$ and $%
k\neq 4,$ there are no solutions. If $h=2$ and $k=4,$ then there are
infinitely many solutions.
\end{hint}
\end{problem}


\begin{problem}\label{prb:2.14}
Choose $h$ and $k$ such that the augmented matrix shown has each of the following:
\begin{enumerate}
\item one solution
\item no solution
\item infinitely many solutions
\end{enumerate}
\begin{equation*}
\left[
\begin{array}{rr|r}
1 & 2 & 2 \\
2 & h & k
\end{array}
\right]
\end{equation*}
\begin{hint}
If $h\neq 4,$ then there is exactly one solution. If $h=4$ and $k\neq 4,$
then there are no solutions. If $h=4$ and $k=4,$ then there are infinitely
many solutions.
\end{hint}
\end{problem}


\begin{problem}\label{prb:2.15}
Determine if the system is consistent. If so, is the solution unique?
\begin{equation*}
\begin{array}{c}
x+2y+z-w=2 \\
x-y+z+w=1 \\
2x+y-z=1 \\
4x+2y+z=5
\end{array}
\end{equation*}
\begin{hint}
There is no solution. The system is inconsistent. You can see this from the
augmented matrix. $\mbox{rref}\left(\left[
\begin{array}{rrrr|r}
1 & 2 & 1 & -1 & 2 \\
1 & -1 & 1 & 1 & 1 \\
2 & 1 & -1 & 0 & 1 \\
4 & 2 & 1 & 0 & 5
\end{array}
\right]\right) = \left[
\begin{array}{rrrr|r}
1 & 0 & 0 & \vspace{0.05in}\frac{1}{3} & 0 \\
0 & 1 & 0 & -\vspace{0.05in}\frac{2}{3} & 0 \\
0 & 0 & 1 & 0 & 0 \\
0 & 0 & 0 & 0 & 1
\end{array}
\right] .$
\end{hint}
\end{problem}

\begin{problem}\label{prb:2.16}
Determine if the system is consistent. If so, is the solution unique?
\begin{equation*}
\begin{array}{c}
x+2y+z-w=2 \\
x-y+z+w=0 \\
2x+y-z=1 \\
4x+2y+z=3
\end{array}
\end{equation*}
\begin{hint}
Solution is: $ w=\frac{3}{2}y-1, x=\frac{2}{3}-\frac{1}{2}y, z=\frac{1}{3} $
\end{hint}
\end{problem}

\begin{problem}\label{prb:2.17} Determine which matrices are in reduced row-echelon form.

\begin{enumerate}
\item $\left[
\begin{array}{rrr}
1 & 2 & 0 \\
0 & 1 & 7
\end{array}
\right] $ \wordChoice{\choice{yes} \choice[correct]{no}} 

\item $\left[
\begin{array}{rrrr}
1 & 0 & 0 & 0 \\
0 & 0 & 1 & 2 \\
0 & 0 & 0 & 0
\end{array}
\right] $ \wordChoice{\choice[correct]{yes} \choice{no}}

\item $\left[
\begin{array}{rrrrrr}
1 & 1 & 0 & 0 & 0 & 5 \\
0 & 0 & 1 & 2 & 0 & 4 \\
0 & 0 & 0 & 0 & 1 & 3
\end{array}
\right] $ \wordChoice{\choice[correct]{yes} \choice{no}}

\end{enumerate}
\end{problem}

\begin{problem}\label{prb:2.18} Row reduce the following matrix to row echelon form. Then continue to obtain the reduced row echelon form.
\begin{equation*}
\left[
\begin{array}{rrrr}
2 & -1 & 3 & -1 \\
1 & 0 & 2 & 1 \\
1 & -1 & 1 & -2
\end{array}
\right]
\end{equation*}
%\begin{hint}
%\end{hint}
\end{problem}

\begin{problem}\label{prb:2.19} Row reduce the following matrix to row echelon form. Then continue to obtain the reduced row echelon form.
\begin{equation*}
\left[
\begin{array}{rrrr}
0 & 0 & -1 & -1 \\
1 & 1 & 1 & 0 \\
1 & 1 & 0 & -1
\end{array}
\right]
\end{equation*}
%\begin{hint}
%\end{hint}
\end{problem}

\begin{problem}\label{prb:2.20} Row reduce the following matrix to row echelon form. Then continue to obtain the reduced row echelon form.
\begin{equation*}
\left[
\begin{array}{rrrr}
3 & -6 & -7 & -8 \\
1 & -2 & -2 & -2 \\
1 & -2 & -3 & -4
\end{array}
\right]
\end{equation*}
%\begin{hint}
%\end{hint}
\end{problem}

\begin{problem}\label{prb:2.21} Row reduce the following matrix to row echelon form. Then continue to obtain the reduced row echelon form.
\begin{equation*}
\left[
\begin{array}{rrrr}
2 & 4 & 5 & 15 \\
1 & 2 & 3 & 9 \\
1 & 2 & 2 & 6
\end{array}
\right]
\end{equation*}
%\begin{hint}
%\end{hint}
\end{problem}

\begin{problem}\label{prb:2.22} Row reduce the following matrix to row echelon form. Then continue to obtain the reduced row echelon form.
\begin{equation*}
\left[
\begin{array}{rrrr}
4 & -1 & 7 & 10 \\
1 & 0 & 3 & 3 \\
1 & -1 & -2 & 1
\end{array}
\right]
\end{equation*}
%\begin{hint}
%\end{hint}
\end{problem}

\begin{problem}\label{prb:2.23} Row reduce the following matrix to row echelon form. Then continue to obtain the reduced row echelon form.
\begin{equation*}
\left[
\begin{array}{rrrr}
3 & 5 & -4 & 2 \\
1 & 2 & -1 & 1 \\
1 & 1 & -2 & 0
\end{array}
\right]
\end{equation*}
%\begin{hint}
%\end{hint}
\end{problem}

\begin{problem}\label{prb:2.24} Row reduce the following matrix to row echelon form. Then continue to obtain the reduced row echelon form.
\begin{equation*}
\left[
\begin{array}{rrrr}
-2 & 3 & -8 & 7 \\
1 & -2 & 5 & -5 \\
1 & -3 & 7 & -8
\end{array}
\right]
\end{equation*}
%\begin{hint}
%\end{hint}
\end{problem}

\begin{problem}\label{prb:2.25} Find the solution of the system whose augmented matrix is
\begin{equation*}
\left[
\begin{array}{rrr|r}
1 & 2 & 0 & 2 \\
1 & 3 & 4 & 2 \\
1 & 0 & 2 & 1
\end{array}
\right]
\end{equation*}
%\begin{hint}
%\end{hint}
\end{problem}

\begin{problem}\label{prb:2.26} Find the solution of the system whose augmented matrix is
\begin{equation*}
\left[
\begin{array}{rrr|r}
1 & 2 & 0 & 2 \\
2 & 0 & 1 & 1 \\
3 & 2 & 1 & 3
\end{array}
\right]
\end{equation*}
\begin{hint}
The reduced row echelon form is $\left[
\begin{array}{rrr|r}
1 & 0 & \vspace{0.05in}\frac{1}{2} & \vspace{0.05in}\frac{1}{2} \\
0 & 1 & -\vspace{0.05in}\frac{1}{4} & \vspace{0.05in}\frac{3}{4} \\
0 & 0 & 0 & 0
\end{array}
\right] .$ Therefore, the solution is of the form $z=t,y=\frac{3}{4}+t\left(
\frac{1}{4}\right) ,x=\frac{1}{2}-\frac{1}{2}t$ where $t\in \mathbb{R}$.
\end{hint}
\end{problem}

\begin{problem}\label{prb:2.27} Find the solution of the system whose augmented matrix is
\begin{equation*}
\left[
\begin{array}{rrr|r}
1 & 1 & 0 & 1 \\
1 & 0 & 4 & 2
\end{array}
\right]
\end{equation*}
\begin{hint}
The reduced row echelon form is $\left[
\begin{array}{rrr|r}
1 & 0 & 4 & 2 \\
0 & 1 & -4 & -1
\end{array}
\right] $ and so the solution is $z=t,y=4t,x=2-4t.$
\end{hint}
\end{problem}

\begin{problem}\label{prb:2.28} Find the solution of the system whose augmented matrix is
\begin{equation*}
\left[
\begin{array}{rrrrr|r}
1 & 0 & 2 & 1 & 1 & 2 \\
0 & 1 & 0 & 1 & 2 & 1 \\
1 & 2 & 0 & 0 & 1 & 3 \\
1 & 0 & 1 & 0 & 2 & 2
\end{array}
\right]
\end{equation*}
\begin{hint}
The reduced row echelon form is $\left[
\begin{array}{rrrrr|r}
1 & 0 & 0 & 0 & 9 & 3 \\
0 & 1 & 0 & 0 & -4 & 0 \\
0 & 0 & 1 & 0 & -7 & -1 \\
0 & 0 & 0 & 1 & 6 & 1
\end{array}
\right] $ and so $x_{5}=t,x_{4}=1-6t,x_{3}=-1+7t,x_{2}=4t,x_{1}=3-9t$.
\end{hint}
\end{problem}

\begin{problem}\label{prb:2.29} Find the solution of the system whose augmented matrix is
\begin{equation*}
\left[
\begin{array}{rrrrr|r}
1 & 0 & 2 & 1 & 1 & 2 \\
0 & 1 & 0 & 1 & 2 & 1 \\
0 & 2 & 0 & 0 & 1 & 3 \\
1 & -1 & 2 & 2 & 2 & 0
\end{array}
\right]
\end{equation*}
\begin{hint}
The reduced row echelon form is $\left[
\begin{array}{rrrrr|r}
1 & 0 & 2 & 0 & -\vspace{0.05in}\frac{1}{2} & \vspace{0.05in}\frac{5}{2} \\
0 & 1 & 0 & 0 & \vspace{0.05in}\frac{1}{2} & \vspace{0.05in}\frac{3}{2} \\
0 & 0 & 0 & 1 & \vspace{0.05in}\frac{3}{2} & -\vspace{0.05in}\frac{1}{2} \\
0 & 0 & 0 & 0 & 0 & 0
\end{array}
\right] $. Therefore, let $x_{5}=t,x_{3}=s.$ Then the other variables are
given by $x_{4}=-\frac{1}{2}-\frac{3}{2}t,x_{2}=\frac{3}{2}-t\frac{1}{2}
,,x_{1}=\frac{5}{2}+\frac{1}{2}t-2s.$
\end{hint}
\end{problem}

\begin{problem}\label{prb:2.30} Find the solution to the system of equations, $7x+14y+15z=22,
$ $2x+4y+3z=5,$ and $3x+6y+10z=13.$
 \begin{align*}
 x&=\answer{1}-\answer{2}t\\
 y&=t\\
 z&=\answer{1}\\
 \end{align*}
\end{problem}

\begin{problem}\label{prb:2.31} Find the solution to the system of equations, $3x-y+4z=6,$
$y+8z=0,$ and $-2x+y=-4.$
 \begin{align*}
 x&=\answer{2}-\answer{4}t\\
 y&=\answer{-8}t\\
 z&=t\\
 \end{align*}
\end{problem}

\begin{problem}\label{prb:2.32} Find the solution to the system of equations, $9x-2y+4z=-17,
$ $13x-3y+6z=-25,$ and $-2x-z=3.$
 \begin{align*}
 x&=\answer{-1}\\
 y&=\answer{2}\\
 z&=\answer{-1}\\
 \end{align*}
\end{problem}

\begin{problem}\label{prb:2.33} Find the solution to the system of equations,
$65x+84y+16z=546,$ $81x+105y+20z=682,$ and $84x+110y+21z=713.$
 \begin{align*}
 x&=\answer{2}\\
 y&=\answer{4}\\
 z&=\answer{5}\\
 \end{align*}
\end{problem}

\begin{problem}\label{prb:2.34} Find the solution to the system of equations,
$8x+2y+3z=-3,8x+3y+3z=-1,$ and $4x+y+3z=-9.$
 \begin{align*}
 x&=\answer{1}\\
 y&=\answer{2}\\
 z&=\answer{-5}\\
 \end{align*}
\end{problem}

\begin{problem}\label{prb:2.35} Find the solution to the system of equations,
$-8x+2y+5z=18,-8x+3y+5z=13,$ and $-4x+y+5z=19.$
 \begin{align*}
 x&=\answer{-1}\\
 y&=\answer{-5}\\
 z&=\answer{4}\\
 \end{align*}
\end{problem}

\begin{problem}\label{prb:2.36} Find the solution to the system of equations, $3x-y-2z=3,$
$y-4z=0,$ and $-2x+y=-2.$
 \begin{align*}
 x&=\answer{1}-\answer{2}t\\
 y&=\answer{4}t\\
 z&=t\\
 \end{align*}
\end{problem}

\begin{problem}\label{prb:2.37} Find the solution to the system of equations,
$-9x+15y=66,-11x+18y=79$, $-x+y=4$, and $z=3$.
 \begin{align*}
 x&=\answer{1}\\
 y&=\answer{5}\\
 z&=\answer{3}\\
 \end{align*}
\end{problem}

\begin{problem}\label{prb:2.38} Find the solution to the system of equations, $-19x+8y=-108,$
$-71x+30y=-404,$ $-2x+y=-12,$ $4x+z=14.$
 \begin{align*}
 x&=\answer{4}\\
 y&=\answer{-4}\\
 z&=\answer{-2}\\
 \end{align*}
\end{problem}

\begin{problem}\label{prb:2.39} Suppose a system of equations has fewer equations than variables and
you have found a solution to this system of equations. Is it possible that
your solution is the only one?\ Explain.
\begin{hint}
No. Consider $x+y+z=2$ and $x+y+z=1.$
\end{hint}
\end{problem}

\begin{problem}\label{prb:2.40} Suppose a system of linear equations has a $2\times 4$ augmented
matrix and the last column is a pivot column. Could the system of linear
equations be consistent? Explain.
\begin{hint}
 No. This would lead to $0=1.$
\end{hint}
\end{problem}

\begin{problem}\label{prb:2.41} Suppose the coefficient matrix of a system of $n$ equations with $n$
variables has the property that every column is a pivot column. Does it
follow that the system of equations must have a solution? If so, must the
solution be unique? Explain.

The solution \wordChoice{\choice[correct]{exists} \choice{does not exist}} and is \wordChoice{\choice[correct]{unique} \choice{not unique}}.
\end{problem}

\begin{problem}\label{prb:2.42} Suppose there is a unique solution to a system of linear equations.
What must be true of the pivot columns in the augmented matrix?
\begin{hint}
The last column must not be a pivot column. The remaining columns must each be pivot
columns.
\end{hint}
\end{problem}


\begin{problem}\label{prb:2.43} The steady state temperature, $u$, of a plate solves Laplace's
equation, $\Delta u=0.$ One way to approximate the solution is to divide the plate into a square mesh and require the temperature
at each node to equal the average of the temperature at the four adjacent
nodes. In the following picture, the numbers represent the observed
temperature at the indicated nodes. Find the temperature at
the interior nodes, indicated by $x,y,z,$ and $w$. One of the equations is
$z=\frac{1}{4}\left( 10+0+w+x\right) $.

\begin{center}
   \begin{tikzpicture}[scale=1]
\node[red] at (-0.3, 1)   (a) {$20$};
\node[red] at (-0.3, 2)   (a) {$20$};
    \node[red] at (3.3, 1)   (b) {$0$};
     \node[red] at (3.3, 2)   (b) {$0$};
     \node[red] at (1, -0.3)   (c) {$10$};
      \node[red] at (2, -0.3)   (c) {$10$};
      \node[red] at (1, 3.3)   (c) {$30$};
      \node[red] at (2, 3.3)   (c) {$30$};
      \node[] at (1.2, 1.2)   (c) {$x$};
      \node[] at (2.2, 1.2)   (c) {$z$};
      \node[] at (1.2, 2.2)   (c) {$y$};
      \node[] at (2.2, 2.2)   (c) {$w$};
  \draw[-] (0,2)--(3,2);
  \draw[-] (0,1)--(3,1);
  \draw[-] (1,0)--(1,3);
  \draw[-] (2,0)--(2,3);
  \fill[] (1,1) circle (0.05cm); 
  \fill[] (1,2) circle (0.05cm); 
   \fill[] (2,1) circle (0.05cm); 
  \fill[] (2,2) circle (0.05cm); 
   \fill[] (0,1) circle (0.05cm); 
  \fill[] (0,2) circle (0.05cm); 
   \fill[] (3,1) circle (0.05cm); 
  \fill[] (3,2) circle (0.05cm); 
   \fill[] (1,0) circle (0.05cm); 
  \fill[] (1,3) circle (0.05cm); 
   \fill[] (2,0) circle (0.05cm); 
  \fill[] (2,3) circle (0.05cm); 
    \end{tikzpicture}
\end{center}

\begin{hint}
You need 
\begin{align*}
&\frac{1}{4}\left( 20+30+w+x\right) -y=0 \\
&\frac{1}{4}\left( y+30+0+z\right) -w=0 \\
&\frac{1}{4}\left( 20+y+z+10\right) -x=0 \\
&\frac{1}{4}\left( x+w+0+10\right) -z=0
\end{align*}
\end{hint}
 \begin{align*}
 w&=\answer{15}\\
 x&=\answer{15}\\
 y&=\answer{20}\\
 z&=\answer{10}\\
 \end{align*}
\end{problem}

\begin{problem}\label{prb:2.44} Find the rank of the following matrix.
\begin{equation*}
\left[
\begin{array}{rrrr}
4 & -16 & -1 & -5 \\
1 & -4 & 0 & -1 \\
1 & -4 & -1 & -2
\end{array}
\right]
\end{equation*}
%\begin{hint}
%\end{hint}
\end{problem}

\begin{problem}\label{prb:2.45} Find the rank of the following matrix.
\begin{equation*}
\left[
\begin{array}{rrrr}
3 & 6 & 5 & 12 \\
1 & 2 & 2 & 5 \\
1 & 2 & 1 & 2
\end{array}
\right]
\end{equation*}
%\begin{hint}
%\end{hint}
\end{problem}

\begin{problem}\label{prb:2.46} Find the rank of the following matrix.
\begin{equation*}
\left[
\begin{array}{rrrrr}
0 & 0 & -1 & 0 & 3 \\
1 & 4 & 1 & 0 & -8 \\
1 & 4 & 0 & 1 & 2 \\
-1 & -4 & 0 & -1 & -2
\end{array}
\right]
\end{equation*}
%\begin{hint}
%\end{hint}
\end{problem}

\begin{problem}\label{prb:2.47} Find the rank of the following matrix.
\begin{equation*}
\left[
\begin{array}{rrrr}
4 & -4 & 3 & -9 \\
1 & -1 & 1 & -2 \\
1 & -1 & 0 & -3
\end{array}
\right]
\end{equation*}
%\begin{hint}
%\end{hint}
\end{problem}

\begin{problem}\label{prb:2.48} Find the rank of the following matrix.
\begin{equation*}
\left[
\begin{array}{rrrrr}
2 & 0 & 1 & 0 & 1 \\
1 & 0 & 1 & 0 & 0 \\
1 & 0 & 0 & 1 & 7 \\
1 & 0 & 0 & 1 & 7
\end{array}
\right]
\end{equation*}
%\begin{hint}
%\end{hint}
\end{problem}

\begin{problem}\label{prb:2.49} Find the rank of the following matrix.
\begin{equation*}
\left[
\begin{array}{rrr}
4 & 15 & 29 \\
1 & 4 & 8 \\
1 & 3 & 5 \\
3 & 9 & 15
\end{array}
\right]
\end{equation*}
%\begin{hint}
%\end{hint}
\end{problem}

\begin{problem}\label{prb:2.50} Find the rank of the following matrix.
\begin{equation*}
\left[
\begin{array}{rrrrr}
0 & 0 & -1 & 0 & 1 \\
1 & 2 & 3 & -2 & -18 \\
1 & 2 & 2 & -1 & -11 \\
-1 & -2 & -2 & 1 & 11
\end{array}
\right]
\end{equation*}
%\begin{hint}
%\end{hint}
\end{problem}

\begin{problem}\label{prb:2.51} Find the rank of the following matrix.
\begin{equation*}
\left[
\begin{array}{rrrrr}
1 & -2 & 0 & 3 & 11 \\
1 & -2 & 0 & 4 & 15 \\
1 & -2 & 0 & 3 & 11 \\
0 & 0 & 0 & 0 & 0
\end{array}
\right]
\end{equation*}
%\begin{hint}
%\end{hint}
\end{problem}

\begin{problem}\label{prb:2.52} Find the rank of the following matrix.
\begin{equation*}
\left[
\begin{array}{rrr}
-2 & -3 & -2 \\
1 & 1 & 1 \\
1 & 0 & 1 \\
-3 & 0 & -3
\end{array}
\right]
\end{equation*}
%\begin{hint}
%\end{hint}
\end{problem}

\begin{problem}\label{prb:2.53} Find the rank of the following matrix.
\begin{equation*}
\left[
\begin{array}{rrrrr}
4 & 4 & 20 & -1 & 17 \\
1 & 1 & 5 & 0 & 5 \\
1 & 1 & 5 & -1 & 2 \\
3 & 3 & 15 & -3 & 6
\end{array}
\right]
\end{equation*}
%\begin{hint}
%\end{hint}
\end{problem}

\begin{problem}\label{prb:2.54} Find the rank of the following matrix.
\begin{equation*}
\left[
\begin{array}{rrrrr}
-1 & 3 & 4 & -3 & 8 \\
1 & -3 & -4 & 2 & -5 \\
1 & -3 & -4 & 1 & -2 \\
-2 & 6 & 8 & -2 & 4
\end{array}
 \right]
\end{equation*}
%\begin{hint}
%\end{hint}
\end{problem}

\begin{problem}\label{prb:2.55} Suppose $A$ is an $m\times n$ matrix. Explain why the rank of $A$ is
always no larger than $\min \left( m,n\right) .$
\begin{hint}
It is because you cannot
have more than $\min \left( m,n\right) $ nonzero rows in the reduced row echelon form. Recall that the number of pivot columns is the same as the
number of nonzero rows from the description of this reduced row echelon form.
\end{hint}
\end{problem}

\begin{problem}\label{prb:2.56} State whether each of the following sets of data are possible for the
matrix equation $A\vec{x}=\vec{b}$. If possible, describe the solution set.
That is, tell whether there exists a unique solution, no solution or
infinitely many solutions. Here, $\left[ A |\vec{b} \right]$ denotes the augmented matrix.

\begin{enumerate}
\item $A$ is a $5\times 6$ matrix, $\mbox{rank}\left( A\right) =4$ and
$\mbox{rank}\left[ A |\vec{b} \right] =4.$
\begin{hint}
This says $B$ is in the span of four of the columns. Thus the columns are not independent. Infinite solution set.
\end{hint}

\item $A$ is a $3\times 4$ matrix, $\mbox{rank}\left( A\right) =3$ and
$\mbox{rank}\left[ A |\vec{b} \right] =2.$
\begin{hint}
This surely can't happen. If you add in another column, the rank does not get smaller.
\end{hint}

\item $A$ is a $4\times 2$ matrix, $\mbox{rank}\left( A\right) =4$ and
$\mbox{rank}\left[ A |\vec{b} \right] =4.$
\begin{hint}
This says $B$ is in the span of the columns and the columns must be
independent. You can't have the rank equal 4 if you only have two columns.
\end{hint}

\item $A$ is a $5\times 5$ matrix, $\mbox{rank}\left( A\right) =4$ and
$\mbox{rank}\left[ A |\vec{b} \right] =5.$
\begin{hint}
This says $B$ is not in the span of the columns. In this case, there is no solution to the system of equations represented by the augmented matrix.
\end{hint}

\item $A$ is a $4\times 2$ matrix, $\mbox{rank}\left( A\right) =2$ and
$\mbox{rank}\left[ A |\vec{b} \right] =2$.

\begin{hint}
In this case, there is a
unique solution since the columns of $A$ are independent.
\end{hint}
\end{enumerate}
\end{problem}

\begin{problem}\label{prb:2.57} Consider the system $-5x+2y-z=0$ and $-5x-2y-z=0.$ Both equations
equal zero and so $-5x+2y-z=-5x-2y-z$ which is equivalent to $y=0.$ Does it follow that $x$
and $z$ can equal anything?  Notice that when $x=1$, $z=-4,$ and $y=0$ are plugged in
to the equations, the equations do not equal $0$. Why?
\begin{hint}
These are not legitimate row
operations. They do not preserve the solution set of the system.
\end{hint}
\end{problem}




\section*{Practice Problem Source}
These problems come from the end of Chapter 1 of Ken Kuttler's \href{https://open.umn.edu/opentextbooks/textbooks/a-first-course-in-linear-algebra-2017}{\it A First Course in Linear Algebra}. (CC-BY)

Ken Kuttler, {\it  A First Course in Linear Algebra}, Lyryx 2017, Open Edition, pp. 42--49. 

\end{document}