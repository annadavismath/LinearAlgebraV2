\documentclass{ximera}
%% You can put user macros here
%% However, you cannot make new environments

\listfiles

\graphicspath{
{./}
{./LTR-0070/}
{./VEC-0060/}
{./APP-0020/}
}

\usepackage{tikz}
\usepackage{tkz-euclide}
\usepackage{tikz-3dplot}
\usepackage{tikz-cd}
\usetikzlibrary{shapes.geometric}
\usetikzlibrary{arrows}
%\usetkzobj{all}
\pgfplotsset{compat=1.13} % prevents compile error.

%\renewcommand{\vec}[1]{\mathbf{#1}}
\renewcommand{\vec}{\mathbf}
\newcommand{\RR}{\mathbb{R}}
\newcommand{\dfn}{\textit}
\newcommand{\dotp}{\cdot}
\newcommand{\id}{\text{id}}
\newcommand\norm[1]{\left\lVert#1\right\rVert}
 
\newtheorem{general}{Generalization}
\newtheorem{initprob}{Exploration Problem}

\tikzstyle geometryDiagrams=[ultra thick,color=blue!50!black]

%\DefineVerbatimEnvironment{octave}{Verbatim}{numbers=left,frame=lines,label=Octave,labelposition=topline}



\usepackage{mathtools}


\author{Anna Davis \and Paul Zachlin \and Paul Bender} \title{Linear Independence} \license{CC-BY 4.0}

\begin{document}

\begin{abstract}
 \end{abstract}
\maketitle

\section*{Linear Independence}
\subsection*{Redundant Vectors}
Need an Intro sentence. 
\begin{exploration}\label{exp:redundantVecs1}
Consider the following collection of vectors:
$$\left\{\begin{bmatrix}2\\-1\end{bmatrix}, \begin{bmatrix}-4\\2\end{bmatrix}, \begin{bmatrix}2\\1\end{bmatrix}\right\}$$
What is the span of these vectors?  \wordChoice{\choice{A line}, \choice[correct]{$\RR^2$}, \choice{A parallelogram}, \choice{A parallelepiped}}

In this Exploration we will examine what can happen to the span of a collection of vectors when a vector is removed from the collection.  

First, let's remove
$\begin{bmatrix}2\\1\end{bmatrix}$ from $\left\{\begin{bmatrix}2\\-1\end{bmatrix}, \begin{bmatrix}-4\\2\end{bmatrix}, \begin{bmatrix}2\\1\end{bmatrix}\right\}$.  Which of the following is true?  
\begin{multipleChoice}  
\choice{$\mbox{span}\left(\begin{bmatrix}2\\-1\end{bmatrix}, \begin{bmatrix}-4\\2\end{bmatrix}\right)=\mbox{span}\left(\begin{bmatrix}2\\-1\end{bmatrix}, \begin{bmatrix}-4\\2\end{bmatrix}, \begin{bmatrix}2\\1\end{bmatrix}\right)$}  
\choice[correct]{$\mbox{span}\left(\begin{bmatrix}2\\-1\end{bmatrix}, \begin{bmatrix}-4\\2\end{bmatrix}\right)$ is a line}  
\choice{$\mbox{span}\left(\begin{bmatrix}2\\-1\end{bmatrix}, \begin{bmatrix}-4\\2\end{bmatrix}\right)=\RR^2$}
\choice{$\mbox{span}\left(\begin{bmatrix}2\\-1\end{bmatrix}, \begin{bmatrix}-4\\2\end{bmatrix}\right)$ is a parallelogram.}
\end{multipleChoice}  

We see that the span \wordChoice{\choice[correct]{has changed}, \choice{has not changed.}}

Now let's remove $\begin{bmatrix}-4\\2\end{bmatrix}$ from the \emph{original} collection of vectors.  Which of the following is true?

\begin{multipleChoice}  
\choice[correct]{$\mbox{span}\left(\begin{bmatrix}2\\-1\end{bmatrix}, \begin{bmatrix}2\\1\end{bmatrix}\right)=\mbox{span}\left(\begin{bmatrix}2\\-1\end{bmatrix}, \begin{bmatrix}-4\\2\end{bmatrix}, \begin{bmatrix}2\\1\end{bmatrix}\right)$}  
\choice{$\mbox{span}\left(\begin{bmatrix}2\\-1\end{bmatrix}, \begin{bmatrix}2\\1\end{bmatrix}\right)$ is a line}  
\choice{$\mbox{span}\left(\begin{bmatrix}2\\-1\end{bmatrix}, \begin{bmatrix}2\\1\end{bmatrix}\right)$ is the right side of the coordinate plane.}
\choice{$\mbox{span}\left(\begin{bmatrix}2\\-1\end{bmatrix}, \begin{bmatrix}2\\1\end{bmatrix}\right)$ is a parallelogram.}
\end{multipleChoice}  

We see that the span \wordChoice{\choice{has changed.}, \choice[correct]{has not changed.}}

\end{exploration}

As you just discovered, removing a vector from a collection of vectors may or may not affect the span of the collection.  We will refer to vectors that can be removed from a collection without changing the span as \dfn{redundant}.  In Exploration \ref{exp:redundantVecs1}, $\begin{bmatrix}-4\\2\end{bmatrix}$ is \dfn{redundant}, while $\begin{bmatrix}2\\1\end{bmatrix}$ is not.

\begin{definition}\label{def:redundant}
Let $\{\vec{v}_1,\vec{v}_2,\dots,\vec{v}_k\}$ be a collection of vectors in $\RR^n$.  If removing $\vec{v}_j$ ($1\leq j\leq k$) from this collection does not change the span of the collection, in other words, if $\mbox{span}\left(\vec{v}_1,\vec{v}_2,\dots,\vec{v}_k\right)=\mbox{span}\left(\vec{v}_1,\vec{v}_2,\dots,\vec{v}_{j-1},\vec{v}_{j+1},\dots,\vec{v}_k\right)$, then we say that $\vec{v}_j$ is a redundant element of $\{\vec{v}_1,\vec{v}_2,\dots,\vec{v}_k\}$, or simply redundant.
\end{definition}

Our next goal is to see what causes $\begin{bmatrix}-4\\2\end{bmatrix}$ to be redundant.  The answer lies not in the vector itself, but in its relationship to the other vectors in the collection. Observe that $\begin{bmatrix}-4\\2\end{bmatrix}=-2\begin{bmatrix}2\\-1\end{bmatrix}$.  In other words, $\begin{bmatrix}-4\\2\end{bmatrix}$ is a scalar multiple of another vector in the set.  To see why this matters, let's pick an arbitrary vector $\vec{w}=\begin{bmatrix}0\\2\end{bmatrix}$ in $\mbox{span}\left(\begin{bmatrix}2\\-1\end{bmatrix}, \begin{bmatrix}-4\\2\end{bmatrix}, \begin{bmatrix}2\\1\end{bmatrix}\right)$.  Vector $\vec{w}$ is in the span because it can be written as a linear combination of the three vectors as follows
$$\vec{w}=\begin{bmatrix}0\\2\end{bmatrix}=\begin{bmatrix}2\\-1\end{bmatrix}+ \begin{bmatrix}-4\\2\end{bmatrix}+ \begin{bmatrix}2\\1\end{bmatrix}$$

But $\begin{bmatrix}-4\\2\end{bmatrix}$ is not essential to this linear combination because it can be replaced with $-2\begin{bmatrix}2\\-1\end{bmatrix}$, as shown below.

$$\begin{bmatrix}0\\2\end{bmatrix}=\begin{bmatrix}2\\-1\end{bmatrix}+ \begin{bmatrix}-4\\2\end{bmatrix}+ \begin{bmatrix}2\\1\end{bmatrix}=\begin{bmatrix}2\\-1\end{bmatrix}+ (-2)\begin{bmatrix}2\\-1\end{bmatrix}+ \begin{bmatrix}2\\1\end{bmatrix}=-\begin{bmatrix}2\\-1\end{bmatrix}+ \begin{bmatrix}2\\1\end{bmatrix}$$

Regardless of what vector $\vec{w}$ we write as a linear combination of$\begin{bmatrix}2\\-1\end{bmatrix}$,$ \begin{bmatrix}-4\\2\end{bmatrix}$ and $\begin{bmatrix}2\\1\end{bmatrix}$, we will always be able to replace $\begin{bmatrix}-4\\2\end{bmatrix}$ with $-2\begin{bmatrix}2\\-1\end{bmatrix}$, placing $\vec{w}$ into the span of $\begin{bmatrix}2\\-1\end{bmatrix}$ and $\begin{bmatrix}2\\1\end{bmatrix}$, and making $\begin{bmatrix}-4\\2\end{bmatrix}$ redundant.  (Note that we can just as easily write $\begin{bmatrix}2\\-1\end{bmatrix}=-\frac{1}{2}\begin{bmatrix}-4\\2\end{bmatrix}$, and argue that $\begin{bmatrix}2\\-1\end{bmatrix}$ is redundant.)  We conclude that only one of $\begin{bmatrix}-4\\2\end{bmatrix}$ and $\begin{bmatrix}2\\-1\end{bmatrix}$ is needed to maintain the span of the original three vectors.  We have
$$\mbox{span}\left(\begin{bmatrix}2\\-1\end{bmatrix}, \begin{bmatrix}-4\\2\end{bmatrix}, \begin{bmatrix}2\\1\end{bmatrix}\right)=\mbox{span}\left(\begin{bmatrix}2\\-1\end{bmatrix}, \begin{bmatrix}2\\1\end{bmatrix}\right)=\mbox{span}\left(\begin{bmatrix}-4\\2\end{bmatrix}, \begin{bmatrix}2\\1\end{bmatrix}\right)$$
The left-most collection in this expression contains redundant vectors; the other two collections do not.

In Exploration \ref{exp:redundantVecs1} we found one vector to be redundant because we could replace it with a scalar multiple of another vector in the set.  The following Exploration delves into what happens when a vector in a given set is a linear combination of the other vectors.

\begin{exploration}\label{exp:redundantVecs2}
    Consider the set of vectors $$\left\{\begin{bmatrix}1\\2\\-1\end{bmatrix},\begin{bmatrix}2\\0\\1\end{bmatrix},\begin{bmatrix}4\\4\\-1\end{bmatrix}\right\}$$
The three vectors are shown below. RIGHT-CLICK and DRAG to rotate the interactive graph.
\begin{center}
\geogebra{x72vbsaw}{400}{400}
\end{center}
$\mbox{span}\left(\begin{bmatrix}1\\2\\-1\end{bmatrix},\begin{bmatrix}2\\0\\1\end{bmatrix},\begin{bmatrix}4\\4\\-1\end{bmatrix}\right)$ is \wordChoice{\choice{A line}, \choice[correct]{A plane}, \choice{$\RR^3$}, \choice{A parallelepiped}}

Can we remove one of the vectors from the set without changing the span?  Observe that we can write $\begin{bmatrix}4\\4\\-1\end{bmatrix}$ as a linear combination of the other two vectors
 \begin{equation}\label{eq:redundant}\begin{bmatrix}4\\4\\-1\end{bmatrix}=\answer{2}\begin{bmatrix}1\\2\\-1\end{bmatrix}+\answer{1}\begin{bmatrix}2\\0\\1\end{bmatrix}\end{equation}

 This means that we can write any vector in $\mbox{span}\left(\begin{bmatrix}1\\2\\-1\end{bmatrix},\begin{bmatrix}2\\0\\1\end{bmatrix},\begin{bmatrix}4\\4\\-1\end{bmatrix}\right)$ as a linear combination of only $\begin{bmatrix}1\\2\\-1\end{bmatrix}$ and $\begin{bmatrix}2\\0\\1\end{bmatrix}$ by replacing $\begin{bmatrix}4\\4\\-1\end{bmatrix}$ with the expression in (\ref{eq:redundant}). For example, $$\begin{bmatrix}7\\6\\-1\end{bmatrix}=\begin{bmatrix}1\\2\\-1\end{bmatrix}+\begin{bmatrix}2\\0\\1\end{bmatrix}+\begin{bmatrix}4\\4\\-1\end{bmatrix}=\answer{3}\begin{bmatrix}1\\2\\-1\end{bmatrix}+\answer{2}\begin{bmatrix}2\\0\\1\end{bmatrix}$$

We have
$$\mbox{span}\left(\begin{bmatrix}1\\2\\-1\end{bmatrix},\begin{bmatrix}2\\0\\1\end{bmatrix},\begin{bmatrix}4\\4\\-1\end{bmatrix}\right)=\mbox{span}\left(\begin{bmatrix}1\\2\\-1\end{bmatrix},\begin{bmatrix}2\\0\\1\end{bmatrix}\right)$$

 We conclude that vector $\begin{bmatrix}4\\4\\-1\end{bmatrix}$ is redundant.  Can each of the other two vectors in the set 
 $\left\{\begin{bmatrix}1\\2\\-1\end{bmatrix},\begin{bmatrix}2\\0\\1\end{bmatrix},\begin{bmatrix}4\\4\\-1\end{bmatrix}\right\}$ be considered redundant?  You will address this question in Practice Problem \ref{prob:redundant1}.
\end{exploration}
 Explorations \ref{exp:redundantVecs1} and \ref{exp:redundantVecs2} gave us a good intuitive understanding that the redundancy status of a vector as part of a collection of vectors depends on whether the given vector is a linear combination of the other vectors in the collection.  The following theorem formalizes our findings.

 \begin{theorem}\label{th:redundant}
     Let $\{\vec{v}_1,\vec{v}_2,\dots,\vec{v}_k\}$ be a collection of vectors in $\RR^n$.  Vector $\vec{v}_j$, ($1\leq j\leq k$) is redundant in this collection if and only if $\vec{v}_j$ can be expressed as a linear combination of the other vectors in the collection.
 \end{theorem}
 \begin{proof}
     First, suppose $\vec{v}_j$ is a linear combination of $\vec{v}_1,\vec{v}_2,\dots,\vec{v}_k$.  We will show that $\vec{v}_j$ is redundant by showing that $$\mbox{span}\left(\vec{v}_1,\vec{v}_2,\dots,\vec{v}_k\right)=\mbox{span}\left(\vec{v}_1,\vec{v}_2,\dots,\vec{v}_{j-1},\vec{v}_{j+1},\dots,\vec{v}_k\right)$$

To show equality of the two spans we will pick a vector in the left span and show that it is also an element of the span on the right.  Then, we will pick a vector in the right span and show that it is also an element of the span on the left.

Observe that if $\vec{w}$ is in $\mbox{span}\left(\vec{v}_1,\vec{v}_2,\dots,\vec{v}_{j-1},\vec{v}_{j+1},\dots,\vec{v}_k\right)$, then it has to be in $\mbox{span}\left(\vec{v}_1,\vec{v}_2,\dots,\vec{v}_k\right)$. (Why?)

Now suppose $\vec{w}$ is in $\mbox{span}\left(\vec{v}_1,\vec{v}_2,\dots,\vec{v}_k\right)$, we need to show that $\vec{w}$ is also in $\mbox{span}\left(\vec{v}_1,\vec{v}_2,\dots,\vec{v}_{j-1},\vec{v}_{j+1},\dots,\vec{v}_k\right)$.

     By assumption, we can write $\vec{v}_j$ as
\begin{equation}\label{eq:vj}
\vec{v}_j=a_1\vec{v}_1+a_2\vec{v}_2+\dots +a_{j-1}\vec{v}_{j-1}+a_{j+1}\vec{v}_{j+1}+\dots +a_k\vec{v}_k
\end{equation}

Since $\vec{w}$ is in $\mbox{span}\left(\vec{v}_1,\vec{v}_2,\dots,\vec{v}_k\right)$, we have
$$\vec{w}=b_1\vec{v}_1+b_2\vec{v}_2+\dots +b_j\vec{v}_j+\dots +b_k\vec{v}_k$$
Substituting the expression in (\ref{eq:vj}) for $\vec{v}_j$ and simplifying, we obtain the following
\begin{eqnarray*}\vec{w}=(b_1+b_ja_1)\vec{v}_1+(b_2+b_ja_2)\vec{v}_2&+&\dots\\
&+&(b_{j-1}+b_ja_{j-1})\vec{v}_{j-1}\\
&+&(b_{j+1}+b_ja_{j+1})\vec{v}_{j+1}\\
&+&\dots\\
&+&(b_k+b_ja_k)\vec{v}_k\end{eqnarray*}
This shows that $\vec{w}$ is in $\mbox{span}\left(\vec{v}_1,\vec{v}_2,\dots,\vec{v}_{j-1},\vec{v}_{j+1},\dots,\vec{v}_k\right)$.
We now have 
$$\mbox{span}\left(\vec{v}_1,\vec{v}_2,\dots,\vec{v}_k\right)=\mbox{span}\left(\vec{v}_1,\vec{v}_2,\dots,\vec{v}_{j-1},\vec{v}_{j+1},\dots,\vec{v}_k\right)$$
and the first part of the proof is complete.

Next, we will assume that $\vec{v}_j$ is redundant, then show that $\vec{v}_j$ is a linear combination of the other vectors.

If $\vec{v}_j$ is redundant then removing it from the set does not change the span. 
$$\mbox{span}\left(\vec{v}_1,\vec{v}_2,\dots,\vec{v}_k\right)=\mbox{span}\left(\vec{v}_1,\vec{v}_2,\dots,\vec{v}_{j-1},\vec{v}_{j+1},\dots,\vec{v}_k\right)$$
Consider a vector $\vec{w}$ in $\mbox{span}\left(\vec{v}_1,\vec{v}_2,\dots,\vec{v}_k\right)$  
\begin{equation}\label{eq:w1}
\vec{w}=a_1\vec{v}_1+a_2\vec{v}_2+\dots +a_k\vec{v}_k
\end{equation}
Since the span contains ALL possible linear combinations of $\vec{v}_1,\vec{v}_2,\dots,\vec{v}_k$, it is easy to ensure that $\vec{w}$ is such that $a_j\neq 0$.

By assumption, $\vec{w}$ is also in $\mbox{span}\left(\vec{v}_1,\vec{v}_2,\dots,\vec{v}_{j-1},\vec{v}_{j+1},\dots,\vec{v}_k\right)$.  Therefore, we can express $\vec{w}$ as a linear combination 
\begin{equation}\label{eq:w2}
\vec{w}=b_1\vec{v}_1+b_2\vec{v}_2+\dots +b_{j-1}\vec{v}_{j-1}+b_{j+1}\vec{v}_{j+1}+\dots +b_k\vec{v}_k
\end{equation}
Subtracting expression (\ref{eq:w2}) from (\ref{eq:w1}) we obtain
\begin{eqnarray}\label{eq:zero}
    \vec{0}=\vec{w}-\vec{w}=(a_1-b_1)\vec{v}_1&+&\dots\\ \nonumber
    &+&(a_{j-1}-b_{j-1})\vec{v}_{j-1}+a_j\vec{v}_j+(a_{j+1}-b_{j+1})\vec{v}_{j+1}\\ \nonumber
    &+&\dots\\ \nonumber
    &+&(a_k-b_k)\vec{v}_k
\end{eqnarray}
Recall that we ensured that $a_j\neq 0$.  This allows us to express $\vec{v}_j$ as a linear combination of the other vectors by solving for $\vec{v}_j$.
\begin{eqnarray*}
    \vec{v}_j=-\frac{1}{a_j}\Big[(a_1-b_1)\vec{v}_1&+&\dots\\
    &+&(a_{j-1}-b_{j-1})\vec{v}_{j-1}+(a_{j+1}-b_{j+1})\vec{v}_{j+1}\\
    &+&\dots\\
    &+&(a_k-b_k)\vec{v}_k\Big]
\end{eqnarray*}
 \end{proof}

 Collections of vectors that do not contain redundant vectors are very important in linear algebra.  We will start referring to such collections as \dfn{linearly independent}.  Collections of vectors that contain redundant vectors will be called \dfn{linearly dependent}.   

\section*{Linear Independence}

Our definition of linear independence and dependence from the previous section is a valuable intuitive tool.  The downside of this definition is that it does not provide a clear computational path for determining whether a collection of vectors is linearly dependent or independent.  The following definition will allow us to easily determine linear dependence and independence of vectors.

%\begin{expandable}
%Observe that this equation must be consistent, regardless of what the vectors are, because $a_1=a_2=\ldots =a_p=0$ is a solution.  This solution is known as the \dfn{trivial solution}.
%\end{expandable} 

\begin{definition}[Linear Independence]\label{def:linearindependence}
Let $\vec{v}_1, \vec{v}_2,\ldots ,\vec{v}_k$ be vectors of $\RR^n$.  We say that the set $\{\vec{v}_1, \vec{v}_2,\ldots ,\vec{v}_k\}$ is \dfn{linearly independent} if the only solution to 
\begin{equation}\label{eq:defLinInd}a_1\vec{v}_1+a_2\vec{v}_2+\ldots +a_p\vec{v}_k=\vec{0}\end{equation}
is the \dfn{trivial solution} $a_1=a_2=\ldots =a_k=0$.

If, in addition to the trivial solution, a \dfn{non-trivial solution} (not all $a_1, a_2,\ldots ,a_k$ are zero) exists, then we say that the set $\{\vec{v}_1, \vec{v}_2,\ldots ,\vec{v}_k\}$ is \dfn{linearly dependent}.
\end{definition}

If the {\it set} of vectors $\{\vec{v}_1, \vec{v}_2,\ldots ,\vec{v}_k\}$ is linearly independent (dependent), we often say that the {\it vectors} $\vec{v}_1, \vec{v}_2,\ldots ,\vec{v}_k$ are linearly independent (dependent).

To see that Definition \ref{def:linearindependence} is equivalent to our earlier definition involving redundant vectors observe that equation (\ref{eq:zero}) provides us with a non-trivial solution to equation (\ref{eq:defLinInd}) when redundant vectors are present.  Conversely, when a non-trivial solution to (\ref{eq:defLinInd}) exists, we can solve for the vector with a non-zero coefficient, expressing it as a linear combination of the other vectors.

\begin{example}\label{ex:linind}Determine whether the vectors in each part are linearly independent.

\begin{enumerate}
\item \label{item:linindpart1}
$$\begin{bmatrix}2\\-3\end{bmatrix}, \begin{bmatrix}0\\3\end{bmatrix},\begin{bmatrix}1\\-1\end{bmatrix},\begin{bmatrix}1\\-2\end{bmatrix}$$

\item \label{item:linindpart2} $$\begin{bmatrix}2\\1\\4\end{bmatrix},\begin{bmatrix}-3\\1\\1\end{bmatrix}$$
\end{enumerate}
\begin{explanation} \ref{item:linindpart1}
We will solve the vector equation
\begin{align}\label{eq:linrelationpart1}a_1\begin{bmatrix}2\\-3\end{bmatrix}+a_2 \begin{bmatrix}0\\3\end{bmatrix}+a_3\begin{bmatrix}1\\-1\end{bmatrix}+a_4\begin{bmatrix}1\\-2\end{bmatrix}=\vec{0}\end{align}
 
 Clearly $a_1=a_2=a_3=a_4=0$ is a solution to the equation.  The question is whether another solution exists.
 
The vector equation translates into the following system:

$$\begin{array}{ccccccccc}
      2a_1 & &&+&a_3&+&a_4&= &0 \\
        -3a_1& +&3a_2&-&a_3&-&2a_4&= &0 \\
      \end{array}$$
  Writing the system in augmented matrix form and applying elementary row operations gives us the following reduced row-echelon form:
  $$\left[\begin{array}{cccc|c}  
 2&0&1&1&0\\-3&3&-1&-2&0
 \end{array}\right]\rightsquigarrow\left[\begin{array}{cccc|c}  
 1&0&1/2&1/2&0\\0&1&1/6&-1/6&0
 \end{array}\right]$$
 This shows that (\ref{eq:linrelationpart1}) has infinitely many solutions:  
 $$a_1=-\frac{1}{2}s-\frac{1}{2}t,\quad a_2=-\frac{1}{6}s+\frac{1}{6}t,\quad a_3=s,\quad a_4=t$$
 Letting $t=s=6$, we obtain the following:
 
 $$-6\begin{bmatrix}2\\-3\end{bmatrix}+0 \begin{bmatrix}0\\3\end{bmatrix}+6\begin{bmatrix}1\\-1\end{bmatrix}+6\begin{bmatrix}1\\-2\end{bmatrix}=\vec{0}$$
 We conclude that the vectors are linearly dependent.
 
 \ref{item:linindpart2} We need to solve the equation
 
 $$a_1\begin{bmatrix}2\\1\\4\end{bmatrix}+a_2\begin{bmatrix}-3\\1\\1\end{bmatrix}=\vec{0}$$
 Converting the equation to augmented matrix form and performing row reduction gives us
 $$\left[\begin{array}{cc|c}  
 2&-3&0\\1&1&0\\4&1&0
 \end{array}\right]\rightsquigarrow\left[\begin{array}{cc|c}  
 1&0&0\\0&1&0\\0&0&0
 \end{array}\right]$$
 This shows that $a_1=a_2=0$ is the only solution.  Therefore the two vectors are linearly independent.
\end{explanation}
\end{example}

In Part \ref{item:linindpart1} of Example \ref{ex:linind} we found that the vectors $\begin{bmatrix}2\\-3\end{bmatrix}, \begin{bmatrix}0\\3\end{bmatrix},\begin{bmatrix}1\\-1\end{bmatrix},\begin{bmatrix}1\\-2\end{bmatrix}$ are linearly dependent because

\begin{equation}\label{eq:nontrivrel}
-6\begin{bmatrix}2\\-3\end{bmatrix}+0 \begin{bmatrix}0\\3\end{bmatrix}+6\begin{bmatrix}1\\-1\end{bmatrix}+6\begin{bmatrix}1\\-2\end{bmatrix}=\vec{0}\end{equation}

Observe that (\ref{eq:nontrivrel}) allows us to solve for one of the vectors and express it as a linear combination of the others. For example, 
$$\begin{bmatrix}2\\-3\end{bmatrix}=0 \begin{bmatrix}0\\3\end{bmatrix}+\begin{bmatrix}1\\-1\end{bmatrix}+\begin{bmatrix}1\\-2\end{bmatrix}$$
This would not be possible if a nontrivial solution to the equation
$$a_1\begin{bmatrix}2\\-3\end{bmatrix}+a_2 \begin{bmatrix}0\\3\end{bmatrix}+a_3\begin{bmatrix}1\\-1\end{bmatrix}+a_4\begin{bmatrix}1\\-2\end{bmatrix}=\vec{0}$$
did not exist.

\begin{theorem}\label{th:lindeplincombofother}
A subset of $\RR^n$ containing two or more vectors is linearly dependent if and only if one of the vectors can be expressed as a linear combination of the others.
\end{theorem}
\begin{proof}
See Practice Problem \ref{prob:lindeplincombofother}.
\end{proof}

\subsection*{Geometry of Linearly Dependent and Linearly Independent Vectors}

Theorem \ref{th:lindeplincombofother} gives us a convenient way of looking at linear dependence/independence geometrically.  When looking at two or more vectors, we ask ``can one of the vectors be written as a linear combination of the others?"  If the answer is ``yes", then the vectors are linearly dependent.
\subsubsection*{A Set of Two Vectors}
Two vectors are linearly dependent if and only if one is a scalar multiple of the other.  Two nonzero linearly dependent vectors may look like this:
\begin{center}
\begin{tikzpicture}[scale=0.8]
% \draw[<->] (-0.5,0)--(2.5,0);
 % \draw[<->] (0,-0.5)--(0,2.5);
    \draw[line width=1pt,blue,-stealth](0,0)--(2,2);
  \draw[line width=1pt,red,-stealth](0,0)--(1,1);
 \end{tikzpicture}
\end{center}
or like this:
\begin{center}
\begin{tikzpicture}[scale=0.8]
%\draw[<->] (-2.5,0)--(2.5,0);
 % \draw[<->] (0,-2.5)--(0,2.5);
    \draw[line width=1pt,blue,-stealth](0,0)--(1,2);
  \draw[line width=1pt,red,-stealth](0,0)--(-0.5,-1);
 \end{tikzpicture}
\end{center}
Two linearly independent vectors will look like this:
\begin{center}
\begin{tikzpicture}[scale=0.8]
%\draw[<->] (-2.5,0)--(2.5,0);
 % \draw[<->] (0,-2.5)--(0,2.5);
    \draw[line width=1pt,blue,-stealth](0,0)--(1,1);
  \draw[line width=1pt,red,-stealth](0,0)--(0.5,-1);
 \end{tikzpicture}
\end{center}
\subsubsection*{A Set of Three Vectors}
Given a set of three nonzero vectors, we have the following possibilities: 
\begin{itemize}
\item (Linearly Dependent Vectors)
The three vectors are scalar multiples of each other.
\begin{center}
\begin{tikzpicture}[scale=0.8]
% \draw[<->] (-0.5,0)--(2.5,0);
 % \draw[<->] (0,-0.5)--(0,2.5);
    \draw[line width=1pt,blue,-stealth](0,0)--(2,2);
  \draw[line width=1pt,red,-stealth](0,0)--(1,1);
  \draw[line width=1pt,-stealth](0,0)--(0.5,0.5);
 \end{tikzpicture}
\end{center}
\item (Linearly Dependent Vectors) Two of the vectors are scalar multiples of each other.
\begin{center}
\begin{tikzpicture}[scale=0.8]
    \draw[line width=1pt,blue,-stealth](0,0)--(2,2);
  \draw[line width=1pt,red,-stealth](0,0)--(1,1);
  \draw[line width=1pt,-stealth](0,0)--(1,0);
 \end{tikzpicture}
\end{center}
\item (Linearly Dependent Vectors) One vector can be viewed as the diagonal of a parallelogram determined by scalar multiples of the other two vectors.  All three vectors lie in the same plane.
\begin{center}
\begin{tikzpicture}[scale=0.8]
  \filldraw[blue, opacity=0.3](0,0)--(-2,2)--(2,4)--(4,2)--cycle;
\draw[line width=1pt,red,-stealth](0,0)--(2,1);
\draw[line width=1pt,red,-stealth, dashed](0,0)--(4,2);
  \draw[line width=1pt,blue,-stealth](0,0)--(-2,2);
  \draw[line width=1pt,-stealth](0,0)--(2,4); 
\end{tikzpicture}
\end{center}
\item (Linearly Independent Vectors)
A set of three vectors is linearly independent if the vectors do not lie in the same plane.  For example, vectors $\vec{i}$, $\vec{j}$ and $\vec{k}$ are linearly independent.
\end{itemize}

\section*{Practice Problems}
\begin{problem}\label{prob:redundant1}
Exploration \ref{exp:redundantVecs2}
\end{problem}
\begin{problem} Are the given vectors linearly independent?

\begin{problem}\label{prob:linindmultchoice1}
$$\begin{bmatrix}-1\\0\end{bmatrix}, \begin{bmatrix}2\\3\end{bmatrix},\begin{bmatrix}4\\-1\end{bmatrix}$$

\begin{multipleChoice}
 \choice{Yes}
  \choice[correct]{No }
 \end{multipleChoice}
\end{problem}

\begin{problem}\label{prob:linindmultchoice2}
$$\begin{bmatrix}1\\0\\5\end{bmatrix}, \begin{bmatrix}2\\2\\3\end{bmatrix},\begin{bmatrix}-1\\0\\1\end{bmatrix}$$

\begin{multipleChoice}
 \choice[correct]{Yes}
  \choice{No }
 \end{multipleChoice}

\end{problem}

\begin{problem}\label{prob:linindmultchoice3}
$$\begin{bmatrix}3\\0\\5\end{bmatrix}, \begin{bmatrix}2\\0\\2\end{bmatrix},\begin{bmatrix}-1\\0\\-5\end{bmatrix}$$

\begin{multipleChoice}
 \choice{Yes}
  \choice[correct]{No }
 \end{multipleChoice}
\end{problem}

\begin{problem}\label{prob:linindmultchoice4}
$$\begin{bmatrix}3\\1\\4\\1\end{bmatrix}, \begin{bmatrix}-2\\1\\1\\1\end{bmatrix}$$

\begin{multipleChoice}
 \choice[correct]{Yes}
  \choice{No }
 \end{multipleChoice}
\end{problem}

\end{problem}

\begin{problem} True or False?
\begin{problem}\label{prob:TFlinind1}
Any set containing the zero vector is linearly dependent.
\begin{multipleChoice}
 \choice[correct]{TRUE}
  \choice{FALSE}
 \end{multipleChoice}
 \end{problem}
\begin{problem}\label{prob:TFlinind2}
A set containing exactly one nonzero vector is linearly dependent.
\begin{multipleChoice}
 \choice{TRUE}
  \choice[correct]{FALSE}
 \end{multipleChoice}
\end{problem}

\end{problem}

\begin{problem}
Each problem below provides information about vectors $\vec{v}_1, \vec{v}_2, \vec{v}_3$.  If possible, determine whether the vectors are linearly dependent or independent.

\begin{problem}\label{prob:linindmultchoice5}
$$0\vec{v}_1+ 0\vec{v}_2+ 0\vec{v}_3=\vec{0}$$
\begin{multipleChoice}
 \choice{The vectors are linearly independent}
  \choice{The vectors are linearly dependent }
  \choice[correct]{There is not enough information given to make a determination }
 \end{multipleChoice}
\end{problem}

\begin{problem}\label{prob:linindmultchoice6}
$$3\vec{v}_1+ 4\vec{v}_2- \vec{v}_3=\vec{0}$$
\begin{multipleChoice}
 \choice{The vectors are linearly independent}
  \choice[correct]{The vectors are linearly dependent }
  \choice{There is not enough information given to make a determination }
 \end{multipleChoice}
\end{problem}

\begin{problem}\label{prob:linindmultchoice7}
$$2\vec{v}_1+ 0\vec{v}_2+ 0\vec{v}_3=\vec{0}$$
\begin{multipleChoice}
 \choice{The vectors are linearly independent}
  \choice[correct]{The vectors are linearly dependent }
  \choice{There is not enough information given to make a determination }
 \end{multipleChoice}
\end{problem}

\end{problem}

\begin{problem}\label{prob:lindeplincombofother}
Prove Theorem \ref{th:lindeplincombofother}.
\begin{hint}
This is an ``if and only if" statement.  First, assume that the set is linearly dependent and show that one vector can be expressed as a linear combination of the others.  Second, assume that one vector can be expressed as a linear combination of the others, then shown that the set is linearly dependent.
\end{hint}
\end{problem}

\begin{problem}
Each diagram below shows a collection of vectors.  Are the vectors linearly dependent or independent?

\begin{problem}\label{prob:linindmultchoice8}
\begin{multipleChoice}
 \choice{The vectors are linearly independent}
  \choice[correct]{The vectors are linearly dependent }
  \choice{There is not enough information given to make a determination }
 \end{multipleChoice}
\begin{center}
\tdplotsetmaincoords{70}{130}
\begin{tikzpicture}
	\draw[->](-2,0,0)--(3,0,0) node[below left]{$y$};
    \draw[->](0,-2,0)--(0,3,0) node[below left]{$z$};
    \draw[->](0,0,-2)--(0,0,3) node[below left]{$x$};
    \draw[->, line width=2pt,blue, -stealth](0,0,0)--(0,2,1);
    \draw[->, line width=2pt,red, -stealth](0,0,0)--(1,2,0);
    \draw[->, line width=2pt, -stealth](0,0,0)--(0.5,1,0);
    
\end{tikzpicture}
\end{center}
\end{problem}

\begin{problem}\label{prob:linindmultchoice9}
\begin{multipleChoice}
 \choice{The vectors are linearly independent}
  \choice[correct]{The vectors are linearly dependent }
  \choice{There is not enough information given to make a determination }
 \end{multipleChoice}
\begin{center}
\begin{tikzpicture}[scale=0.8]
  \draw[<->] (-2,0)--(2,0);
  \draw[<->] (0,-2)--(0,2);
  
  \draw[line width=2pt,blue,-stealth](0,0)--(1.5,1.5);
  \draw[line width=2pt,-stealth](0,0)--(1,-1);
  \draw[line width=2pt,red,-stealth](0,0)--(0,1.5);

 \end{tikzpicture}
\end{center}
\end{problem}

\begin{problem}\label{prob:linindmultchoice10}
\begin{multipleChoice}
 \choice[correct]{The vectors are linearly independent}
  \choice{The vectors are linearly dependent }
  \choice{There is not enough information given to make a determination }
 \end{multipleChoice}
\begin{center}
\begin{tikzpicture}[scale=0.8]
  \draw[<->] (-2.5,0)--(2.5,0);
  \draw[<->] (0,-1)--(0,2.5);
  
  \draw[line width=2pt,blue,-stealth](0,0)--(2,2);
  \draw[line width=2pt,red,-stealth](0,0)--(-1,1);

 \end{tikzpicture}
\end{center}
\end{problem}


\begin{problem}\label{prob:linindmultchoice11}
\begin{multipleChoice}
 \choice{The vectors are linearly independent}
  \choice[correct]{The vectors are linearly dependent }
  \choice{There is not enough information given to make a determination }
 \end{multipleChoice}
\begin{center}
\tdplotsetmaincoords{70}{130}
\begin{tikzpicture}
	\draw[->](-2,0,0)--(2,0,0) node[below left]{$y$};
    \draw[->](0,-2,0)--(0,2,0) node[below left]{$z$};
    \draw[->](0,0,-2)--(0,0,2) node[below left]{$x$};
    \draw[->, line width=2pt,blue, -stealth](0,0,0)--(0,-1,0);
    \draw[->, line width=2pt,red, -stealth](0,0,0)--(1,0,0);
    \draw[->, line width=2pt,orange, -stealth](0,0,0)--(0,1,0);
    
\end{tikzpicture}
\end{center}
\end{problem}

\end{problem}

\begin{problem}\label{prob:Adding1OK}
If $\{\vec{v}_{1}, \dots , \vec{v}_{m}\}$ is a linearly independent set in $\RR^n$, and if $\vec{v}_{m+1}$ is not in $\mbox{span}\left(\vec{v}_{1}, \dots , \vec{v}_{m}\right)$, then $\{\vec{v}_{1}, \dots , \vec{v}_{m}, \vec{v}_{m+1}\}$ is also linearly independent.
\end{problem}
\end{document} 