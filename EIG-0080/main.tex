\documentclass{ximera}
%% You can put user macros here
%% However, you cannot make new environments

\listfiles

\graphicspath{
{./}
{./LTR-0070/}
{./VEC-0060/}
{./APP-0020/}
}

\usepackage{tikz}
\usepackage{tkz-euclide}
\usepackage{tikz-3dplot}
\usepackage{tikz-cd}
\usetikzlibrary{shapes.geometric}
\usetikzlibrary{arrows}
%\usetkzobj{all}
\pgfplotsset{compat=1.13} % prevents compile error.

%\renewcommand{\vec}[1]{\mathbf{#1}}
\renewcommand{\vec}{\mathbf}
\newcommand{\RR}{\mathbb{R}}
\newcommand{\dfn}{\textit}
\newcommand{\dotp}{\cdot}
\newcommand{\id}{\text{id}}
\newcommand\norm[1]{\left\lVert#1\right\rVert}
 
\newtheorem{general}{Generalization}
\newtheorem{initprob}{Exploration Problem}

\tikzstyle geometryDiagrams=[ultra thick,color=blue!50!black]

%\DefineVerbatimEnvironment{octave}{Verbatim}{numbers=left,frame=lines,label=Octave,labelposition=topline}



\usepackage{mathtools}


\title{Gershgorin's Theorem} \license{CC BY-NC-SA 4.0}

\begin{document}

\begin{abstract}
\end{abstract}
\maketitle

\section*{Gershgorin's Theorem}
We have seen that eigenvalues are the roots of the characteristic polynomial, and therefore may be complex numbers, even when a matrix has entries that are real.  This section is devoted to a remarkable theorem proved in 1931 called Gersgorin's theorem, which says that the $n$ eigenvalues of an $n \times n$ matrix can be found in a region in the complex plane consisting of $n$ disks.  We explore the theorem for the case $n=3$ before proving the result.  


\begin{exploration}\label{init:GershDisks3x3}


\begin{center}
\geogebra{a5jpgbnp}{800}{600}
\end{center}

\end{exploration}

\end{document}