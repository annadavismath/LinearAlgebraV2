\documentclass{ximera}
%% You can put user macros here
%% However, you cannot make new environments

\listfiles

\graphicspath{
{./}
{./LTR-0070/}
{./VEC-0060/}
{./APP-0020/}
}

\usepackage{tikz}
\usepackage{tkz-euclide}
\usepackage{tikz-3dplot}
\usepackage{tikz-cd}
\usetikzlibrary{shapes.geometric}
\usetikzlibrary{arrows}
%\usetkzobj{all}
\pgfplotsset{compat=1.13} % prevents compile error.

%\renewcommand{\vec}[1]{\mathbf{#1}}
\renewcommand{\vec}{\mathbf}
\newcommand{\RR}{\mathbb{R}}
\newcommand{\dfn}{\textit}
\newcommand{\dotp}{\cdot}
\newcommand{\id}{\text{id}}
\newcommand\norm[1]{\left\lVert#1\right\rVert}
 
\newtheorem{general}{Generalization}
\newtheorem{initprob}{Exploration Problem}

\tikzstyle geometryDiagrams=[ultra thick,color=blue!50!black]

%\DefineVerbatimEnvironment{octave}{Verbatim}{numbers=left,frame=lines,label=Octave,labelposition=topline}



\usepackage{mathtools}


\author{Paul Zachlin \and Anna Davis} \title{Orthogonal and Orthonormal Sets of Vectors} \license{CC-BY 4.0}

\begin{document}
\begin{abstract}

\end{abstract}
\maketitle
\subsection{Orthogonal and Orthonormal Sets}

In this section, we examine what it means for vectors (and sets of
vectors) to be orthogonal and orthonormal. First, it is necessary to
review some important concepts. You may recall the definitions for the
span of a set of vectors and a linear independent set of vectors. We
include the definitions and examples here for convenience.

\begin{definition}\label{def:spanReprise} Let ${\bf v}_1, {\bf v}_2,\ldots ,{\bf v}_p$ be vectors in $\RR^n$.  The set $S$ of all linear combinations of ${\bf v}_1, {\bf v}_2,\ldots ,{\bf v}_p$ is called the \dfn{span} of ${\bf v}_1, {\bf v}_2,\ldots ,{\bf v}_p$.  We write 
$$S=\mbox{span}({\bf v}_1, {\bf v}_2,\ldots ,{\bf v}_p)$$
and we say that vectors ${\bf v}_1, {\bf v}_2,\ldots ,{\bf v}_p$ \dfn{span} $S$.  Any vector in $S$ is said to be \dfn{in the span} of ${\bf v}_1, {\bf v}_2,\ldots ,{\bf v}_p$.  The set $\{{\bf v}_1, {\bf v}_2,\ldots ,{\bf v}_p\}$ is called a \dfn{spanning set} for $S$.
\end{definition}

Consider the following example.

\begin{example}\label{spanvectors}
Describe the span of the vectors ${\bf u}=\begin{bmatrix}
1  \\ 1 \\ 0
\end{bmatrix}$ and
${\bf v}=\begin{bmatrix}
3  \\ 2 \\ 0
\end{bmatrix} \in \mathbb{R}^{3}$.
\end{example}

\begin{explanation}
It is clear that any linear combination of the vectors ${\bf u}$ and ${\bf v}$ yields a vector
$\begin{bmatrix}
x  \\ y \\ 0
\end{bmatrix}$ in the $XY$-plane.

Moreover every vector in the $XY$-plane is in fact such a linear
combination of the vectors ${\bf u}$ and ${\bf v}$. To see this, note that
$$\begin{bmatrix}
x  \\ y \\ 0
\end{bmatrix}
=
(-2x+3y) \begin{bmatrix}
1  \\ 1 \\ 0
\end{bmatrix}
+
(x-y)\begin{bmatrix}
3  \\ 2 \\ 0
\end{bmatrix}
$$

Thus  span$\{{\bf u},{\bf v}\}$ is precisely the $XY$-plane.
\end{explanation}

Note that a collection of the form $\mbox{span}\left( \{{\bf u}_1, \cdots , {\bf u}_k\}\right)$
a subspace of $\mathbb{R}^{n}$, as by definition it is closed under addition and scalar multiplication.

Recall that another important property of sets of vectors is linear independence.

\begin{definition}[Linear Independence]\label{def:linearindependenceReprise}
Let ${\bf v}_1, {\bf v}_2,\ldots ,{\bf v}_p$ be vectors of $\RR^n$.  We say that the set $\{{\bf v}_1, {\bf v}_2,\ldots ,{\bf v}_p\}$ is \dfn{linearly independent} if the only solution to 
$$a_1{\bf v}_1+a_2{\bf v}_2+\ldots +a_p{\bf v}_p={\bf 0}$$
is the \dfn{trivial solution} $a_1=a_2=\ldots =a_p=0$.

If, in addition to the trivial solution, a \dfn{non-trivial solution} (not all $a_1, a_2,\ldots ,a_p$ are zero) exists, then we say that the set $\{{\bf v}_1, {\bf v}_2,\ldots ,{\bf v}_p\}$ is \dfn{linearly dependent}.
\end{definition}

\begin{example}\label{linearlyindependentvectors}
Consider the vectors ${\bf u}=\begin{bmatrix}
1  \\ 1 \\ 0
\end{bmatrix}$, 
${\bf v}=\begin{bmatrix}
3  \\ 2 \\ 0
\end{bmatrix}$, and
${\bf w}=\begin{bmatrix}
4  \\ 5 \\ 0
\end{bmatrix} \in \mathbb{R}^{3}$.
Determine if the set $\{{\bf u}, {\bf v}, {\bf w}\}$ is linearly independent.
\end{example}

\begin{explanation}
We already verified in Example \ref{spanvectors} that
$\mbox{span}\left( \{{\bf u}, {\bf v} \}\right)$ is the $XY$-plane. Since ${\bf w}$
is clearly also in the $XY$-plane, then the set $\{{\bf u}, {\bf v},
{\bf w}\}$ is \textbf{not} linearly independent.
\end{explanation}

Recall that if a subspace is spanned by a linearly independent set of vectors, then we say that those vectors form a \dfn{basis} for the subspace.  Thus the set of vectors $\{{\bf u}, {\bf v}\}$ from Example
\ref{linearlyindependentvectors} is a basis for the $XY$-plane in
$\mathbb{R}^{3}$ since it is both linearly independent and spans
the $XY$-plane.

\begin{example}\label{OrthoVectoSpanningSet}
Let $\{{\bf x}_1, {\bf x}_2, \ldots, {\bf x}_k\}\in\mathbb{R}^n$ and
suppose $\mathbb{R}^n=\mbox{span}\left(\{{\bf x}_1, {\bf x}_2, \ldots, {\bf x}_k\}\right)$.
Furthermore, suppose that there exists a vector ${\bf u}\in\mathbb{R}^n$ for which ${\bf u}\dotp {\bf x}_j=0$ for all $j$, $1\leq j\leq k$.
What can be said about the vector ${\bf u}$?
\end{example}

\begin{explanation}
Write ${\bf u}=t_1{\bf x}_1 + t_2{\bf x}_2 +\cdots +t_k{\bf x}_k$
for some $t_1, t_2, \ldots, t_k\in\mathbb{R}$
(this is possible because
${\bf x}_1, {\bf x}_2, \ldots, {\bf x}_k$ span $\mathbb{R}^n$).

Then
\begin{eqnarray*}
\norm{{\bf u}}^2 & = & {\bf u}\dotp{\bf u} \\
& = & {\bf u}\dotp(t_1{\bf x}_1 + t_2{\bf x}_2 +\cdots +t_k{\bf x}_k) \\
& = & {\bf u}\dotp (t_1{\bf x}_1) +  {\bf u}\dotp (t_2{\bf x}_2) +
\cdots +  {\bf u}\dotp (t_k{\bf x}_k) \\
& = & t_1({\bf u}\dotp {\bf x}_1) + t_2({\bf u}\dotp {\bf x}_2) + \cdots
+ t_k({\bf u}\dotp {\bf x}_k) \\
& = & t_1(0) + t_2(0) + \cdots + t_k(0) = 0.
\end{eqnarray*}
Since $\norm{{\bf u}}^2 =0$, $\norm{{\bf u}} =0$.
We know that $\norm{{\bf u}}=0$ if and only if
${\bf u}={\bf 0}$.
Therefore, ${\bf u}={\bf 0}$.
In conclusion, the only vector orthogonal to every vector of
a spanning set of $\mathbb{R}^n$ is the zero vector.
\end{explanation}

We can now discuss what is meant by an \dfn{orthogonal set of vectors}. 

\begin{definition}\label{orthset}
Let $\{ {\bf u}_1, {\bf u}_2, \cdots, {\bf u}_m \}$ be a set of
vectors in $\mathbb{R}^n$. Then this set is called an
\dfn{orthogonal set} if the following conditions hold:
\begin{enumerate}
\item
${\bf u}_i \dotp {\bf u}_j = 0$ for all $i \neq j$
\item
${\bf u}_i \neq {\bf 0}$ for all $i$
\end{enumerate}
\end{definition}

If we have an orthogonal set of vectors and normalize each vector so
they have length 1, the resulting set is called an \dfn{orthonormal
set} of vectors. They can be described as follows.

\begin{definition}\label{orthoSetVectors}
A set of vectors, $\left\{ {\bf w}_{1},\cdots ,{\bf w}_{m}\right\} $
is said to be an
\dfn{orthonormal} set if
\[
{\bf w}_i \dotp {\bf w}_j = \delta _{ij} = \left\{
\begin{array}{c}
1\text{ if }i=j \\
0\text{ if }i\neq j
\end{array}
\right.
\]
\end{definition}

Note that all orthonormal sets are orthogonal, but the converse is not
necessarily true since the vectors may not be normalized. In order to
normalize the vectors, we simply need divide each one by its length.

\begin{definition}\label{normalizing}
\dfn{Normalizing} an orthogonal set is the process of turning an orthogonal set into an orthonormal set.
If $\{ {\bf u}_1, {\bf u}_2, \ldots, {\bf u}_k\}$
is an orthogonal subset of $\mathbb{R}^n$,
then
\[ \left\{
\frac{1}{\norm{{\bf u}_1}}{\bf u}_1,
\frac{1}{\norm{{\bf u}_2}}{\bf u}_2, \ldots,
\frac{1}{\norm{{\bf u}_k}}{\bf u}_k \right\}
\]
is an orthonormal set.
\end{definition}

We illustrate this concept in the following example.

\begin{example}\label{ex:orthonormalset}
Consider the set of vectors  given by
\[
\left\{ {\bf u}_1, {\bf u}_2 \right\} = \left\{
\begin{bmatrix}
1 \\
1
\end{bmatrix},
\begin{bmatrix}
-1 \\
1
\end{bmatrix}
\right\}
\]
Show that this is an orthogonal set of vectors  but not an orthonormal one. Find the corresponding orthonormal set.
\end{example}

\begin{explanation}
One easily verifies that ${\bf u}_1 \dotp {\bf u}_2 = 0$ and
$\left\{ {\bf u}_1, {\bf u}_2 \right\}$ is an orthogonal set of
vectors. On the other hand one can compute that ${\norm{{\bf u}_1}}= {\norm{{\bf u}_2}} =
\sqrt{2} \neq 1$ and so the set is not orthonormal.

Thus to find a corresponding orthonormal set, we simply need to
normalize each vector. We will write $\{ {\bf w}_1, {\bf w}_2 \}$
for the corresponding orthonormal set. Then,
\begin{eqnarray*}
{\bf w}_1 &=& \frac{1}{\norm{{\bf u}_1}}{\bf u}_1\\
&=& \frac{1}{\sqrt{2}} \begin{bmatrix}
1 \\
1
\end{bmatrix} \\
&=&
\begin{bmatrix}
\frac{1}{\sqrt{2}} \\
\frac{1}{\sqrt{2}}
\end{bmatrix}
\end{eqnarray*}

Similarly,
\begin{eqnarray*}
{\bf w}_2 &=& \frac{1}{\norm{{\bf u}_2}}{\bf u}_2\\
&=& \frac{1}{\sqrt{2}} \begin{bmatrix}
-1 \\
1
\end{bmatrix} \\
&=&
\begin{bmatrix}
-\frac{1}{\sqrt{2}} \\
\frac{1}{\sqrt{2}}
\end{bmatrix}
\end{eqnarray*}

Therefore the corresponding orthonormal set is
\[
\left\{ {\bf w}_1, {\bf w}_2 \right\} =
\left\{
\begin{bmatrix}
\frac{1}{\sqrt{2}} \\
\frac{1}{\sqrt{2}}
\end{bmatrix},
\begin{bmatrix}
-\frac{1}{\sqrt{2}} \\
\frac{1}{\sqrt{2}}
\end{bmatrix}
\right\}
\]

You can verify that this set is orthonormal.
\end{explanation}

Consider an orthogonal set of vectors in $\RR^n$, written $\{
{\bf w}_1, \cdots, {\bf w}_k \}$ with $k \leq n$. The span of these
vectors is a subspace $W$ of $\RR^n$. If we
could show that this orthogonal set is also linearly independent, we
would have a basis of $W$. This is precisely what we do in the next theorem.

\begin{theorem}[Orthogonal Basis of a Subspace]\label{orthbasis}
Let $ \{ {\bf w}_1, {\bf w}_2, \cdots, {\bf w}_k \}$ be an
orthonormal set of vectors in $\mathbb{R}^n$. Then this set is
linearly independent and forms a basis for the subspace $W =
\mbox{span}\left( \{ {\bf w}_1, {\bf w}_2, \cdots, {\bf w}_k \}\right)$.
\end{theorem}

\begin{proof}
To show this set is linearly independent, suppose a linear combination
of these vectors equals ${\bf 0}$, such as:
\[
a_1 {\bf w}_1 + a_2 {\bf w}_2 + \cdots + a_k {\bf w}_k = {\bf 0}, a_i \in \RR
\]
We need to show that all $a_i = 0$. To do so, we take the dot product of
each side of the above equation with the vector ${\bf w}_i$ and obtain the following.

\begin{eqnarray*}
{\bf w}_i \dotp (a_1 {\bf w}_1 + a_2 {\bf w}_2 + \cdots + a_k {\bf w}_k ) &=& {\bf w}_i \dotp {\bf 0}\\
a_1 ({\bf w}_i \dotp {\bf w}_1) + a_2 ({\bf w}_i \dotp {\bf w}_2) + \cdots + a_k ({\bf w}_i \dotp {\bf w}_k)  &=& 0
\end{eqnarray*}
Now since the set is orthogonal, ${\bf w}_i \dotp {\bf w}_m = 0$ for all $m \neq i$, so we have:
\[
a_1 (0) + \cdots + a_i({\bf w}_i \dotp {\bf w}_i) + \cdots + a_k (0) = 0
\]
\[
a_i \norm{{\bf w}_i}^2 = 0
\]

Since the set is orthogonal, we know that $\norm{{\bf w}_i}^2  \neq 0$. It follows that $a_i =0$. Since the $a_i$ was chosen arbitrarily, the set $\{ {\bf w}_1, {\bf w}_2, \cdots, {\bf w}_k \}$ is linearly independent.

Finally since $W = \mbox{span} \{ {\bf w}_1, {\bf w}_2, \cdots,
{\bf w}_k \}$, the set of vectors also spans $W$ and therefore forms a basis of $W$.

\end{proof}

If an orthogonal set is a basis for a subspace, we call it an
\dfn{orthogonal basis}. Similarly, if an orthonormal set is a basis, we call it an \dfn{orthonormal basis}.

We conclude this section with a discussion of Fourier expansions. Given any orthogonal basis $B$ of $\RR^n$ and an arbitrary vector ${\bf x} \in \RR^n$, how do we express ${\bf x}$ as a linear combination of vectors in $B$? The solution is Fourier expansion.

\begin{theorem}[Fourier Expansion]\label{fourierexpansion}
Let $V$ be a subspace of $\mathbb{R}^n$ and suppose $\{ {\bf u}_1, {\bf u}_2, \ldots, {\bf u}_m \}$
is an orthogonal basis of $V$.
Then for any ${\bf x}\in V$,
\[ {\bf x} =
\left(\frac{{\bf x}\dotp {\bf u}_1}{\norm{{\bf u}_1}^2}\right) {\bf u}_1 +
\left(\frac{{\bf x}\dotp {\bf u}_2}{\norm{{\bf u}_2}^2}\right) {\bf u}_2 +
\cdots +
\left(\frac{{\bf x}\dotp {\bf u}_m}{\norm{{\bf u}_m}^2}\right) {\bf u}_m.
\]
This expression is called the Fourier expansion
of ${\bf x}$, and
\[ \frac{{\bf x}\dotp {\bf u}_j}{\norm{{\bf u}_j}^2},\]
$j=1,2,\ldots,m$
are the Fourier coefficients.
\end{theorem}

Consider the following example.

\begin{example}\label{fourier}
Let
${\bf u}_1= \begin{bmatrix}
1 \\ -1 \\ 2
\end{bmatrix},
{\bf u}_2= \begin{bmatrix}
0 \\ 2 \\ 1 
\end{bmatrix},
{\bf u}_3 =\begin{bmatrix}
5 \\ 1 \\ -2
\end{bmatrix}$,
and let
${\bf x} =\begin{bmatrix}
1 \\ 1 \\ 1
\end{bmatrix}$.  

Then $B=\{ {\bf u}_1, {\bf u}_2, {\bf u}_3\}$
is an orthogonal basis of $\mathbb{R}^3$.

Use this fact to write ${\bf x}$ as  a linear combination of the vectors of $B$.
\end{example}

\begin{explanation}
Since $B$ is a basis (verify!) there is a unique way to express ${\bf x}$ as a
linear combination of vectors of $B$. Moreover, since $B$ is an
orthogonal basis (verify!), then this can be done by computing the
Fourier expansion of ${\bf x}$.

That is:

\[
{\bf x}   =
\left(\frac{{\bf x}\dotp {\bf u}_1}{\norm{{\bf u}_1}^2}\right) {\bf u}_1 +
\left(\frac{{\bf x}\dotp {\bf u}_2}{\norm{{\bf u}_2}^2}\right) {\bf u}_2 +
\left(\frac{{\bf x}\dotp {\bf u}_3}{\norm{{\bf u}_3}^2}\right) {\bf u}_3.
\]

We readily compute:

\[
\frac{{\bf x}\dotp{\bf u}_1}{\norm{{\bf u}_1}^2} = \frac{2}{6}, \;
\frac{{\bf x}\dotp{\bf u}_2}{\norm{{\bf u}_2}^2} = \frac{3}{5},
\mbox{ and }
\frac{{\bf x}\dotp{\bf u}_3}{\norm{{\bf u}_3}^2} = \frac{4}{30}.\]

Therefore,
\[ \begin{bmatrix}
1 \\ 1 \\ 1
\end{bmatrix}
= \frac{1}{3}\begin{bmatrix}
1 \\ -1 \\ 2
\end{bmatrix}
+\frac{3}{5}\begin{bmatrix}
0 \\ 2 \\ 1
\end{bmatrix}
+\frac{2}{15}\begin{bmatrix}
5 \\ 1 \\ -2
\end{bmatrix}.\]
\end{explanation} 

\section*{Practice Problems}
\begin{problem}\label{prob:diagonalizematrix1}
Verify that the set $B=\{ {\bf u}_1, {\bf u}_2, {\bf u}_3\}$
of Example~\ref{fourier} is, in fact, a basis of $\mathbb{R}^3$.

\end{problem}
  
\section*{Text Source}
A large portion of the text in this module is an adaptation of Section 4.11.1 of Ken Kuttler's \href{https://open.umn.edu/opentextbooks/textbooks/a-first-course-in-linear-algebra-2017}{\it A First Course in Linear Algebra}. (CC-BY)

Ken Kuttler, {\it  A First Course in Linear Algebra}, Lyryx 2017, Open Edition, p. 233-238.  

\end{document}
