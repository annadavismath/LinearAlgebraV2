\documentclass{ximera}
%% You can put user macros here
%% However, you cannot make new environments

\listfiles

\graphicspath{
{./}
{./LTR-0070/}
{./VEC-0060/}
{./APP-0020/}
}

\usepackage{tikz}
\usepackage{tkz-euclide}
\usepackage{tikz-3dplot}
\usepackage{tikz-cd}
\usetikzlibrary{shapes.geometric}
\usetikzlibrary{arrows}
%\usetkzobj{all}
\pgfplotsset{compat=1.13} % prevents compile error.

%\renewcommand{\vec}[1]{\mathbf{#1}}
\renewcommand{\vec}{\mathbf}
\newcommand{\RR}{\mathbb{R}}
\newcommand{\dfn}{\textit}
\newcommand{\dotp}{\cdot}
\newcommand{\id}{\text{id}}
\newcommand\norm[1]{\left\lVert#1\right\rVert}
 
\newtheorem{general}{Generalization}
\newtheorem{initprob}{Exploration Problem}

\tikzstyle geometryDiagrams=[ultra thick,color=blue!50!black]

%\DefineVerbatimEnvironment{octave}{Verbatim}{numbers=left,frame=lines,label=Octave,labelposition=topline}



\usepackage{mathtools}


\title{Least-Squares Approximation} \license{CC BY-NC-SA 4.0}

\begin{document}

\begin{abstract}
\end{abstract}
\maketitle


\begin{center}
\begin{tikzpicture}[scale=1]
  \filldraw[orange](-0.25,3.5)--(0.25,3.5)--(1.5,0)--(-1.5,0)--cycle;
  \filldraw[orange] (0,0) ellipse (2cm and 1cm);
  \filldraw[orange] (0,3.5) ellipse (0.25cm and 0.15cm);
\end{tikzpicture}
 
UNDER CONSTRUCTION -- COMING SOON
\end{center}


This is from Nicholson's section on QR, and may be helpful:

In Section~\ref{sec:5_6} we found how to find a best approximation $\vec{z}$ to a solution of a (possibly inconsistent) system $A\vec{x} = \vec{b}$ of linear equations: take $\vec{z}$ to be any solution of the ``normal'' equations $(A^{T}A)\vec{z} = A^{T}\vec{b}$. If $A$ has independent columns this $\vec{z}$ is unique ($A^{T}A$ is invertible by Theorem~\ref{thm:015672}), so it is often desirable to compute $(A^{T}A)^{-1}$. This is particularly useful in least squares approximation (Section~\ref{sec:5_6}). This is simplified if we have a QR-factorization of $A$ (and is one of the main reasons for the importance of Theorem~\ref{th:QR-025133}). For if $A = QR$ is such a factorization, then $Q^{T}Q = I_{n}$ because $Q$ has orthonormal columns (verify), so we obtain
\begin{equation*}
A^TA = R^TQ^TQR = R^TR
\end{equation*}
Hence computing $(A^{T}A)^{-1}$ amounts to finding $R^{-1}$, and this is a routine matter because $R$ is upper triangular. Thus the difficulty in computing $(A^{T}A)^{-1}$ lies in obtaining the QR-factorization of $A$.

\end{document}