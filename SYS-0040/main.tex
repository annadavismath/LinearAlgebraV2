\documentclass{ximera}
%% You can put user macros here
%% However, you cannot make new environments

\listfiles

\graphicspath{
{./}
{./LTR-0070/}
{./VEC-0060/}
{./APP-0020/}
}

\usepackage{tikz}
\usepackage{tkz-euclide}
\usepackage{tikz-3dplot}
\usepackage{tikz-cd}
\usetikzlibrary{shapes.geometric}
\usetikzlibrary{arrows}
%\usetkzobj{all}
\pgfplotsset{compat=1.13} % prevents compile error.

%\renewcommand{\vec}[1]{\mathbf{#1}}
\renewcommand{\vec}{\mathbf}
\newcommand{\RR}{\mathbb{R}}
\newcommand{\dfn}{\textit}
\newcommand{\dotp}{\cdot}
\newcommand{\id}{\text{id}}
\newcommand\norm[1]{\left\lVert#1\right\rVert}
 
\newtheorem{general}{Generalization}
\newtheorem{initprob}{Exploration Problem}

\tikzstyle geometryDiagrams=[ultra thick,color=blue!50!black]

%\DefineVerbatimEnvironment{octave}{Verbatim}{numbers=left,frame=lines,label=Octave,labelposition=topline}



\usepackage{mathtools}


\title{Iterative Methods for Solving Linear Systems} \license{CC BY-NC-SA 4.0}

\begin{document}
\begin{abstract}

\end{abstract}
\maketitle
\section*{Iterative Methods for Solving Linear Systems}

The approaches to solving linear systems we have studied up to this point are often called \dfn{direct methods}, to set them apart from \dfn{iterative methods}, which we study in this section.  When we solve linear systems using \dfn{iterative methods}, we repeat the same procedure (called an \dfn{iteration} many times (usually using a computer), and we look for the outputs to ``converge to'' the actual solution.  Let's begin with an exploration to depict this idea.

\begin{exploration}\label{exp:gauss-seidel2x2}
In the following GeoGebra interactive, the red line is the graph of $ax+by=r$, and the blue line is the graph of $cx+dy=s$.  Our initial guess, point $A$, may be dragged anywhere on the plane before starting the iterative procedure, it is referred to as our ``initial guess''.  Then use the slide-flipping bar on the bottom to proceed through a number of iterations.  One iteration will consist of sliding horizontally from the current point until we hit the red line, followed by sliding vertically until we hit the blue line.  These two steps are then repeated from the new point on the blue line.
    
    \begin{enumerate}
        \item Try solving the system of equations
\begin{equation*}\label{eq:diagdom1}
\begin{array}{ccccc}
      5x& -&2y&=&10\\
      x & +&3y&= &36 
    \end{array}
\end{equation*}
from various starting points $A$.  Does the iterative method ``converge'' to the actual solution of this system?
        \item Now try solving the same system of equations, but changing which equation is the red line and which equation is the blue line.  In other words, try
\begin{equation*}\label{eq:diagdom2}
\begin{array}{ccccc}
      x & +&3y&= &36  \\
     5x& -&2y&=&10
    \end{array}
\end{equation*}
from various starting points $A$.  Now what happens when we try the iterative method?
\item Try other $2 \times 2$ systems of equations and see how the iterative method works.
    \end{enumerate}

    \begin{center}
\geogebra{hndq9nmq}{800}{650}
\end{center}
\end{exploration}

\subsection*{Jacobi's method}
As you saw in the Exploration above, iterative methods do not give an exact answer, but an approximate answer.  More iterations give a more accurate answer.

In \dfn{Jacobi's method} for solving a linear system of $n$ equations in $n$ variables, we isolate the first variable in the first equation, we isolate the second variable in the second equation, and in general, we isolate the $i$th variable in the $i$th equation.

Let's return to the system of equations from Exploration \ref{exp:gauss-seidel2x2}.  To use Jacobi's method to approximate the solution to the system, we solve for a different variable in each of the two equations.  This gives us two formulas, one for each unknown.  We begin with our initial guess $\vec{v}_0$, and use these formulas to compute the next iteration $\vec{v}_1$.  Taking the vector $\vec{v}_1$, and we again apply these formulas to compute the next iteration to arrive at $\vec{v}_2$.


\begin{equation*}\begin{array}{ccccc}
      5x& -&2y&=&10\\
      x & +&3y&= &36 
    \end{array}
 \rightarrow\text{ becomes } \rightarrow
\begin{array}{ccc}
      x& =&\dfrac{10+2y}{5}\\
      y& =&\dfrac{36-x}{3}
    \end{array}
\end{equation*}

Let's begin with the initial guess $\vec{v}_0 = \begin{bmatrix} 0\\0 \end{bmatrix}$.  Applying the pair of formulas gives
$$\vec{v}_1 = \begin{bmatrix} x\\y \end{bmatrix} = \begin{bmatrix} \dfrac{10+2(0)}{5}\\\dfrac{36-0}{3} \end{bmatrix}
= \begin{bmatrix} 2\\12 \end{bmatrix}.$$
For the next iteration, we apply these same formulas to the $x$ and $y$ values of $\vec{v}_1$.  We get
$$\vec{v}_2 = \begin{bmatrix} x\\y \end{bmatrix} = \begin{bmatrix} \dfrac{10+2(12)}{5}\\\dfrac{36-2}{3} \end{bmatrix}
= \begin{bmatrix} 34/5\\34/3 \end{bmatrix}\approx
\begin{bmatrix} 6.80\\11.33 \end{bmatrix}.$$
We use this to perform another iteration, giving us 
$$\vec{v}_3 = \begin{bmatrix} x\\y \end{bmatrix} = \begin{bmatrix} \dfrac{10+2(11.33)}{5}\\\dfrac{36-6.80}{3} \end{bmatrix}
= \begin{bmatrix} 34/5\\38/3 \end{bmatrix}\approx
\begin{bmatrix} 6.53\\9.93 \end{bmatrix}.$$
We observe that at each iteration our approximate solution gets closer to the actual solution to the system, $x=6, y=10$.

\subsection*{The Gauss-Seidel method}

The \dfn{Gauss-Seidel method} for solving a linear system of $n$ equations in $n$ variables, we use the same set of $n$ formulas as we use in Jacobi's method.  The difference is that we implement the formulas \emph{sequentially} rather than simultaneously.

To demonstrate, we return to the same system of equations examined above.

\begin{equation*}\begin{array}{ccccc}
      5x& -&2y&=&10\\
      x & +&3y&= &36 
    \end{array}
 \rightarrow\text{ becomes } \rightarrow
      x=\dfrac{10+2y}{5},
      y=\dfrac{36-x}{3}
\end{equation*}

Once again we start with the initial guess $\vec{v}_0 = \begin{bmatrix} 0\\0 \end{bmatrix}$.  We apply the first formula and get
$$x= \dfrac{10+2(0)}{5}=2.$$
However, we will compute $y$ using this new value $x=2$ rather than using $x=0$.  We get
$$y=\dfrac{36-2}{3} 
= 34/3 \approx 11.33.$$  So after our first iteration we have $\vec{v}_1 = \begin{bmatrix} 2\\11.33 \end{bmatrix}$.
We repeat this procedure to perform another iteration.  We get 
$$x= \dfrac{10+2(11.33)}{5}\approx 6.53$$
followed by
$$y=\dfrac{36-6.53}{3} \approx 9.82,$$
so we have $\vec{v}_2 \approx \begin{bmatrix} 6.53\\9.82 \end{bmatrix}$.

We observe that in this example the second iteration of the Gauss-Seidel method is closer to the actual solution of $x=6, y=10$ than the third iteration of Jacobi's method.  The Gauss-Seidel method is more efficient than Jacobi's method because it makes use of a better approximation of one coordinate to find another.

\begin{exploration}\label{exp:jacobi&gauss-seidel2x2spreadsheet}
The Jacobi and Gauss-Seidel methods are easily implemented on a computer.  Below is a link to a spreadsheet which can be used to solve any linear system of 2 equations in 2 unknowns using each of these methods.

To use the spreadsheet, SAVE A COPY of the sheet.  Then you will be able to change the yellow boxes to input a system of equations as well as the initial values.  The video below the link to the spreadsheet gives additional information.
    
\href{https://docs.google.com/spreadsheets/d/1_QM-yfkcIHnBi71AXPul3c87erR3wdlgaUlm1SYoHAk/edit?usp=sharing}{Link to Spreadsheet}

\youtube{https://youtu.be/wKEjUZ1fzaw}
\end{exploration}

Both techniques described above can be used to solve linear systems with more variables.

\begin{example}
Use direct methods to solve the linear system.  Then try using Jacobi's Method and the Gauss-Seidel method.
$$\begin{array}{ccccccc}\label{eq:3x3iterative}
      5x & +&y&+&2z&= &1 \\
	 2x& +&5y&+&z&=&1\\
     2x& +&y&+&5z&=&1
    \end{array}$$

\begin{explanation}
We perform Gauss-Jordan elimination by applying elementary row operations as described in \href{https://ximera.osu.edu/oerlinalg/LinearAlgebra/SYS-0030/main}{Gaussian Elimination and Rank}.  
\begin{equation*}
\left[\begin{array}{ccc|c}  5&1&2&1\\2&5&1&1\\2&1&5&1
 \end{array}\right]
 \rightarrow\text{ becomes (after a number of steps) } \rightarrow
\left[\begin{array}{ccc|c}  1&0&0&1/8\\0&1&0&1/8\\0&0&1&1/8
 \end{array}\right]
\end{equation*}
This shows that the three planes in $\RR^3$ share a common intersection point at $\left(\frac{1}{8},\frac{1}{8},\frac{1}{8} \right)$, which is easily verified by looking at the original equations, and can also be seen in the GeoGebra graph below (RIGHT-CLICK AND DRAG TO ROTATE).
\begin{center}
\geogebra{nuvt2rde}{800}{650}
\end{center}
Now we solve the same linear system using our iterative methods.

ANNA, you can see my work in the same Google Sheet a little lower.  Let's discuss Monday how to present the rest of this example.

\end{explanation}
\end{example}

\subsection{Convergence of Iterative Methods}

We observed in Exploration \ref{exp:gauss-seidel2x2} that the Gauss-Seidel method converged for the linear system \ref{eq:diagdom1}.  However, when we wrote the same linear system using \ref{eq:diagdom2}, the method failed to converge.  We now take time to address questions about convergence of these two iterative methods we studied.

\href{https://www.jstor.org/stable/2132758}{link to a paper where this fact is proven}

\end{document}
