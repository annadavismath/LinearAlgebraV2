\documentclass{ximera}
%% You can put user macros here
%% However, you cannot make new environments

\listfiles

\graphicspath{
{./}
{./LTR-0070/}
{./VEC-0060/}
{./APP-0020/}
}

\usepackage{tikz}
\usepackage{tkz-euclide}
\usepackage{tikz-3dplot}
\usepackage{tikz-cd}
\usetikzlibrary{shapes.geometric}
\usetikzlibrary{arrows}
%\usetkzobj{all}
\pgfplotsset{compat=1.13} % prevents compile error.

%\renewcommand{\vec}[1]{\mathbf{#1}}
\renewcommand{\vec}{\mathbf}
\newcommand{\RR}{\mathbb{R}}
\newcommand{\dfn}{\textit}
\newcommand{\dotp}{\cdot}
\newcommand{\id}{\text{id}}
\newcommand\norm[1]{\left\lVert#1\right\rVert}
 
\newtheorem{general}{Generalization}
\newtheorem{initprob}{Exploration Problem}

\tikzstyle geometryDiagrams=[ultra thick,color=blue!50!black]

%\DefineVerbatimEnvironment{octave}{Verbatim}{numbers=left,frame=lines,label=Octave,labelposition=topline}



\usepackage{mathtools}


\author{Anna Davis \and Paul Zachlin} \title{LTR-0027: Linear Transformations of Abstract Vector Spaces} \license{CC-BY 4.0}

\begin{document}
\begin{abstract}
We define linear transformation for vector spaces, and  establish that a linear transformation of a vector space is completely determined by its effect on a basis. 
\end{abstract}
\maketitle

\section*{LTR-0027: Linear Transformations of Abstract Vector Spaces}

Recall that a transformation $T:\mathbb{R}^n\rightarrow \mathbb{R}^m$ is called a \dfn{linear transformation} if the following are true for all vectors ${\bf u}$ and ${\bf v}$ in $\mathbb{R}^n$, and scalars $k$.
\begin{equation*}
T(k{\bf u})= kT({\bf u})
\end{equation*}
\begin{equation*}
T({\bf u}+{\bf v})= T({\bf u})+T({\bf v})
\end{equation*}

We generalize this definition as follows.

\begin{definition}\label{def:lintransgeneral}
Let $V$ and $W$ be vector spaces. A transformation $T:V\rightarrow W$ is called a \dfn{linear transformation} if the following are true for all vectors ${\bf u}$ and ${\bf v}$ in $V$, and scalars $k$.
\begin{equation*}
T(k{\bf u})= kT({\bf u})
\end{equation*}
\begin{equation*}
T({\bf u}+{\bf v})= T({\bf u})+T({\bf v})
\end{equation*}
\end{definition}

\begin{center}
\begin{tikzpicture}[scale=1]
  \filldraw[orange](-0.25,3.5)--(0.25,3.5)--(1.5,0)--(-1.5,0)--cycle;
  \filldraw[orange] (0,0) ellipse (2cm and 1cm);
  \filldraw[orange] (0,3.5) ellipse (0.25cm and 0.15cm);
\end{tikzpicture}

UNDER CONSTRUCTION -- COMING SOON
\end{center}



\end{document}
